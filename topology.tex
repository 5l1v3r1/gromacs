%
% 
%       This source code is part of
% 
%        G   R   O   M   A   C   S
% 
% GROningen MAchine for Chemical Simulations 
% 
%               VERSION 2.0
% 
% Copyright (c) 1991-1999
% BIOSON Research Institute, Dept. of Biophysical Chemistry
% University of Groningen, The Netherlands
% 
% Please refer to:
% GROMACS: A message-passing parallel molecular dynamics implementation
% H.J.C. Berendsen, D. van der Spoel and R. van Drunen
% Comp. Phys. Comm. 91, 43-56 (1995)
% 
% Also check out our WWW page:
% http://md.chem.rug.nl/~gmx
% or e-mail to:
% gromacs@chem.rug.nl
% 
% And Hey:
% Giving Russians Opium May Alter Current Situation
%

\chapter{Topologies}
\label{ch:top}
\section{Introduction}
{\gromacs} must know on which atoms and combinations of atoms the
various contributions to the potential functions (see
\chref{ff}) must act. It must
also know what \normindex{parameter}s must be applied to the various
functions. All this is described in the {\em \normindex{topology}} file
\verb'*.top', which lists the {\em constant attributes} of each atom.
There are many more atom types than elements, but only atom types
present in biological systems are parameterized in the force field,
plus some metals, ions and silicon. The bonded and special
interactions are determined by fixed lists that are included in the
topology file. Certain non-bonded interactions must be excluded (first
and second neighbors), as these are already treated in bonded
interactions.  In addition there are {\em dynamic attributes} of
atoms: their positions, velocities and forces, but these do not
strictly belong to the molecular topology.

This Chapter describes the set up of the topology file, the
{\tt *.top} file and the database files: what the parameters
stand for and how/where to change them if needed.
First all file formats are explained.
Section \ssecref{fffiles} describes the organization of
the force-field files.

{\bf Note:} if you construct your own topologies, we encourage you
to upload them to our topology archive at www.gromacs.org! Just imagine
how thankful you'd have been if your topology had been available
there before you started. The same goes for new force field or
modified versions of the standard force fields - contribute them
to the force field archive!

\section{Particle type}
\label{sec:parttype}

In {\gromacs} there are 5 types of \normindex{particle}s, see
\tabref{ptype}. Only regular atoms and virtual interaction-sites are used
in {\gromacs}; shells are necessary for
polarizable models like the Shell-Water models~\cite{Maaren2001a}.

\begin{table}
\centerline{
\begin{tabular}{|l|c|}
\dline
Particle        		& Symbol        \\
\hline
\seeindex{atom}{particle}s      & A   \\
\seeindex{shell}{particle}s     & S   \\
\normindex{virtual interaction-sites}	& V (or D)   \\
\dline
\end{tabular}
}
\caption{Particle types in {\gromacs}}
\label{tab:ptype}
\end{table}

\subsection{Atom types}
\label{subsec:atomtype}
Depending on the force field {\gromacs} uses different
\swapindex{atom}{type}s, a sample from the deprecated ``gromacs''
force field is listed below, with their corresponding masses (in
a.m.u.). This is the same listing as in the file {\tt ff???.atp} (.atp
= {\bf a}tom {\bf t}ype {\bf p}arameter file), therefore in this file
you can change and/or add an atom type.

\begin{tabular}{lll}
    O  &15.99940 &;     carbonyl oxygen (C=O)\\
   OM  &15.99940 &;     carboxyl oxygen (CO-)\\
   OA  &15.99940 &;     hydroxyl oxygen (OH)\\
   OW  &15.99940 &;     water oxygen\\
    N  &14.00670 &;     peptide nitrogen (N or NH)\\
   NT  &14.00670 &;     terminal nitrogen (NH2)\\
   NL  &14.00670 &;     terminal nitrogen (NH3)\\
  NR5  &14.00670 &;     aromatic N (5-ring,2 bonds)\\
 NR5*  &14.00670 &;     aromatic N (5-ring,3 bonds)\\
   NP  &14.00670 &;     porphyrin nitrogen\\
    C  &12.01100 &;     bare carbon (peptide,C=O,C-N)\\
  CH1  &13.01900 &;     aliphatic CH-group\\
  CH2  &14.02700 &;     aliphatic CH2-group\\
  CH3  &15.03500 &;     aliphatic CH3-group\\
\end{tabular}

Atomic detail is used except for hydrogen atoms bound to (aliphatic)
carbon atoms, which are treated as {\em \swapindex{united}{atom}s}. No
special \normindex{hydrogen-bond} term is included. Note that other force field
like OPLS/AA and Amber99 use all atoms.

For backward compatibility we retain here some reference to parameters
present in the ``gromacs'' force field. The last 10 atom types were
not part of the original \gromosv{87} force field~\cite{biomos} and
when you use them you can refer to one or more of the following
papers:
\begin{itemize}
\item F was taken from ref.~\cite{Buuren93a}, 
\item CP2 and CP3 from ref.~\cite{Buuren93b} and references cited therein, 
\item CR5, CR6 and HCR from ref.~\cite{Spoel96c}
\item OWT3 from ref.~\cite{Jorgensen83}
\item SD, OD and CD from ref.~\cite{Liu95}
\end{itemize}
{\bf Note that we recommend against using these parameters in new projects
since they are not well-tested.}

{\bf Note:} {\gromacs} makes use of the atom types as a name, {\em
not} as a number (as {\eg} in {\gromos}).

%\subsection{Nucleus}
%{\em Necessary for \normindex{polarisability}, not implemented yet.}
%
%\subsection{Shell}
%{\em Necessary for polarisability, not implemented yet.}
%
%\subsection{Bond shell}
%{\em Necessary for polarisability, not implemented yet.}

\subsection{Virtual sites}
\label{sec:vsitetop}
Some force fields use \normindex{virtual interaction-sites}
(interaction sites that are constructed from other particle postions)
on which certain interactions are located
({\eg} on benzene rings, to reproduce the correct
\normindex{quadrupole}). This is described in~\secref{virtual_sites}.

To make virtual sites in your system, you should include a section
{\tt [~virtual\_sites?~]} (for backward compatibility the old name
{\tt [~dummies?~]} can also be used) in your topology file,
where the `{\tt ?}' stands
for the number constructing particles for the virtual site. This will be
`{\tt 2}' for type 2, `{\tt 3}' for types 3, 3fd, 3fad and 3out and
`{\tt 4}' for type 4fd (the different types are explained
in~\secref{virtual_sites}).

Parameters for type 2 should look like this:\\
{\small\begin{tt}
[ virtual\_sites2 ] \\
; Site  from        funct  a \\
5       1     2     1      0.7439756\\
\end{tt}}

for type 3 like this:\\
{\small\begin{tt}
[ virtual\_sites3 ]\\
; Site  from               funct   a          b\\
5       1     2     3      1       0.7439756  0.128012\\
\end{tt}}

for type 3fd like this:\\
{\small\begin{tt}
[ virtual\_sites3 ]\\
; Site  from               funct   a          d\\
5       1     2     3      2       0.5        -0.105\\
\end{tt}}

for type 3fad like this:\\
{\small\begin{tt}
[ virtual\_sites3 ]\\
; Site  from               funct   theta      d\\
5       1     2     3      3       120        0.5\\
\end{tt}}

for type 3out like this:\\
{\small\begin{tt}
[ virtual\_sites3 ]\\
; Site  from               funct   a          b          c\\
5       1     2     3      4       -0.4       -0.4       6.9281\\
\end{tt}}

for type 4fd like this:\\
{\small\begin{tt}
[ virtual\_sites4 ]\\
; Site  from                      funct   a          b          d\\
5       1     2     3     4       1       0.33333    0.33333    -0.105\\
\end{tt}}

This will result in the construction of a virtual site, number 5
(first column `{\tt Site}'), based on the positions of 1 and 2 or 1,
2 and 3 or 1, 2, 3 and 4 (next two, three or four columns
`{\tt from}') following the rules determined by the function number
(next column `{\tt funct}') with the parameters specified (last one,
two or three columns `{\tt a b} . .').

Note that if any constant bonded interactions defined between
virtual sites and/or normal atoms will be removed by {\tt grompp},
this happens after the exclusions have been generated.
This way, exclusions will not be affected by an atom being
defined as virtual site or not, but by the bonding configuration of the
atom.

\section{Parameter files}
\label{sec:paramfiles}
\subsection{Atoms}
A number of {\em static} properties are assigned to the atom types in
the {\gromacs} force field: Type, Mass, Charge, $\epsilon$ and
$\sigma$ (see \tabref{statprop} The mass is listed in {\tt ff???.atp}
(see~\ssecref{atomtype}), whereas the charge is listed in {\tt
ff???.rtp} (.rtp = {\bf r}esidue {\bf t}opology {\bf p}arameter file,
see~\ssecref{rtp}).  This implies that the charges are only defined in
the \normindex{building block}s of amino acids or user defined
building blocks.  When generating a topology ({\tt *.top}) using the
{\tt \normindex{pdb2gmx}} program the information from these files is
combined.
 
\begin{table}
\centerline{
\begin{tabular}{|l|c|c|}
\dline
Property        & Symbol        & Unit          \\
\hline
Type            & -             & -             \\
Mass            & m             & a.m.u.        \\
Charge          & q             & electron      \\
epsilon         & $\epsilon$    & kJ/mol        \\
sigma           & $\sigma$      & nm            \\
\dline
\end{tabular}
}
\caption{Static atom type properties in {\gromacs}}
\label{tab:statprop}
\end{table}

The following {\em dynamic} quantities are associated with an atom
\begin{itemize}
\item   Position {\bf x}
\item   Velocity {\bf v}
\end{itemize}
These quantities are listed in the coordinate file, {\tt *.gro}
(see section File format,~\ssecref{grofile}).

\subsection{Bonded parameters}
\label{subsec:bondparam}
The \swapindex{bonded}{parameter}s ({\ie} bonds, bond angles, improper and proper
dihedrals) are listed in {\tt ff???bon.itp}. The term {\tt func} is 1 for
harmonic and 2 for \gromosv{96} bond and angle potentials.
For the dihedral, this is explained after this listing.

\begin{tt}
[ bondtypes ]\\
  ; i    j func        b0          kb\\
    C    O    1   0.12300     502080.\\
    C   OM    1   0.12500     418400.\\
    ......\\
\end{tt}

\begin{tt}
[ angletypes ] \\
  ; i    j    k func       th0         cth\\
   HO   OA    C    1   109.500     397.480\\
   HO   OA  CH1    1   109.500     397.480\\
   ......\\
\end{tt}

\begin{tt}
[ dihedraltypes ] \\
  ; i    l func        q0          cq\\
 NR5*  NR5    2     0.000     167.360\\
 NR5* NR5*    2     0.000     167.360\\
 ......\\
\end{tt}

\begin{tt}
[ dihedraltypes ] \\
  ; j    k func      phi0          cp   mult\\
    C   OA    1   180.000      16.736      2\\
    C    N    1   180.000      33.472      2\\
    ......\\
\end{tt}

\begin{tt}
[ dihedraltypes ] \\
;\\
; Ryckaert-Bellemans Dihedrals\\
;\\
; aj    ak      funct\\
CP2     CP2     3       9.2789  12.156  -13.120 -3.0597 26.240  -31.495\\
\end{tt}

Also in this file are the
\normindex{Ryckaert-Bellemans}~\cite{Ryckaert78} parameters for the
CP2-CP2 dihedrals in alkanes or alkane tails with the following
constants:

\begin{center}
(kJ/mol)\\
\begin{tabular}{llrllrllr}
$C_0$ & $=$ & $~ 9.28$ & $C_2$ & $=$ & $-13.12$ & $C_4$ & $=$ & $ 26.24$ \\
$C_1$ & $=$ & $ 12.16$ & $C_3$ & $=$ & $~-3.06$ & $C_5$ & $=$ & $-31.5 $ \\
\end{tabular}
\end{center}

({\bf Note:} The use of this potential implies the exclusion of LJ interactions
between the first and the last atom of the dihedral, and $\psi$ is defined
according to the '\swapindex{polymer}{convention}' ($\psi_{trans}=0$)).

So there are three types of dihedrals in the {\gromacs} force field:
\begin{itemize}
\item \swapindex{proper}{dihedral} : funct = 1, with mult = multiplicity, so the
                                   number of possible angles
\item \swapindex{improper}{dihedral} : funct = 2
\item Ryckaert-Bellemans dihedral : funct = 3
\end{itemize}
In the file {\tt ff???bon.itp} you can add bonded parameters. If you
want to include parameters for new atom types, make sure you define
this new atom type in {\tt ff???.atp} as well.

\subsection{Non-bonded parameters}
\label{subsec:nbpar}
The \swapindex{non-bonded}{parameter}s consist of the Van der Waals parameters
V ({\tt c6}) and W ({\tt c12}), as listed in the file {\tt ff???nb.itp},
where {\tt ptype} is the particle type (see \tabref{ptype}):

\begin{tt}
[ atomtypes ]\\
;name        mass      charge   ptype            c6           c12\\
    O    15.99940       0.000       A   0.22617E-02   0.74158E-06\\
   OM    15.99940       0.000       A   0.22617E-02   0.74158E-06\\
   .....\\
\end{tt}

\begin{tt}
[ nonbond\_params ]\\
  ; i    j func          c6           c12\\
    O    O    1 0.22617E-02   0.74158E-06\\
    O   OA    1 0.22617E-02   0.13807E-05\\
    .....\\
\end{tt}

\begin{tt}
[ pairtypes ]\\
  ; i    j func         cs6          cs12 ; THESE ARE 1-4 INTERACTIONS\\
    O    O    1 0.22617E-02   0.74158E-06\\
    O   OM    1 0.22617E-02   0.74158E-06\\
    .....\\
\end{tt}

The parameters V and W can be defined in two different ways, depending on
the combination rule that was chosen in the {\tt [~defaults~]} section
op the topology file (see~\ssecref{topfile}):
\begin{eqnarray}
\mbox{for combination rule 1}: & &
\begin{array}{llllll}
  \mbox{V}_{ii} & = & C^{(6)}_{i}  & = & 4\,\epsilon_i\sigma_i^{6} &
  \mbox{[ kJ mol$^{-1}$ nm$^{6}$ ]}\\
  \mbox{W}_{ii} & = & C^{(12)}_{i} & = & 4\,\epsilon_i\sigma_i^{12} &
  \mbox{[ kJ mol$^{-1}$ nm$^{12}$ ]}\\
\end{array}
\\
\mbox{for combination rules 2 and 3}: & &
\begin{array}{llll}
  \mbox{V}_{ii} & = & \sigma_i   & \mbox{[ nm ]} \\
  \mbox{W}_{ii} & = & \epsilon_i & \mbox{[ kJ mol$^{-1}$ ]}
\end{array}
\end{eqnarray}
Some or all combinations for different atom-types can be given in
the {\tt [~nonbond\_params~]} section. Any combination that is
not given will be 
computed according to the \normindex{combination rule}:
\begin{eqnarray}
\mbox{for combination rules 1 and 3}: & &
\begin{array}{lll}
  C^{(6)}_{ij}  & = & \left(C^{(6)}_i\,C^{(6)}_j\right)^{\frac{1}{2}} \\
  C^{(12)}_{ij} & = & \left(C^{(12)}_i\,C^{(12)}_j\right)^{\frac{1}{2}}
\end{array}
\\
\mbox{for combination rule 2}: & &
\begin{array}{lll}
  \sigma_{ij}   & = & \frac{1}{2}(\sigma_i+\sigma_j) \\
  \epsilon_{ij} & = & \sqrt{\epsilon_i\,\epsilon_j}
\end{array}
\end{eqnarray}

\subsection{\normindex{Pair interaction}s}
Extra Lennard-Jones and electrostatic interactions between pairs
of atoms in a molecule can be added in the {\tt [~pairs~]} section of
a molecule definition. The parameters for these interactions can
be set independently from the non-bonded interaction parameters.
In the {\gromos} force fields pairs are only used
to modify the \normindex{1-4 interaction}s (interactions of atoms
separated by three bonds). In these forcefields the 1-4 interactions
are excluded from the non-bonded interactions (see \secref{excl}).

The pair interaction parameters for the atom types
in {\tt ff???nb.itp} are listed in the {\tt [~pairtypes~]} section.
The {\gromos} force fields list all these interaction parameters
explicitly, but this section might be empty for force fields like
OPLS that calculate the 1-4 interactions by uniformly scaling the parameters.
Pair parameters which are not present in the {\tt [~pairtypes~]} section
are only generated when generate pairs is set to yes in the topology
(see \ssecref{topfile}). When generate pairs is set to no, {\tt grompp}
will give a warning for each pair type for which no parameters are given.

\ifthenelse{\equal{\gmxmajor}{4}}{
The normal pair interactions, intended for 1-4 interactions,
have function type 1. Function types 2 and 3 are intended
for free-energy simulations. When determining hydration
free-energies, the solute needs to be decoupled from the solvent.
This can be done by adding a B-state topology (see \secref{fecalc})
with all non-bonded parameters, i.e. charges and LJ parameters,
of the solute set to zero. But the free-energy difference between the A and
B state is not the total hydration free-energy, one has to
add the free-energy for reintroducing the internal Coulomb and 
interactions in the solute. This second step can be combined with
the first step when the Coulomb and LJ interactions within
the solute are not modified. For this purpose there is a pairs
function type 2, which is identical to function type 1, except
that the B-state parameters are always identical to the A-state
parameters. For searching the parameters in the {\tt [~pairtypes~]} section
no distinction is made between function type 1 and 2.
Function type 3 is intended to replace the non-bonded interaction.
It uses the unscaled charges and the non-bonded LJ parameters.
Type 3 also only uses the A-state parameters. Note that
one should add exclusions for all atom pairs participating in pair
interactions type 3, otherwise such pairs will also end
up in the normal neighborlists.

All three pair types always use plain Coulomb interactions,
even when Reaction-field, PME, Ewald or shifted Coulomb interactions
are selected for the non-bonded interactions.
Energies for types 1 and 2 are written to the energy and log file
in seperate ``14'' LJ and Coulomb entries per energy group pair.
Energies for type 3 are added to the LJ and Coulomb SR terms.
}{}

\section{Exclusions}
\label{sec:excl}
The \normindex{exclusions} for bonded particles are generated by {\tt
grompp} for neighboring atoms up to a certain number of bonds away, as
defined in the {\tt [~moleculetype~]} section in the topology file
(see \ssecref{topfile}). Particles are considered bonded when they are
connected by bonds ({\tt [~bonds~]} types 1 to 5, 7 or 8) or constraints
({\tt [~constraints~]} type 1).
{\tt[~bonds~]} type 5 can be used to create a \normindex{connection}
between two atoms without creating an interaction.
There is a \normindex{harmonic interaction}
({\tt[~bonds~]} type 6) which does not connect the atoms by a chemical bond.
There is also a second constraint type ({\tt[~constraints~]} type 2)
which fixes the distance, but does not connect
the atoms by a chemical bond.
For a complete list of all these interactions see \tabref{topfile2}.

Extra exclusions within a molecule can be added manually
in a {\tt [~exclusions~]} section. Each line should start with one
atom index, followed by one or more atom indices. All non-bonded
interactions between the first atom and the other atoms will be excluded.

When all non-bonded interactions within or between groups of atoms need
to be excluded, is it more convenient and much more efficient to use
energy monitor group exclusions (see \secref{group}).

\section{\normindex{Constraint}s}
\label{sec:constraints}
Constraints are defined in the {\tt [~constraints~]} section.
The format is two atom numbers followed by the function type,
which can be 1 or 2 and the constraint distance.
The only difference between the two types is that type 1 is used
for generating exclusions and type 2 is not (see \secref{excl}).
The distances are constrained using the LINCS or the SHAKE algorithm,
which can be selected in the {\tt *.mdp} file.
Both types of constraints can be perturbed in free-energy calculations
by adding a second constraint distance (see \ssecref{constraintforce}).
Several types of bonds and angles (see \tabref{topfile2}) can
be converted automatically to constraints by {\tt grompp}.
There are several options for this in the {\tt *.mdp} file.

We have also implemented the \normindex{SETTLE} algorithm~\cite{Miyamoto92}
which is an analytical solution of SHAKE specifically for water. 
SETTLE can be selected in the topology file. Check for instance the
SPC molecule definition:\\
\begin{tt}
[ moleculetype ]\\
; molname       nrexcl\\
SOL             1\\
\end{tt}\\
\begin{tt}
[ atoms ]\\
; nr    at type res nr  ren nm  at nm   cg nr   charge\\
1       OW      1       SOL     OW1     1       -0.82\\
2       HW      1       SOL     HW2     1        0.41\\
3       HW      1       SOL     HW3     1        0.41\\
\end{tt}\\
\begin{tt}
[ settles ]\\
; OW    funct   doh     dhh\\
1       1       0.1     0.16333\\
\end{tt}\\
\begin{tt}
[ exclusions ]\\
1       2       3\\
2       1       3\\
3       1       2\\
\end{tt}
The section {\tt [ settles ]} defines the first atom of the watery molecule.
The settle funct is always one, and the distance between O-H and H-H distances
must be given. Note that the algorithm can also be used
for TIP3P and TIP4P~\cite{Jorgensen83}.
TIP3P just has another geometry. TIP4P has a virtual site, but since 
that is generated it does not need to be shaken (nor stirred).

\section{\normindex{Database}s}

\subsection{Residue database}
\label{subsec:rtp}
The file holding the residue database is {\tt ff???.rtp}. Originally
this file contained building blocks (amino acids) for proteins, and is
the {\gromacs} interpretation of the {\tt rt37c4.dat} file of {\gromos}. So
the residue file contains information (bonds, charge, charge groups
and improper dihedrals) for a frequently used building block. It is
better {\em not} to change this file because it is standard input for
{\tt pdb2gmx}, but if changes are needed make them in the
{\tt *.top} file (see~\ssecref{topfile}). 
However, in the {\tt ff???.rtp} file the user can define a new
\normindex{building block} or molecule: see for example 2,2,2-trifluoroethanol
(TFE) or {\em n}-decane (C10). But when defining new molecules
(non-protein) it is preferable to create a {\tt *.itp}
file. This will be discussed in section~\ssecref{molitp}.
When adding a new protein residue to the database, don't forget to
add the residue name to the {\tt \normindex{aminoacids.dat}} file,
so that {\tt grompp}, {\tt make\_ndx} and analysis tools can recognize
the residue as a protein residue (see \ssecref{defaultgroups}).

The file {\tt ff???.rtp} is only used by {\tt pdb2gmx}.
As mentioned before, the only extra information this
program needs from {\tt ff???.rtp} is bonds, charges of atoms,
charge groups and improper dihedrals, because the rest is read from
the coordinate input file (in the case of {\tt pdb2gmx}, a pdb format
file). Some proteins contain residues that are not standard, but are
listed in the coordinate file. You have to construct a building block
for this ``strange'' residue, otherwise you will not obtain a
{\tt *.top} file. This also holds for molecules in the
coordinate file such as phosphate or sulphate ions.
The residue database is constructed in the following way:\\
\begin{small}
\begin{tt}
[ bondedtypes ]  ; mandatory\\
; bonds  angles  dihedrals  impropers\\
     1       1          1          2  ; mandatory\\
\end{tt}\\
\begin{tt}
[ GLY ]  ; mandatory\\
\end{tt}\\
\begin{tt}
 [ atoms ]  ; mandatory \\
; name  type  charge  chargegroup \\
     N     N  -0.280     0\\
     H     H   0.280     0\\
    CA   CH2   0.000     1\\
     C     C   0.380     2\\
     O     O  -0.380     2\\
\end{tt}\\
\begin{tt}
 [ bonds ]  ; optional\\
;atom1 atom2      b0      kb\\
     N     H\\
     N    CA\\
    CA     C\\
     C     O\\
    -C     N\\
\end{tt}\\
\begin{tt}
 [ exclusions ]  ; optional\\
;atom1 atom2\\
\end{tt}\\
\begin{tt}
 [ angles ]  ; optional\\
;atom1 atom2 atom3    th0    cth\\
\end{tt}\\
\begin{tt}
 [ dihedrals ]  ; optional\\
;atom1 atom2 atom3 atom4   phi0     cp   mult\\
\end{tt}\\
\begin{tt}
 [ impropers ]  ; optional\\
;atom1 atom2 atom3 atom4     q0     cq\\
     N    -C    CA     H\\
    -C   -CA     N    -O\\
\end{tt}\\
\begin{tt}
[ ZN ]\\
\end{tt}\\
\begin{tt}
 [ atoms ]\\
    ZN    ZN   2.000     0\\
\end{tt}
\end{small}

The file is free format, the only restriction is that there can be at
most one entry on a line.  The first field in the file is the {\tt
[~bondedtypes~]} field, which is followed by four numbers, that
indicate the interaction type for bonds, angles, dihedrals and
improper dihedrals.  The file contains residue entries, which consist
of atoms and optionally bonds, angles dihedrals and impropers.  The
charge group codes denote the charge group numbers. Atoms in the same
charge group should always be below each other. When using the
hydrogen database with {\tt pdb2gmx} for adding missing hydrogens, the
atom names defined in the {\tt .rtp} entry should correspond exactly
to the naming convention used in the hydrogen database,
see~\ssecref{hdb}. The atom names in the bonded interaction can be
preceded by a minus or a plus, indicating that the atom is in the
preceding or following residue respectively.  Parameters can be added
to bonds, angles, dihedrals and impropers, these parameters override
the standard parameters in the {\tt .itp} files.  This should only be
used in special cases. Instead of parameters, a string can be added
for each bonded interaction, this is used in \gromos{96} {\tt .rtp}
files. These strings are copied to the topology file and can be
replaced by force field parameters by the C-preprocessor in {\tt
grompp} using {\tt \#define} statements.

{\tt pdb2gmx} automatically generates all angles. This means that for the
{\gromacs} force field
the {\tt [~angles~]} field is only useful for overriding {\tt .itp}
parameters. For the {\gromosv{96}} force field the interaction number
off all angles need to be specified.

{\tt pdb2gmx} automatically generates one proper dihedral for every rotatable
bond, preferably on heavy atoms. When the {\tt [~dihedrals~]} field is used,
no other dihedrals will be generated for the bonds corresponding to the
specified  dihedrals. It is possible to put more than one dihedral on a
rotatable bond. 

{\tt pdb2gmx} sets the number of exclusions to 3, which
means that interactions between atoms connected by at most 3 bonds are
excluded. Pair interactions are generated for all pairs of atoms which are
separated by 3 bonds (except pairs of hydrogens).
When more interactions need to be excluded, or some pair interactions should
not be generated, an {\tt [~exclusions~]} field can be added, followed by
pairs of atom names on separate lines. All non-bonded and pair interactions
between these atoms will be excluded.

\subsection{Hydrogen database}
\label{subsec:hdb}
The \swapindex{hydrogen}{database} is stored in {\tt ff???.hdb}. It
contains information for the {\tt pdb2gmx} program on how to connect
hydrogen atoms to existing atoms. In versions of the database before
{\gromacs} 3.3, hydrogen atoms were named after the atom they are
connected to: the first letter of the atom name ws replaced by an
'H'. In the versions from 3.3 onwards, the H atom has to be listed explicitly,
because the old behaviour was protein-specific and hence could not
be generalized to other molecules.
If more then one hydrogen atom is connected to the same atom, a
number will be added to the end of the hydrogen atom name. For
example, adding two hydrogen atoms to {\tt ND2} (in asparagine), the
hydrogen atoms will be named {\tt HD21} and {\tt HD22}. This is
important since atom naming in the {\tt .rtp} file (see~\ssecref{rtp})
must be the same. The format of the hydrogen database is as follows:\\
%
\begin{small}
\begin{tt}
; res   \# additions\\
        \# H add type    H       i       j       k\\
ALA     1\\
        1       1       H       N       -C      CA\\
ARG     4\\
        1       2       H       N       CA      C\\
        1       1       HE      NE      CD      CZ\\
        2       3       HH1     NH1     CZ      NE\\
        2       3       HH2     NH2     CZ      NE\\
\end{tt}
\end{small}

On the first line we see the residue name (ALA or ARG) and the number
of additions. After that follows one line for each addition, on which
we see:
\begin{itemize}
\item The number of H atoms added
\item The way of adding H atoms, can be any of
\begin{enumerate}
\item[1]{\em one planar hydrogen, {\eg} rings or peptide bond}\\
one hydrogen atom (n) is generated, lying in the plane of atoms
(i,j,k) on the plane bisecting angle (j-i-k) at a distance of 0.1 nm
from atom i, such that the angles (n-i-j) and (n-i-k) are $>$ 90$^{\rm o}$

\item[2]{\em one single hydrogen, {\eg} hydroxyl}\\
one hydrogen atom (n) is generated at a distance of 0.1 nm from atom
i, such that angle (n-i-j)=109.5 degrees and dihedral (n-i-j-k)=trans

\item[3]{\em two planar hydrogens, {\eg} -NH{$_2$}}\\
two hydrogens (n1,n2) are generated at a distance of 0.1 nm from atom
i, such that angle (n1-i-j)=(n2-i-j)=120 degrees and dihedral
(n1-i-j-k)=cis and (n2-i-j-k)=trans, such that names are according to
IUPAC standards~\cite{iupac70}

\item[4]{\em two or three tetrahedral hydrogens, {\eg} -CH{$_3$}}\\
three (n1,n2,n3) or two (n1,n2) hydrogens are generated at a distance
of 0.1 nm from atom i, such that angle
(n1-i-j)=(n2-i-j)=(n3-i-j)=109.47$^{\rm o}$, dihedral (n1-i-j-k)=trans,
(n2-i-j-k)=trans+120 and (n3-i-j-k)=trans+240 degrees

\item[5]{\em one tetrahedral hydrogen, {\eg} C{$_3$}CH}\\
one hydrogen atom (n$\prime$) is generated at a distance of 0.1 nm from atom
i in tetrahedral conformation such that angle
(n$\prime$-i-j)=(n$\prime$-i-k)=(n$\prime$-i-l)=109.47$^{\rm o}$

\item[6]{\em two tetrahedral hydrogens, {\eg} C-CH{$_2$}-C}\\
two hydrogen atoms (n1,n2) are generated at a distance of 0.1 nm from
atom i in tetrahedral conformation on the plane bissecting angle i-j-k
with angle (n-i-n2)=(n1-i-j)=(n1-i-k)=109.5

\item[7]{\em two water hydrogens}\\
two hydrogens are generated around atom i according to
SPC~\cite{Berendsen81} water geometry. The symmetry axis will
alternate between three coordinate axes in both directions

\item[10]{\em three water ``hydrogens''}\\
two hydrogens are generated around atom i according to
SPC~\cite{Berendsen81} water geometry. The symmetry axis will
alternate between three coordinate axes in both directions. In addition
an extra particle is generated on the position of the oxygen. This is for
use with four-atom water models such as TIP4P~\cite{Jorgensen83}

\item[10]{\em four water ``hydrogens''}\\
Same as above, except that two  additional
particles are generated on the position of the oxygen. This is for
use with five-atom water models such as TIP5P~\cite{Mahoney2000a}
\end{enumerate}

\item
The name of the new H atom

\item
Three or four control atoms (i,j,k,l), where the first always is the
atom to which the H atoms are connected. The other two or three depend
on the code selected (for water there is only one control atom).
\end{itemize}

\subsection{Termini database}
\label{subsec:tdb}
The \swapindex{termini}{database}s are stored in {\tt ff???-n.tdb} and
{\tt ff???-c.tdb} for the N- and C-termini respectively. They contain
information for the {\tt pdb2gmx} program on how to connect new atoms
to existing ones, which atoms should be removed or changed and which
bonded interactions should be added. The format of the is as follows
(this is an example from the {\tt ffgmx-c.tdb}):

\begin{tt}
[ None ]\\
\end{tt}\\
\begin{tt}
[ COO- ]\\
\end{tt}\\
\begin{tt}
[ replace ]\\
C       C       C       12.011  0.27\\
\end{tt}\\
\begin{tt}
[ add ]\\
2       8       O       C       CA      N\\
        OM      15.9994 -0.635\\
\end{tt}\\
\begin{tt}
[ delete ]\\
O\\
\end{tt}\\
\begin{tt}
[ impropers ]\\
C       O1      O2      CA\\
\end{tt}

The file is organized in blocks, each with a header specifying the
name of the block. These blocks correspond to different types of
termini that can be added to a molecule. In this example {\tt
[~None~]} is the first block, corresponding to a terminus that leaves
the molecule as it is; {\tt [~COO-~]} is the second terminus type,
corresponding to changing the terminal carbon atom into a deprotonated
carboxyl group. Block names cannot be any of the following: {\tt
replace}, {\tt add}, {\tt delete}, {\tt bonds}, {\tt angles}, {\tt
dihedrals}, {\tt impropers}; this would interfere with the parameters
of the block, and would probably also be very confusing to human
readers.

Per block the following options are present:
\begin{itemize}
\item {\tt [~replace~]} \\
replace an existing atom by one with a different atom type, atom name,
charge and/or mass. For each atom to be replaced on line should be
entered with the following fields:
\begin{itemize}
\item name of the atom to be replaced
\item new atom name
\item new atom type
\item new mass
\item new charge
\end{itemize}
\item {\tt [~add~]} \\
add new atoms. For each (group of) added atom(s), a two-line entry is
necessary. The first line contains the same fields as an entry in the
hydrogen database (name of the new atom, 
number of atoms, type of addition, control atoms,
see~\ssecref{hdb}), but the possible types of addition are extended
by two more, specifically for C-terminal additions:
\begin{enumerate}
\item[8]{\em two carboxyl oxygens, -COO{$^-$}}\\
two oxygens (n1,n2) are generated according to rule 3, at a distance
of 0.136 nm from atom i and an angle (n1-i-j)=(n2-i-j)=117 degrees
\item[9]{\em carboxyl oxygens and hydrogen, -COOH}\\
two oxygens (n1,n2) are generated according to rule 3, at distances of
0.123 nm and 0.125 nm from atom i for n1 and n2 resp. and angles
(n1-i-j)=121 and (n2-i-j)=115 degrees. One hydrogen (n') is generated
around n2 according to rule 2, where n-i-j and n-i-j-k should be read
as n'-n2-i and n'-n2-i-j resp.
\end{enumerate}
After this line another line follows which specifies the details of
the added atom(s), in the same way as for replacing atoms, {\ie}: 
\begin{itemize}
\item atom type
\item mass
\item charge
\end{itemize}
Like in the hydrogen database (see~\ssecref{rtp}), when more then
one atom is connected to an existing one, a number will be appended to
the end of the atom name. Note that, like in the hydrogen database the
atom name is now on the same line as the control atoms, whereas it was
at the beginning of the second line prior to {\gromacs} version 3.3.
\item {\tt [~delete~]}\\
delete existing atoms. One atom name per line.
\item {\tt [~bonds~]}, {\tt [~angles~]}, {\tt [~dihedrals~]} and {\tt [~impropers~]}\\
add additional bonded parameters. The format is identical to that used
in the {\tt ff???.rtp}, see~\ssecref{rtp}.
\end{itemize}

\section{File formats}
\subsection{\swapindex{Topology}{file}}
\label{subsec:topfile}
The topology file is built following the {\gromacs} specification for a
molecular topology.  A {\tt *.top} file can be generated by
{\tt pdb2gmx}.
All possible entries in the topology file are listed in
Tables \ref{tab:topfile1}, \ref{tab:topfile2}  and \ref{tab:topfile3}.
Also listed are all the units
of the parameters, which interactions can be perturbed for free energy
calculations, which bonded interactions are used by {\tt grompp}
for generating exclusions and which bonded interactions can be converted
to constraints by {\tt grompp}.

%\renewcommand\floatpagefraction{.2}

%\newcommand{\tts}{\tt \footnotesize}
\newcommand{\tts}{\tt \small}

% move these figures so they end up on facing pages 
% (first figure on even page)
\newcommand{\kJmol}{kJ mol$^{-1}$}
\newcommand{\kJmolnm}[1]{\kJmol nm$^{#1}$}
\newcommand{\kJmolrad}[1]{\kJmol rad$^{#1}$}
\newcommand{\kJmoldeg}[1]{\kJmol deg$^{#1}$}
\begin{table}[p]
\centerline{\begin{tabular}{|l|llllc|}
\multicolumn{6}{c}{\bf \large Parameters} \\
\dline
interaction 	& directive   	      & \#  & f. & parameters 				& F. E. \\
type	&		      	      & at. & tp &					& 	\\
\dline
{\em mandatory} & {\tts defaults}	& & &	non-bonded function type; & \\
		&			& & &	combination rule$^{(cr)}$; &\\
		&			& & &   generate pairs (no/yes); & \\
		&			& & &	fudge LJ (); fudge QQ () & \\
\hline
{\em mandatory} & {\tts atomtypes}	&   & 	& atom type; m (u); q (e); particle type; & \\
		&			&   &	& V$^{(cr)}$; W$^{(cr)}$ & \\
%\hline
		& {\tts bondtypes}	& \multicolumn{3}{l}{(see \tabref{topfile2}, directive {\tts bonds})}		& \\
		& {\tts pairtypes}	& \multicolumn{3}{l}{(see \tabref{topfile2}, directive {\tts pairs})}		& \\
		& {\tts angletypes}	& \multicolumn{3}{l}{(see \tabref{topfile2}, directive {\tts angles})}		& \\
           	& {\tts dihedraltypes}$^{(*)}$ & \multicolumn{3}{l}{(see \tabref{topfile2}, directive {\tts dihedrals})}& \\
		& {\tts constrainttypes}& \multicolumn{3}{l}{(see \tabref{topfile3}, directive {\tts constraints})}	& \\
LJ 		& {\tts nonbond\_params}	& 2 & 1	& $V^{(a)}$; $W^{(a)}$ & \\
Buckingham    	& {\tts nonbond\_params}	& 2 & 2	& $a$ (\kJmol); $b$ (nm$^{-1})$;  & \\
 & & & & $c_6$ (\kJmolnm{6}) & \\
\dline
\multicolumn{6}{c}{~} \\
\multicolumn{6}{c}{\bf \large Molecule definition(s)} \\
\dline
{\em mandatory} & {\tts moleculetype}	& & & 	molecule name; $n_{ex}^{(nrexcl)}$  &	\\
\hline
{\em mandatory} & {\tts atoms}		& 1 & 	& atom type; residue number; 	& type	\\
		&			&   &	& residue name; atom name; 	& 	\\
		&			&   &	& charge group number; $q$ (e); $m$ (u) 	& $q,m$ \\
\hline
\multicolumn{6}{|c|}{} \\
\multicolumn{6}{|c|}{intramolecular interaction and geometry definitions as described
in Tables \ref{tab:topfile2} and \ref{tab:topfile3}} \\
\multicolumn{6}{|c|}{} \\
\dline
\multicolumn{6}{c}{~} \\
\multicolumn{6}{c}{\bf \large System} \\
\dline
{\em mandatory} & {\tts system}		& & &	system name	&	\\
\hline
{\em mandatory} & {\tts molecules}	& & &	\multicolumn{2}{l|}{molecule name; number of molecules}	\\
\dline
\multicolumn{6}{c}{~} \\
\multicolumn{6}{l}{'\# at' is the number of atom types} \\
\multicolumn{6}{l}{'f. tp' is function type} \\
\multicolumn{6}{l}{'F. E.' indicates which parameters can be interpolated
during free energy calculations} \\
\multicolumn{6}{l}{~$^{(cr)}$ the combination rule determines the type of LJ parameters, see~\ssecref{nbpar}}\\
\multicolumn{6}{l}{~$^{(*)}$ for {\tts dihedraltypes} one can specify 4 atoms or the inner (outer for improper) 2 atoms}\\
\multicolumn{6}{l}{~$^{(nrexcl)}$ exclude neighbors $n_{ex}$ bonds away for non-bonded interactions}\\
\multicolumn{6}{l}{For free energy calculations, type, $q$ and $m$  or no parameters should be added}\\
\multicolumn{6}{l}{for topology 'B' ($\lambda = 1$) on the same line, after the normal parameters.}
\end{tabular}
}
\caption{The topology ({\tt *.top}) file.}
\label{tab:topfile1}
\end{table}
\begin{table}[p]
\centerline{\begin{tabular}{|l|llllc|}
\multicolumn{6}{c}{\bf \large Intramolecular interaction definitions} \\
\dline
interaction 	& directive   	      & \#  & f. & parameters 				& F. E. \\
type	&		      	      & at. & tp &					& 	\\
\dline
bond		& {\tts bonds}$^{(excl,con)}$ & 2 & 1	& $b_0$ (nm); $k_b$ (\kJmolnm{-2})	& all	\\
G96 bond	& {\tts bonds}$^{(excl,con)}$ & 2 & 2	& $b_0$ (nm); $k_b$ (\kJmolnm{-4})	& all	\\
morse		& {\tts bonds}$^{(excl,con)}$ & 2 & 3	& $b_0$ (nm); $D$ (\kJmol); $\beta$ (nm$^{-1}$) & \\
cubic bond	& {\tts bonds}$^{(excl,con)}$ & 2 & 4	& $b_0$ (nm); $C_{i=2,3}$ (\kJmolnm{-i}); & \\
connection	& {\tts bonds}$^{(excl)}$     & 2 & 5	& & \\
harmonic pot.	& {\tts bonds}		      & 2 & 6	& $b_0$ (nm); $k_b$ (\kJmolnm{-2})	& all	\\
FENE bond	& {\tts bonds}$^{(excl)}$     & 2 & 7	& $b_m$ (nm); $k_b$ (\kJmolnm{-2})	& 	\\
tab. bond	& {\tts bonds}$^{(excl)}$     & 2 & 8	& table number ($\geq 0$); $k$ (\kJmol) & $k$ 	\\
tab. bond n.c.  & {\tts bonds}                & 2 & 9	& table number ($\geq 0$); $k$ (\kJmol) & $k$ 	\\
LJ/Coul. 1-4	& {\tts pairs}		      & 2 & 1	& $V^{(cr)}$; $W^{(cr)}$ & all \\
LJ/Coul. 1-4	& {\tts pairs}		      & 2 & 2	& fudge QQ (); $q_i$, $q_j$ (e), $V^{(cr)}$; $W^{(cr)}$ &     \\
LJ/C. pair NB	& {\tts pairs\_nb}	      & 2 & 1	& $q_i$, $q_j$ (e); $V^{(cr)}$; $W^{(cr)}$ &    \\
angle		& {\tts angles}$^{(con)}$     & 3 & 1	& $\theta_0$ (deg); $k_\theta$ (\kJmolrad{-2}) & all	\\
G96 angle	& {\tts angles}$^{(con)}$     & 3 & 2	& $\theta_0$ (deg); $k_\theta$ (\kJmol) & all	\\
Cross bond-bond	& {\tts angles}               & 3 & 3	& $r_{1e}$, $r_{2e}$ (nm); $k_{rr'}$ (\kJmolnm{-2}) & 	\\
Cross bond-angle& {\tts angles}               & 3 & 4	&$r_{1e}$, $r_{2e}$ $r_{3e}$ (nm); $k_{r\theta}$ (\kJmolnm{-2}) & 	\\
Urey-Bradley    & {\tts angles}$^{(con)}$     & 3 & 5	& $\theta_0$ (deg); $k_\theta$ (\kJmol); $r_{13}$ (nm); & \\
                &                             &   &     & $k_{UB}$ (\kJmol) & \\
quartic angle	& {\tts angles}$^{(con)}$     & 3 & 6	& \multicolumn{2}{l|}{$\theta_0$ (deg); $C_{i=0,1,2,3,4}$ (\kJmolrad{-i})}	\\
tab. angle	& {\tts angles}               & 3 & 8	& table number ($\geq 0$); $k$ (\kJmol) & $k$ 	\\
proper dih.	& {\tts dihedrals}	      & 4 & 1	& $\phi_s$ (deg); $k_\phi$ (\kJmol); multiplicity & $\phi,k$	\\
improper dih.	& {\tts dihedrals}	      & 4 & 2	& $\xi_0$ (deg); $k_\xi$ (\kJmolrad{-2}) & all	\\
RB dihedral	& {\tts dihedrals}	      & 4 & 3	& $C_0$, $C_1$, $C_2$, $C_3$, $C_4$, $C_5$ (\kJmol) 		& all	\\
Fourier dih.	& {\tts dihedrals}	      & 4 & 5	& $C_1$, $C_2$, $C_3$, $C_4$ (\kJmol) 	& all	\\
tab. dihedral	& {\tts dihedrals}            & 4 & 8	& table number ($\geq 0$); $k$ (\kJmol) & $k$ 	\\
exclusions	& {\tts exclusions}	      & 1 & 	& one or more atom indices				& 	\\
\dline
\multicolumn{6}{c}{~} \\
\multicolumn{6}{l}{'\# at' is the number of atom indices}\\
\multicolumn{6}{l}{'f. tp' is function type}\\
\multicolumn{6}{l}{'F. E.' indicates which parameters
can be interpolated during free energy calculations}\\
\multicolumn{6}{l}{~$^{(cr)}$ the combination rule determines the type of LJ parameters, see~\ssecref{nbpar}}\\
\multicolumn{6}{l}{~$^{(excl)}$ used by {\tt grompp} for generating exclusions}\\
\multicolumn{6}{l}{~$^{(con)}$ can be converted to constraints by {\tt grompp}}\\
\multicolumn{6}{l}{For free energy calculations, all or no parameters for topology 'B' ($\lambda = 1$) should be added}\\
\multicolumn{6}{l}{on the same line, after the normal parameters, in the same order as the normal parameters.}
\end{tabular}
}
\caption{Intramolecular interaction definitions.}
\label{tab:topfile2}
\end{table}
\begin{table}[p]
\centerline{\begin{tabular}{|l|llllc|}
\multicolumn{6}{c}{\bf \large Intramolecular geometry and restraint definitions} \\
\dline
interaction 	& directive   	      & \#  & f. & parameters 				& F. E. \\
type	&		      	      & at. & tp &					& 	\\
\dline
constraint	& {\tts constraints}$^{(excl)}$& 2 & 1	& $b_0$ (nm) 				& all	\\
constr. n.c.    & {\tts constraints}	& 2 & 2	& $b_0$ (nm) 				& all	\\
settle		& {\tts settles}		& 3 & 1	& $d_{\mbox{\sc oh}}$, $d_{\mbox{\sc hh}}$ (nm) 		& 	\\
vsite2		& {\tts virtual\_sites2}	& 3 & 1	& $a$ ()					& 	\\
vsite3		& {\tts virtual\_sites3}	& 4 & 1	& $a$, $b$ ()				& 	\\
vsite3fd	& {\tts virtual\_sites3}	& 4 & 2	& $a$ (); $d$ (nm)				& 	\\
vsite3fad	& {\tts virtual\_sites3}	& 4 & 3	& $\theta$ (deg); $d$ (nm) 		& 	\\
vsite3out	& {\tts virtual\_sites3}	& 4 & 4	& $a$, $b$ (); $c$ (nm$^{-1}$) 		& 	\\
vsite4fd	& {\tts virtual\_sites4}	& 5 & 1	& $a$, $b$ (); $d$ (nm);	   		& 	\\
vsite COG 	& {\tts virtual\_sitesn}	& 1 & 1	& one or more construc. atom ind.   		& 	\\
vsite COM	& {\tts virtual\_sitesn}	& 1 & 2	& one or more construc. atom ind.   		& 	\\
vsite COW  	& {\tts virtual\_sitesn}	& 1 & 3	& one or more pairs consisting of	& 	\\
                &                               &   &   & a construc. atom ind. and weight           &       \\
position res.	& {\tts position\_restraints}   & 1 & 1	& $k_{x}$, $k_{y}$, $k_{z}$ (\kJmolnm{-2}) & all	\\
distance res.	& {\tts distance\_restraints}   & 2 & 1	& type; label; low, up$_1$, up$_2$ (nm); & \\
                &                               &   &   & weight () & \\
orient. res.    & \multicolumn{3}{l}{\tts orientation\_restraints} & &\\
                &	                        & 2 & 1	& exp.; label; $\alpha$; $c$ (U nm$^\alpha$); & \\
                &                               &   &   &  obs. (U); weight (U$^{-1}$) &\\
angle res.	& {\tts angle\_restraints}      & 4 & 1	& $\theta_0$ (deg); $k_c$ (\kJmol); & $\theta,k$	\\
                &                               &   &   & multiplicity & \\
angle res. z & {\tts angle\_restraints\_z}      & 2 & 1	& $\theta_0$ (deg); $k_c$ (\kJmol); & $\theta,k$	\\
                &                               &   &   & multiplicity & \\
\dline
\multicolumn{6}{c}{~} \\
\multicolumn{6}{l}{'\# at' is the number of atom indices}\\
\multicolumn{6}{l}{'f. tp' is function type}\\
\multicolumn{6}{l}{'F. E.' indicates which parameters
can be interpolated during free energy calculations}\\
\multicolumn{6}{l}{~$^{(excl)}$ used by {\tt grompp} for generating exclusions}\\
\multicolumn{6}{l}{For free energy calculations, all or no parameters for topology 'B' ($\lambda = 1$) should be added}\\
\multicolumn{6}{l}{on the same line, after the normal parameters, in the same order as the normal parameters.}
\end{tabular}
}
\caption{Intramolecular geometry and restraint definitions.}
\label{tab:topfile3}
\end{table}

%\renewcommand\floatpagefraction{.5}


Description of the file layout:
\begin{itemize}
\item semicolon (;) and newline surround comments
\item on a line ending with $\backslash$ the newline character is ignored.
\item directives are surrounded by {\tt [} and {\tt ]}
\item the topology consists of three levels:
\begin{itemize}
\item the parameter level (see \tabref{topfile1})
\item the molecule level, which should contain one or more molecule
      definitions (see \tabref{topfile2})
\item the system level: {\tt [~system~]}, {\tt [~molecules~]}
\end{itemize}
\item items should be separated by spaces or tabs, not commas
\item atoms in molecules should be numbered consecutively starting at 1
\item the file is parsed once only which implies that no forward
      references can be treated: items must be defined before they
      can be used
\item exclusions can be generated from the bonds or
      overridden manually
\item the bonded force types can be generated from the atom types or
      overridden per bond
\item it is possible to apply multiple bonded interactions of the same type
      on the same atoms
\item descriptive comment lines and empty lines are highly recommended
\item starting with {\gromacs} version 3.1.3 all directives at the
      parameter level can be used multiple times and there are no
      restrictions on the order, except that an atom type needs to be
      defined before it can be used in other parameter definitions
\item If parameters for a certain interaction are defined multiple times
      for the same combination of atom types the last definition is used;
      starting with {\gromacs} version 3.1.3 {\tt grompp} generates a
      warning for parameter redefinitions with different values
\item using one of the {\tt [~atoms~]}, {\tt [~bonds~]}, 
      {\tt [~pairs~]}, {\tt [~angles~]}, etc. without having used 
      {\tt [~moleculetype~]} 
      before is meaningless and generates a warning
\item using {\tt [~molecules~]} without having used
      {\tt [~system~]} before is meaningless and generates a warning.
\item after {\tt [~system~]} the only allowed directive is {\tt [~molecules~]}
\item using an unknown string in {\tt [~]} causes all the data until
      the next directive to be ignored, and generates a warning
\end{itemize}

Here is an example of a topology file, {\tt urea.top}:

\begin{tt}
;\\
;       Example topology file\\
;\\
; The force field files to be included\\
\#include "ffgmx.itp"\\
\end{tt}\\
\begin{tt}
[ moleculetype ]\\
; name  nrexcl\\
Urea         3\\
\end{tt}\\
\begin{tt}
[ atoms ]\\
;   nr    type   resnr  residu    atom    cgnr  charge\\
     1       C       1    UREA      C1       1   0.683\\
     2       O       1    UREA      O2       1  -0.683\\
     3      NT       1    UREA      N3       2  -0.622\\
     4       H       1    UREA      H4       2   0.346\\
     5       H       1    UREA      H5       2   0.276\\
     6      NT       1    UREA      N6       3  -0.622\\
     7       H       1    UREA      H7       3   0.346\\
     8       H       1    UREA      H8       3   0.276\\
\end{tt}\\
\begin{tt}
[ bonds ]\\
;  ai    aj funct           b0           kb\\
    3     4     1 1.000000e-01 3.744680e+05 \\
    3     5     1 1.000000e-01 3.744680e+05 \\
    6     7     1 1.000000e-01 3.744680e+05 \\
    6     8     1 1.000000e-01 3.744680e+05 \\
    1     2     1 1.230000e-01 5.020800e+05 \\
    1     3     1 1.330000e-01 3.765600e+05 \\
    1     6     1 1.330000e-01 3.765600e+05 \\
\end{tt}\\
\begin{tt}
[ pairs ]\\
;  ai    aj funct           c6          c12\\
    2     4     1 0.000000e+00 0.000000e+00 \\
    2     5     1 0.000000e+00 0.000000e+00 \\
    2     7     1 0.000000e+00 0.000000e+00 \\
    2     8     1 0.000000e+00 0.000000e+00 \\
    3     7     1 0.000000e+00 0.000000e+00 \\
    3     8     1 0.000000e+00 0.000000e+00 \\
    4     6     1 0.000000e+00 0.000000e+00 \\
    5     6     1 0.000000e+00 0.000000e+00 \\
\end{tt}\\
\begin{tt}
[ angles ]\\
;  ai    aj    ak funct          th0          cth\\
    1     3     4     1 1.200000e+02 2.928800e+02 \\
    1     3     5     1 1.200000e+02 2.928800e+02 \\
    4     3     5     1 1.200000e+02 3.347200e+02 \\
    1     6     7     1 1.200000e+02 2.928800e+02 \\
    1     6     8     1 1.200000e+02 2.928800e+02 \\
    7     6     8     1 1.200000e+02 3.347200e+02 \\
    2     1     3     1 1.215000e+02 5.020800e+02 \\
    2     1     6     1 1.215000e+02 5.020800e+02 \\
    3     1     6     1 1.170000e+02 5.020800e+02 \\
\end{tt}\\
\begin{tt}
[ dihedrals ]\\
;  ai    aj    ak    al funct          phi           cp         mult\\
    2     1     3     4     1 1.800000e+02 3.347200e+01 2.000000e+00 \\
    6     1     3     4     1 1.800000e+02 3.347200e+01 2.000000e+00 \\
    2     1     3     5     1 1.800000e+02 3.347200e+01 2.000000e+00 \\
    6     1     3     5     1 1.800000e+02 3.347200e+01 2.000000e+00 \\
    2     1     6     7     1 1.800000e+02 3.347200e+01 2.000000e+00 \\
    3     1     6     7     1 1.800000e+02 3.347200e+01 2.000000e+00 \\
    2     1     6     8     1 1.800000e+02 3.347200e+01 2.000000e+00 \\
    3     1     6     8     1 1.800000e+02 3.347200e+01 2.000000e+00 \\
\end{tt}\\
\begin{tt}
[ dihedrals ]\\
;  ai    aj    ak    al funct           q0           cq\\
    3     4     5     1     2 0.000000e+00 1.673600e+02 \\
    6     7     8     1     2 0.000000e+00 1.673600e+02 \\
    1     3     6     2     2 0.000000e+00 1.673600e+02 \\
\end{tt}\\
\begin{tt}
[ position\_restraints ]\\
; you wouldn't normally use this for a molecule like Urea,\\
; but we include it here for didactic purposes\\
; ai   funct    fc\\
   1     1     1000    1000    1000 ; Restrain to a point\\
   2     1     1000       0    1000 ; Restrain to a line (Y-axis)\\
   3     1     1000       0       0 ; Restrain to a plane (Y-Z-plane)\\
\\
; Include SPC water topology\\
\#include "spc.itp"\\
\end{tt}\\
\begin{tt}
[ system ]\\
Urea in Water\\
\end{tt}\\
\begin{tt}
[ molecules ]\\
;molecule name   nr.\\
Urea             1\\
SOL              1000\\
\end{tt}

Here follows the explanatory text.

{\bf {\tt [~defaults~]} :}
\begin{itemize}
\item non-bond type = 1 (Lennard-Jones) or 2 (Buckingham)\\
\item \normindex{combination rule} = 
\begin{enumerate}
\item For Lennard Jones: supply $C^{(6)}$ and $C^{(N)}$,
$C^{M}_{ij}=\sqrt{C^M_i\,C^M_j}$ (M = 6,N). 
Default value for N = 12, but it can be
overridden using the last parameter on this line.
For Buckingham potentials the combination rule is such that you give the
A, B and C parameters. $A_{ij} = \sqrt{A_i\, A_j}$ and similar for 
$C_{ij}$, $B_{ij} = 2/(1/B_i + 1/B_j)$.
\item supply $\sigma$ and $\epsilon$,
$\sigma_{ij}=\frac{1}{2}(\sigma_i+\sigma_j)$ and 
$\epsilon_{ij}=\sqrt{\epsilon_i\,\epsilon_j}$
\item supply $\sigma$ and $\epsilon$, $\sigma_{ij}=\sqrt{\sigma_i\,\sigma_j}$,
$\epsilon_{ij}=\sqrt{\epsilon_i\,\epsilon_j}$
\end{enumerate}
\item generate pairs = no
(the default, get 1-4 interactions from the pair list, when parameters
are not present in the list give a warning and use zeros)
or yes (generate 1-4 interactions which are not present in the pair list
from normal Lennard-Jones parameters using FudgeLJ)
\item FudgeLJ = factor to multiply Lennard-Jones 1-4 interactions with, default 1
\item FudgeQQ = factor to multiply electrostatic 1-4 interactions with, default 1
\item N = power for the repulsion term in a 6-N potential (with 
nonbonded-type Lennard Jones only)
\end{itemize}
{\bf note:} generate pairs, FudgeLJ, FudgeQQ and N are optional,
FudgeLJ is only used when generate pairs is set to 'yes'. However if you
want to specify N you need to give a value for the other parameters as well.

% move these figures so they end up on facing pages 
% (first figure on even page)
%\newcommand{\kJmol}{kJ mol$^{-1}$}
\newcommand{\kJmolnm}[1]{\kJmol nm$^{#1}$}
\newcommand{\kJmolrad}[1]{\kJmol rad$^{#1}$}
\newcommand{\kJmoldeg}[1]{\kJmol deg$^{#1}$}
\begin{table}[p]
\centerline{\small\begin{tabular}{|l|lllll|}
\multicolumn{6}{c}{\bf \large Parameters} \\
\hline
interaction 	& directive   	      & \#  & f. & parameters 				& pert \\
type	&		      	      & at. & tp &					& 	\\
\dline
{\em mandatory} & {\tt defaults}	& & &	non-bonded function type; & \\
		&			& & &	combination rule; 	 &\\
		&			& & &   generate pairs (no/yes); & \\
		&			& & &	fudge LJ (); fudge QQ () & \\
\hline
{\em mandatory} & {\tt atomtypes}	&   & 	& atom type; m (u); q (e); particle type; & \\
		&			&   &	& c$_6$ (\kJmolnm{6}); c$_{12}$ (\kJmolnm{12}) & \\
\hline
		& {\tt bondtypes}	& \multicolumn{3}{l}{(see Table~\ref{ta:topfile2}, directive {\tt bonds})}		& 	\\
		& {\tt constrainttypes}	& \multicolumn{3}{l}{(see Table~\ref{ta:topfile2}, directive {\tt constraints})}		& 	\\
		& {\tt pairtypes}	& \multicolumn{3}{l}{(see Table~\ref{ta:topfile2}, directive {\tt pairs})}		& 	\\
		& {\tt angletypes}	& \multicolumn{3}{l}{(see Table~\ref{ta:topfile2}, directive {\tt angles})}		& 	\\
proper dih.	& {\tt dihedraltypes}	& 2$^{(b)}$ & 1	& $\theta_{max}$ (deg); f$_c$ (\kJmol); mult & X$^{(a)}$ \\
improper dih.	& {\tt dihedraltypes}	& 2$^{(c)}$ & 2	& $\theta_0$ (deg); f$_c$ (\kJmolrad{-2}) & X	\\
RB dihedral	& {\tt dihedraltypes}	& 2$^{(b)}$ & 3	&  C$_0$, C$_1$, C$_2$, C$_3$, C$_4$, C$_5$ (\kJmol) 		&	\\
LJ 		& {\tt nonbond\_params}	& 2 & 1	& c$_6$ (\kJmolnm{6}); c$_{12}$ (\kJmolnm{12}) & \\
Buckingham    	& {\tt nonbond\_params}	& 2 & 2	& a (\kJmol); b (nm$^{-1})$;  & \\
 & & & & c$_6$ (\kJmolnm{6}) & \\
\hline
\multicolumn{6}{c}{~} \\
\multicolumn{6}{l}{'\# at' is the number of atom types} \\
\multicolumn{6}{l}{'f. tp' is function type} \\
\multicolumn{6}{l}{'pert' indicates if this interaction type
can be modified during free energy perturbation} \\
\multicolumn{6}{l}{~$^{(a)}$ multiplicities can not be modified} \\
\multicolumn{6}{l}{~$^{(b)}$ the outer two atoms in the dihedral} \\
\multicolumn{6}{l}{~$^{(c)}$ the inner two atoms in the dihedral} \\
\multicolumn{6}{l}{For free energy perturbation, the parameters for topology 'B' (lambda = 1) should be added} \\
\multicolumn{6}{l}{on the same line, after the normal parameters,
in the same order as the normal parameters.} \\
\end{tabular}
}
\caption{The topology ({\tt *.top}) file, part 1.}
\label{ta:topfile1}
\end{table}
\begin{table}[p]
\centerline{\small\begin{tabular}{|l|lllll|}
\multicolumn{6}{c}{\bf \large Molecule definition} \\
\hline
interaction 	& directive   	      & \#  & f. & parameters 				& pert \\
type	&		      	      & at. & tp &					& 	\\
\dline
{\em mandatory} & {\tt moleculetype}	& & & 	molecule name; &	\\
		&			& & & 	exclude neighbors \# bonds away &	\\
		&			& & & 	for non-bonded interactions & \\
\hline
{\em mandatory} & {\tt atoms}		& 1 & 	& atom type; residue number; 	& 	\\
		&			&   &	& residue name; atom name; 	& 	\\
		&			&   &	& charge group number; q (e); m (u) 	& X$^{(b)}$ \\
\hline
bond		& {\tt bonds}		& 2 & 1	& b$_0$ (nm); f$_c$ (\kJmolnm{-2})	& X	\\
G96 bond	& {\tt bonds}		& 2 & 2	& b$_0$ (nm); f$_c$ (\kJmolnm{-4})	& X	\\
morse		& {\tt bonds}		& 2 & 3	& b$_0$ (nm); D (\kJmol); $\beta$ (nm$^{-1}$) & X \\
LJ 1-4		& {\tt pairs}		& 2 & 1	& c$_6$ (\kJmolnm{6}); & \\
 & & & & c$_{12}$ (\kJmolnm{12}) & X \\
angle		& {\tt angles}		& 3 & 1	& $\theta_0$ (deg); f$_c$ (\kJmolrad{-2}) & X	\\
G96 angle	& {\tt angles}		& 3 & 2	& $\theta_0$ (deg); f$_c$ (\kJmol) & X	\\
proper dih.	& {\tt dihedrals}	& 4 & 1	& $\theta_{max}$ (deg); f$_c$ (\kJmol); mult & X$^{(a)}$	\\
improper dih.	& {\tt dihedrals}	& 4 & 2	& $\theta_0$ (deg); f$_c$ (\kJmolrad{-2}) & X	\\
RB dihedral	& {\tt dihedrals}	& 4 & 3	& C$_0$, C$_1$, C$_2$, C$_3$, C$_4$, C$_5$ (\kJmol) 		&	\\
constraint	& {\tt constraints}	& 2 & 1	& b$_0$ (nm) 				& X	\\
constr. n.c.    & {\tt constraints}	& 2 & 2	& b$_0$ (nm) 				& X	\\
settle		& {\tt settles}		& 3 & 1	& d$_{\mbox{\sc oh}}$, d$_{\mbox{\sc hh}}$ (nm) 		& 	\\
dummy2		& {\tt dummies2}	& 2 & 1	& a ()					& 	\\
dummy3		& {\tt dummies3}	& 3 & 1	& a, b ()				& 	\\
dummy3fd	& {\tt dummies3}	& 3 & 2	& a (); d (nm)				& 	\\
dummy3fad	& {\tt dummies3}	& 3 & 3	& d (nm); $\theta$ (deg) 		& 	\\
dummy3out	& {\tt dummies3}	& 3 & 4	& a, b (); c (nm$^{-1}$) 		& 	\\
dummy4fd	& {\tt dummies4}	& 4 & 1	& a, b (); d (nm);	   		& 	\\
position res.	& {\tt position\_restraints}	& 1 & 1	& k$_{x}$, k$_{y}$, k$_{z}$ (\kJmolnm{-2}) & 	\\
%wpol	& {\tt position\_restraints}	& 1 & 2	& ???	& 	\\
distance res.	& {\tt distance\_restraints}	& 2 & 1	& type; index; low, up$_1$, up$_2$ (nm); & \\
 & & & & factor () & \\
angle res.	& {\tt angle\_restraints}	& 4 & 1	& $\theta_0$ (deg); f$_c$ (\kJmol); mult & X$^{(a)}$	\\
angle res. z & {\tt angle\_restraints\_z}	& 2 & 1	& $\theta_0$ (deg); f$_c$ (\kJmol); mult & X$^{(a)}$	\\
exclusions	& {\tt exclusions}	& 1 & 	& one or more atom indices				& 	\\
\hline
\multicolumn{6}{c}{~} \\
\multicolumn{6}{c}{\bf \large System} \\
\hline
{\em mandatory} & {\tt system}		& & &	system name				&	\\
\hline
{\em mandatory} & {\tt molecules}	& & &	\multicolumn{2}{l|}{molecule name; number of molecules}	\\
\hline
\multicolumn{6}{c}{~} \\
\multicolumn{6}{l}{'\# at' is the number of atom indices} \\
\multicolumn{6}{l}{'f. tp' is function type} \\
\multicolumn{6}{l}{'pert' indicates if this interaction type
can be modified during free energy perturbation} \\
\multicolumn{6}{l}{~$^{(a)}$ multiplicities can not be modified} \\
\multicolumn{6}{l}{~$^{(b)}$ only the atom type, charge and mass can be modified} \\
\multicolumn{6}{l}{For free energy perturbation, the parameters for topology 'B' (lambda = 1) should be added} \\
\multicolumn{6}{l}{on the same line, after the normal parameters,
in the same order as the normal parameters.} \\
\end{tabular}
}
\caption{The topology ({\tt *.top}) file, part 2.}
\label{ta:topfile2}
\end{table}



{\bf {\tt \#include "ffgmx.itp"} :} this includes the bonded and
non-bonded {\gromacs} parameters, the {\tt gmx} in {\tt ffgmx} will be
replaced by the name of the force field you are actually using.

{\bf {\tt [~moleculetype~]} :} defines the name of your molecule in
this {\tt *.top} and nrexcl = 3 stands for excluding non-bonded
interactions between atoms that are no further than 3 bonds away.

{\bf {\tt [~atoms~]} :} defines the molecule, where {\tt nr} and {\tt
type} are fixed, the rest is user defined. So {\tt atom} can be named
as you like, {\tt cgnr} made larger or smaller (if possible, the total
charge of a charge group should be zero), and charges can be changed
here too.

{\bf {\tt [~bonds~]} :} no comment.

{\bf {\tt [~pairs~]} :} LJ and Coulomb 1-4 interactions

{\bf {\tt [~angles~]} :} no comment

{\bf {\tt [~dihedrals~]} :} in this case there are 9 proper dihedrals
(funct = 1), 3 improper (funct = 2) and no Ryckaert-Bellemans type
dihedrals. If you want to include Ryckaert-Bellemans type dihedrals
in a topology, do the following (in case of {\eg} decane):
\begin{tt}
[ dihedrals ]\\
;  ai    aj    ak    al funct       c0       c1       c2\\
    1    2     3     4     3 \\
    2    3     4     5     3\\
\end{tt}
and do not forget to {\em erase the 1-4 interaction} 
in {\tt [~pairs~]}!

{\bf {\tt [~position\_restraints~]} :} harmonically restrain the selected particles
to reference positions (\secref{posre}). 
The reference positions are read from a 
separate coordinate file by \normindex{grompp}.

{\bf {\tt \#include "spc.itp"} :} includes a topology file that was already
constructed (see next section, molecule.itp).

{\bf {\tt [~system~]} :} title of your system, user defined

{\bf {\tt [~molecules~]} :} this defines the total number of (sub)molecules
in your system that are defined in this {\tt *.top}. In this
example file it stands for 1 urea molecules dissolved in 1000 water
molecules. The molecule type SOL is defined in the {\tt spc.itp} file.

\subsection{Molecule.itp file}
\label{subsec:molitp}
If you construct a topology file you will use frequently (like a water
molecule, {\tt spc.itp}) it is better to make a {\tt molecule.itp}
file, which only lists the information of the molecule: 

\begin{tt}
[ moleculetype ]\\
; name  nrexcl\\
Urea       3\\
\end{tt}\\
\begin{tt}
[ atoms ]\\
;   nr    type   resnr  residu    atom    cgnr  charge\\
     1       C       1    UREA      C1       1   0.683  \\
     .................\\
     .................\\
     8       H       1    UREA      H8       3   0.276\\
\end{tt}\\
\begin{tt}
[ bonds ]\\
;  ai    aj funct           c0           c1\\
    3     4     1 1.000000e-01 3.744680e+05 \\
     .................\\
     .................\\
    1     6     1 1.330000e-01 3.765600e+05 \\
\end{tt}\\
\begin{tt}
[ pairs ]\\
;  ai    aj funct           c0           c1\\
    2     4     1 0.000000e+00 0.000000e+00 \\
     .................\\
     .................\\
    5     6     1 0.000000e+00 0.000000e+00 \\
\end{tt}\\
\begin{tt}
[ angles ]\\
;  ai    aj    ak funct           c0           c1\\
    1     3     4     1 1.200000e+02 2.928800e+02 \\
     .................\\
     .................\\
    3     1     6     1 1.170000e+02 5.020800e+02 \\
\end{tt}\\
\begin{tt}
[ dihedrals ]\\
;  ai    aj    ak    al funct           c0           c1           c2\\
    2     1     3     4     1 1.800000e+02 3.347200e+01 2.000000e+00 \\
     .................\\
     .................\\
    3     1     6     8     1 1.800000e+02 3.347200e+01 2.000000e+00 \\
\end{tt}\\
\begin{tt}
[ dihedrals ]\\
;  ai    aj    ak    al funct           c0           c1\\
    3     4     5     1     2 0.000000e+00 1.673600e+02 \\
    6     7     8     1     2 0.000000e+00 1.673600e+02 \\
    1     3     6     2     2 0.000000e+00 1.673600e+02 \\
\end{tt}

This results in a very short {\tt *.top} file as described in the
previous section, but this time you only need to include files:

\begin{tt}
; The force field files to be included\\
\#include "ffgmx.itp"\\
  \\      
; Include urea topology\\
\#include "urea.itp"\\
\\
; Include SPC water topology\\
\#include "spc.itp"\\
\end{tt}\\
\begin{tt}
[ system ]\\
Urea in Water\\
\end{tt}\\
\begin{tt}
[ molecules ]\\
;molecule name  number\\
Urea              1\\
SOL               1000\\
\end{tt}

\subsection{Ifdef option}
\label{subsec:ifdef}
A very powerful feature in {\gromacs} is the use of {\tt \#ifdef}
statements in your {\tt *.top} file. By making use of this statement,
different parameters for one molecule can be used in the same {\tt
*.top} file. An example is given for TFE, where there is an option to
use different charges on the atoms: charges derived by De Loof
{\etal}~\cite{Loof92} or by Van Buuren and
Berendsen~\cite{Buuren93a}. In fact you can use all the options of the
C-Preprocessor, {\tt cpp}, because this is used to scan the file.  The
way to make use of the {\tt \#ifdef} option is as follows:
\begin{itemize}
\item in {\tt grompp.mdp} (the {\gromacs} preprocessor input
      parameters) use the option\\{\tt define = -DDeloof}\\ or
      \\{\tt define = }
\item put the {\tt \#ifdef} statements in your {\tt *.top}, as
      shown below: 
\end{itemize}
{\small\begin{tt}
...\\
\end{tt}\\
\begin{tt}
[ atoms ]\\
; nr   type   resnr  residu   atom    cgnr    charge      mass\\
\#ifdef DeLoof\\
; Use Charges from DeLoof\\
   1      C      1    TFE       C       1       0.74\\
   2      F      1    TFE       F       1      -0.25\\
   3      F      1    TFE       F       1      -0.25\\
   4      F      1    TFE       F       1      -0.25\\
   5    CH2      1    TFE     CH2       1       0.25\\
   6     OA      1    TFE      OA       1      -0.65\\
   7     HO      1    TFE      HO       1       0.41\\
\#else\\
; Use Charges from VanBuuren\\
   1      C      1    TFE       C       1       0.59\\
   2      F      1    TFE       F       1      -0.2\\
   3      F      1    TFE       F       1      -0.2\\
   4      F      1    TFE       F       1      -0.2\\
   5    CH2      1    TFE     CH2       1       0.26\\
   6     OA      1    TFE      OA       1      -0.55\\
   7     HO      1    TFE      HO       1       0.3\\
\#endif\\
\end{tt}\\
\begin{tt}
[ bonds ]\\
;  ai    aj funct           c0           c1\\
    6     7     1 1.000000e-01 3.138000e+05 \\
    1     2     1 1.360000e-01 4.184000e+05 \\
    1     3     1 1.360000e-01 4.184000e+05 \\
    1     4     1 1.360000e-01 4.184000e+05 \\
    1     5     1 1.530000e-01 3.347000e+05 \\
    5     6     1 1.430000e-01 3.347000e+05 \\
\\
...\\
\end{tt}}

\subsection{Topologies for free energy calculations}
\index{free energy topologies}
Free energy differences between two systems A and B can be calculated as
described in \secref{fecalc}.
The systems A and B are described by topologies
consisting of the same number of molecules with the same number of
atoms. Masses and non-bonded interactions can be perturbed by adding B
parameters in the {\tt [~atoms~]} field. Bonded interactions can be 
perturbed by adding B parameters to the bonded types or the bonded
interactions. The parameters that can be perturbed are listed in  
Tables \ref{tab:topfile1}, \ref{tab:topfile2} and \ref{tab:topfile3}.
The $\lambda$-dependence of the interactions is described
in section \secref{feia}.
Which bonded parameters are used, the one on the line of the bonded
interaction definition, or the ones looked up on atom types
in the bonded type lists, is explained in \tabref{topfe}.
In most cases things should work intuitively.
When the A and B atom types in a bonded interaction
are not all identical and parameters are not present for the B-state,
either on the line on in the bonded types,
{\tt grompp} uses the A-state parameters and issues a warning.

\begin{table}
\centerline{
\begin{tabular}{|c|cc|cc|cc|c|}
\dline
B-state atom types & \multicolumn{2}{c|}{parameters} & \multicolumn{4}{c|}{parameters in bonded types} & \\
all indentical to      & \multicolumn{2}{c|}{on line} & \multicolumn{2}{c|}{A atom types} & \multicolumn{2}{c|}{B atom types} & message \\
A-state atom types & A & B & A & B & A & B & \\
\dline
    & +AB & $-$ &  x  &  x  &     &     & \\
    & +A  & +B  &  x  &  x  &     &     & \\
yes & $-$ & $-$ & $-$ & $-$ &     &     & error \\
    & $-$ & $-$ & +AB & $-$ &     &     & \\
    & $-$ & $-$ & +A  & +B  &     &     & \\
\hline
    & +AB & $-$ &  x  &  x  &  x  &  x  & warning \\
    & +A  & +B  &  x  &  x  &  x  &  x  & \\
    & $-$ & $-$ & $-$ & $-$ &  x  &  x  & error \\
no  & $-$ & $-$ & +AB & $-$ & $-$ & $-$ & warning \\
    & $-$ & $-$ & +A  & +B  & $-$ & $-$ & warning \\
    & $-$ & $-$ & +A  &  x  & +B  & $-$ & \\
    & $-$ & $-$ & +A  &  x  &  +  & +B  & \\
\dline
\end{tabular}
}
\caption{The bonded parameters that are used for free energy topologies,
on the line of the bonded interaction definition or looked up
in the bond types section based on atom types. A and B indicate the
parameters used for state A and B respectively, + and $-$ indicate
the (non-)presence of parameters in the topology, x indicates that
the presence has no influence.}
\label{tab:topfe}
\end{table}

Below is an example of a topology which changes from 200 propanols to
200 pentanes using the \gromosv{96} force field.

\begin{tt}
; Include forcefield parameters\\
\#include "ffG43a1.itp"\\
\end{tt}\\
\begin{tt}
[ moleculetype ]\\
; Name            nrexcl\\
PropPent          3\\
\end{tt}\\
\begin{tt}
[ atoms ]\\
; nr type resnr residue atom cgnr  charge    mass  typeB chargeB  massB\\
  1    H    1     PROP    PH    1   0.398    1.008  CH3     0.0  15.035\\
  2   OA    1     PROP    PO    1  -0.548  15.9994  CH2     0.0  14.027\\
  3  CH2    1     PROP   PC1    1   0.150   14.027  CH2     0.0  14.027\\
  4  CH2    1     PROP   PC2    2   0.000   14.027\\
  5  CH3    1     PROP   PC3    2   0.000   15.035\\
\end{tt}\\
\begin{tt}
[ bonds ]\\
;  ai    aj funct    par\_A  par\_B \\
    1     2     2    gb\_1   gb\_26\\
    2     3     2    gb\_17  gb\_26\\
    3     4     2    gb\_26  gb\_26\\
    4     5     2    gb\_26\\
\end{tt}\\
\begin{tt}
[ pairs ]\\
;  ai    aj funct\\
    1     4     1\\
    2     5     1\\
\end{tt}\\
\begin{tt}
[ angles ]\\
;  ai    aj    ak funct    par\_A   par\_B\\
    1     2     3     2    ga\_11   ga\_14\\
    2     3     4     2    ga\_14   ga\_14\\
    3     4     5     2    ga\_14   ga\_14\\
\end{tt}\\
\begin{tt}
[ dihedrals ]\\
;  ai    aj    ak    al funct    par\_A   par\_B\\
    1     2     3     4     1    gd\_12   gd\_17\\
    2     3     4     5     1    gd\_17   gd\_17\\
\end{tt}\\
\begin{tt}
[ system ]\\
; Name\\
Propanol to Pentane\\
4\end{tt}\\
\begin{tt}
[ molecules ]\\
; Compound        \#mols\\
PropPent            200\\
\end{tt}

Atoms that are not perturbed, {\tt PC2} and {\tt PC3}, do not need B parameter
specifications, the B parameters will be copied from the A parameters.
Bonded interactions between atoms that are not perturbed do not need B
parameter specifications, here this is the case for the last bond.
Topologies using the OPLS/AA force field need no bonded parameters at all,
since both the A and B parameters are determined by the atom types.
Non-bonded interactions involving one or two perturbed atoms use the 
free-energy perturbation functional forms.
Non-bonded interaction between two non-perturbed atoms use the normal
functional forms.
This means that when, for instance, only the charge of a particle is
perturbed, its Lennard-Jones interactions will also be affected when
lambda is not equal to zero or one.

Note that this topology uses the \gromosv{96} force field, in which the bonded
interactions are not determined by the atom types. The bonded interaction
strings are converted by the C-preprocessor. The force field parameter
files contain lines like:

\begin{tt}
\#define gb\_26       0.1530  7.1500e+06\\
\\
\#define gd\_17     0.000       5.86          3\\
\end{tt}

\subsection{\swapindex{Constraint}{force}}
\label{subsec:constraintforce}
The constraint force between two atoms in one molecule can be calculated
with the free energy perturbation code by adding a constraint between the
two atoms, with a different length in the A and B topology. When the B length
is 1 nanometer longer than the A length and lambda is kept constant at zero,
the derivative of the Hamiltonian with respect to lambda is the constraint
force. For constraints between molecules the pull code can be used,
see \secref{pull}.
Below is an example for calculating the constraint force at 0.7 nanometer
between two methanes in water, by combining the two methanes into one molecule.
The added constraint is of function type 2, which means that it is not
used for generating exclusions (see~\secref{excl}).

\begin{tt}
; Include forcefield parameters\\
\#include "ffG43a1.itp"\\
\end{tt}\\
\begin{tt}
[ moleculetype ]\\
; Name            nrexcl\\
Methanes               1\\
\end{tt}\\
\begin{tt}
[ atoms ]\\
; nr   type   resnr  residu   atom    cgnr     charge    mass\\
   1    CH4     1     CH4      C1       1          0    16.043\\
   2    CH4     1     CH4      C2       2          0    16.043\\
\end{tt}\\
\begin{tt}
[ constraints ]\\
;  ai    aj funct   length\_A  length\_B\\
    1     2     2        0.7       1.7\\
\\
\#include "spc.itp"\\
\end{tt}\\
\begin{tt}
[ system ]\\
; Name\\
Methanes in Water\\
\end{tt}\\
\begin{tt}
[ molecules ]\\
; Compound        \#mols\\
Methanes              1\\
SOL                2002\\
\end{tt}

\subsection{Coordinate file}
\label{subsec:grofile}
Files with the {\tt .gro} file extension contain a molecular structure in 
\gromos{87} format. A sample piece is included below:

\begin{tt}
MD of 2 waters, reformat step, PA aug-91\\
    6\\
    1WATER  OW1    1   0.126   1.624   1.679  0.1227 -0.0580  0.0434\\
    1WATER  HW2    2   0.190   1.661   1.747  0.8085  0.3191 -0.7791\\
    1WATER  HW3    3   0.177   1.568   1.613 -0.9045 -2.6469  1.3180\\
    2WATER  OW1    4   1.275   0.053   0.622  0.2519  0.3140 -0.1734\\
    2WATER  HW2    5   1.337   0.002   0.680 -1.0641 -1.1349  0.0257\\
    2WATER  HW3    6   1.326   0.120   0.568  1.9427 -0.8216 -0.0244\\
   1.82060   1.82060   1.82060\\
\end{tt}

This format is fixed, {\ie} all columns are in a fixed position. If you
want to read such a file in your own program without using the
{\gromacs} libraries you can use the following formats:

{\bf C-format:} \verb'"%5i%5s%5s%5i%8.3f%8.3f%8.3f%8.4f%8.4f%8.4f"'

Or to be more precise, with title {\em etc.} it looks like this:

\begin{tt}
  "%s\n", Title\\
  "%5d\n", natoms\\
  for (i=0; (i<natoms); i++) {\\
    "%5d%5s%5s%5d%8.3f%8.3f%8.3f%8.4f%8.4f%8.4f\n",\\
      residuenr,residuename,atomname,atomnr,x,y,z,vx,vy,vz\\
  }\\
  "%10.5f%10.5f%10.5f%10.5f%10.5f%10.5f%10.5f%10.5f%10.5f\n",\\
    box[X][X],box[Y][Y],box[Z][Z],\\
    box[X][Y],box[X][Z],box[Y][X],box[Y][Z],box[Z][X],box[Z][Y]\\
\end{tt}

{\bf Fortran format:} {\tt (i5,2a5,i5,3f8.3,3f8.4)}

So {\tt confin.gro} is the {\gromacs} coordinate file and is almost
the same as the \gromosv{87} file (for {\gromos} users: when used with
ntx=7).  The only difference is the box for which {\gromacs} uses a
tensor, not a vector.

\section{Force-field organization \index{force-field organization}}
\label{sec:fforganization}

\subsection{Force-field files}
\label{subsec:fffiles}
{\gromacs} {\gmxver} includes five forcefields.
They are listed the file {\tt FF.dat}:

\begin{tt}
5\\
ffgmx   Gromacs Forcefield (see manual)\\
ffgmx2  Gromacs Forcefield with all hydrogens (proteins only)\\
ffG43a1 GROMOS96 43a1 Forcefield (official distribution)\\
ffG43b1 GROMOS96 43b1 Vacuum Forcefield (official distribution)\\
ffG43a2 GROMOS96 43a2 Forcefield (development) (improved ...)\\
\end{tt}

All files for each force field have names beginning with the {\tt ff???}
string in the {\tt FF.dat} file.
A force field is included at the beginning of a topology file with an
{\tt \#include} statement followed by {\tt ff???.itp}.
This statement includes the force-field file,
which in turn may include other forcefield files. A the five force fields
are organized in the same way. As an example we show the {\tt ffgmx.itp}
force-field file:

\begin{tt}
\#define \_FF\_GROMACS\\
\#define \_FF\_GROMACS1\\
\end{tt}\\
\begin{tt}
[ defaults ]\\
; nbfunc        comb-rule       gen-pairs       fudgeLJ fudgeQQ\\
  1             1               no              1.0     1.0\\

\#include "ffgmxnb.itp"\\
\#include "ffgmxbon.itp"\\
\end{tt}

The first {\tt \#define} can be used in topologies to parse data which is
specific for all {\gromacs} force-fields, the second {\tt \#define} to parse
data which is specific for this force field. The {\tt defaults} section is
explained in \ssecref{topfile}. The included file {\tt ffgmxnb.itp} contains
all atom types and non-bonded parameters. The included file {\tt ffgmxbon.itp}
contains all bonded parameters.

For each force field there a five files which are only used by {\tt pdb2gmx}.
These are: the residue database ({\tt .rtp}, see \ssecref{rtp})
the hydrogen database ({\tt .hdb}, see \ssecref{hdb}), two termini databases
({\tt .tdb}, see \ssecref{tdb}) and
the atom type database ({\tt .atp}) which contains only the masses.


\subsection{Changing force-field parameters
\index{force-field, changing parameters}}
If one wants to change the parameters of few bonded interactions in
a molcule, this is most easily accomplished by typing the parameters
behind the definition of the bonded interaction in the
{\tt [ moleculetype ]} section (see \ssecref{topfile} for the format
and units).
If one wants to change the parameters for all instances of a certain
interaction one can change them in the force-field file or add a
new {\tt [ ???types ]} section after including the force field.
When parameters for a certain interaction are defined multiple times
the last definition is used. As of {\gromacs} version 3.1.3 a warning is
generated when parameters are redefined with a different value.
Changing the Lennard-Jones parameters of an atom type is not
recommended, because in the {\gromacs} and {\gromos} force-fields
the Lennard-Jones parameters for several combinations of atom types
are not according to the standard combination rules.
Such combinations (and possibly also combinations that do follow the
combionation rules) are defined in the {\tt [ nonbonded\_params ]}
section and changing the Lennard-Jones parameters of an atom type
has no effect on these combinations.

\subsection{\swapindex{Adding}{atom types}}
As of {\gromacs} version 3.1.3 atom types can be added in an extra
{\tt [ atomtypes ]} section after the the inclusing of the normal
forcefield. After the definition of the new atom type(s), additional
non-bonded and pair parameters can be defined.
In pre 3.1.3 versions of {\gromacs} the new atom types needed to be
added in the {\tt [ atomtypes ]} section of the forcefield files,
because all non-bonded parameters above the last {\tt [ atomtypes ]}
section would be overwritten using the standard combination rules.
