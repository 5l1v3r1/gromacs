\chapter{Creating a Topology}
\label{ch:top}
\section{Introduction}
{\gromacs} must know on which atoms and combinations of atoms the
various contributions to the potential functions (see
chapter~\ref{ch:ff}) must act. It must  
also know what \normindex{parameters} must be applied to the various
functions. All this is described in the {\em \normindex{topology}} file
\verb'*.top', which lists the {\em constant attributes} of each atom.
There are many more atom
types than elements, but only atom types present in biological 
systems are parametrized in the force field, plus some metals, ions and 
silicon. The bonded and special interactions are determined by fixed
lists that are included in the topology file. Certain non-bonded
interactions must be excluded (first and second neighbours), as these
are already treated in bonded interactions.  
In addition there are {\em dynamic attributes} of atoms: their
positions, velocities and forces, but these do not strictly belong to
the molecular topology.  

This Chapter describes the set up of the topology file, the
\verb'*.top' file: what the parameters stand for and how/where to
change them if needed. 

{\bf Note:} if you have constructed your own \verb'*.top', please
send a copy plus description to:
\centerline{\bf gromacs@chem.rug.nl}

so we can extend our topology database and prevent {\gromacs} users
from ``inventing the wheel twice''. This also applies for {\em new
force field parameters} that were originally not included in the
{\gromacs} force field.

\section{Particle type}
\label{sec:parttype}

In {\gromacs} there are 4 types of \normindex{particles}.  Only the
first type is used in {\gromacs} {\gmxver}, the others are reserved
for future versions. which are listed in \verb'ffgmxbon.itp' ({\bf
f}orce {\bf f}ield {\gromacs} {\bf bon}ded and .itp stands for {\em
include topology parameter file}). See Table~\ref{ta:ptype}.

\begin{table}
\centerline{
\begin{tabular}{|l|c|}
\dline
Particle	& Symbol	\\
\hline
\seeindex{atom}{particles}	& A   \\
\seeindex{nucleus}{particles}	& N   \\
\seeindex{shell}{particles}     & S   \\
\seeindex{bond shell}{particles}& B   \\
\seeindex{dummy}{particles}     & D   \\
\dline
\end{tabular}
}
\caption{Particle types in Gromacs}
\label{ta:ptype}
\end{table}

\subsection{Atom types}
\label{subsec:atomtype}
{\gromacs} uses 47 different \normindex{atom types}, 
as listed below, with their
corresponding masses (in a.m.u.). This is the same listing as in the
file \verb'ffgmx.atp' (.atp = {\bf a}tom {\bf t}ype {\bf p}arameter
file), therefore in this file you can change and/or add an atom type.
{\small\begin{verbatim}
    O  15.99940 ;     CARBONYL OXYGEN (C=O)              
   OM  15.99940 ;     CARBOXYL OXYGEN (CO-)              
   OA  15.99940 ;     HYDROXYL OXYGEN (OH)               
   OW  15.99940 ;     WATER OXYGEN                       
    N  14.00670 ;     PEPTIDE NITROGEN (N OR NH)         
   NT  14.00670 ;     TERMINAL NITROGEN (NH2)            
   NL  14.00670 ;     TERMINAL NITROGEN (NH3)            
  NR5  14.00670 ;     AROMATIC N (5-RING,2 BONDS)        
 NR5*  14.00670 ;     AROMATIC N (5-RING,3 BONDS)        
   NP  14.00670 ;     PORPHYRIN NITROGEN                 
    C  12.01100 ;     BARE CARBON (PEPTIDE,C=O,C-N)      
  CH1  13.01900 ;     ALIPHATIC CH-GROUP                 
  CH2  14.02700 ;     ALIPHATIC CH2-GROUP                
  CH3  15.03500 ;     ALIPHATIC CH3-GROUP                
 CR51  13.01900 ;     AROMATIC CH-GROUP (5-RING), united 
 CR61  13.01900 ;     AROMATIC CH-GROUP (6-RING), united 
   CB  12.01100 ;     BARE CARBON (5-,6-RING)            
    H   1.00800 ;     HYDROGEN BONDED TO NITROGEN        
   HO   1.00800 ;     HYDROXYL HYDROGEN                  
   HW   1.00800 ;     WATER HYDROGEN                     
   HS   1.00800 ;     HYDROGEN BONDED TO SULFUR          
    S  32.06000 ;     SULFUR                             
   FE  55.84700 ;     IRON                               
   ZN  65.37000 ;     ZINC                               
   NZ  14.00670 ;     ARG NH (NH2)                       
   NE  14.00670 ;     ARG NE (NH)                        
    P  30.97380 ;     PHOSPHOR                           
   OS  15.99940 ;     SUGAR OR ESTER OXYGEN              
  CS1  13.01900 ;     SUGAR CH-GROUP                     
  NR6  14.00670 ;     AROMATIC N (6-RING,2 BONDS)        
 NR6*  14.00670 ;     AROMATIC N (6-RING,3 BONDS)        
  CS2  14.02700 ;     SUGAR CH2-GROUP                    
   SI  28.08000 ;     SILICON                            
   NA  22.98980 ;     SODIUM (1+)                        
   CL  35.45300 ;     CHLORINE (1-)                      
   CA  40.08000 ;     CALCIUM (2+)                       
   MG  24.30500 ;     MAGNESIUM (2+)                     
    F  18.99840 ;     FLUORINE (COV. BOUND)              
  CP2  14.02700 ;     ALIPHATIC CH2-GROUP USING Ryckaert-Bell.
  CP3  15.03500 ;     ALIPHATIC CH3-GROUP USING Ryckaert-Bell.
  CR5  12.01100 ;     AROMATIC CH-GROUP (5-RING)+H       
  CR6  12.01100 ;     AROMATIC C- bonded to H (6-RING)+H 
  HCR   1.00800 ;     H attached to aromatic C (5 or 6 ri
 OWT3  15.99940 ;     TIP3P WATER OXYGEN                 
   SD  32.06000 ;     DMSO Sulphur                       
   OD  15.99940 ;     DMSO Oxygen                        
   CD  15.03500 ;     DMSO Carbon                        
\end{verbatim}}
Atomic detail is used except for hydrogen atoms bound to (aliphatic)
carbon atoms, which are treated as {\em \normindex{united atoms}}. No
special \normindex{hydrogen-bond} term is included.

The last 10 atomtypes are extra atomtypes with respect to the GROMOS87
force field~\cite{biomos}: 
\begin{itemize}
\item F was taken from ref.~\cite{Buuren93a}, 
\item CP2 and CP3 from ref.~\cite{Buuren93b} and references cited therein, 
\item CR5, CR6 and HCR from ref.~\cite{King93}
\item OWT3 from ref.~\cite{Jorgensen83}
\item SD, OD and CD from ref.~\cite{Liu95}
\end{itemize}
{\bf Therefore, if you use the {\gromacs} force field as it is, make
sure you use the references in your publications as mentioned above.}

{\bf Note:} {\gromacs} makes use of the atom types as a name, {\em
not} as a number (as e.g. in GROMOS).

\subsection{Nucleus}
{\em Necessary for \normindex{polarisability}, not implemented yet.}

\subsection{Shell}
{\em Necessary for polarisability, not implemented yet.}

\subsection{Bond shell}
{\em Necessary for polarisability, not implemented yet.}

\subsection{Dummy atoms}
\label{sec:dummytop}
Some \normindex{force field}s use \normindex{dummy atom}s 
(\normindex{virtual site}s that are constructed
from real atoms) on which certain interaction functions are located
(e.g. on benzene rings, to reproduce the correct
\normindex{quadrupole}). This is described in~\secref{dummy}.

To make dummy atoms in your system, you should include a section 
\verb'[ dummies? ]' in your topology file, where the `\verb|?|' stands for
the number constructing atoms for the dummy atom. This will be
`\verb|2|' for type 2, `\verb|3|' for types 3, 3fd, 3fad and 3out and
`\verb|4|' for type 4fd (the different types are explained
in~\secref{dummy}).

Parameters for type 2 should look like this:
{\small\begin{verbatim}
[ dummies2 ] 
; Dummy from          funct  a 
5       1      2      1      0.7439756
\end{verbatim}}

for type 3 like this:
{\small\begin{verbatim}
[ dummies3 ]
; Dummy from                 funct   a          b
5       1      2      3      1       0.7439756  0.128012
\end{verbatim}}

for type 3fd like this:
{\small\begin{verbatim}
[ dummies3 ]
; Dummy from                 funct   a          d
5       1      2      3      2       0.5        -0.105
\end{verbatim}}

for type 3fad like this:
{\small\begin{verbatim}
[ dummies3 ]
; Dummy from                 funct   d          theta
5       1      2      3      3       0.5        120
\end{verbatim}}

for type 3out like this:
{\small\begin{verbatim}
[ dummies3 ]
; Dummy from                 funct   a          b          c
5       1      2      3      4       -0.4       -0.4       6.9281
\end{verbatim}}

for type 4fd like this:
{\small\begin{verbatim}
[ dummies4 ]
; Dummy from                         funct   a          b          d
5       1      2      3      4       1       0.33333    0.33333    -0.105
\end{verbatim}}

This will result in the construction of a dummy `atom', number 5
(first column `\verb'Dummy''), based on the positions of 1 and 2 or 1,
2 and 3 or 1, 2, 3 and 4 (next two, three or four columns
`\verb'from'') following the rules determined by the function number
(next column `\verb'funct'') with the parameters specified (last one,
two or three columns `\verb'a b' . .').

Note that any bonds defined between dummy atoms and/or normal atoms
will be removed by {\tt grompp} after the exclusions have been
generated. This way, exclusions will not be affected by an atom being
defined as dummy atom or not, but by the bonding configuration of the
atom.

\section{Parameter files}
\label{sec:paramfiles}
\subsection{Atoms}
A number of {\em static} properties are assigned to the atom types in the 
{\gromacs} force field: Type, Mass, Charge, $\epsilon$ and $\sigma$
(see Table~\ref{tab:statprop}
The mass is listed in \verb'ffgmx.atp' (see~\ref{subsec:atomtype}), 
whereas the charge is listed
in \verb'ffgmx.rtp' (.rtp = {\bf r}esidue {\bf t}opology {\bf
p}arameter file, see~\ref{subsec:rtp}). 
This implies that the charges are only defined
in the \normindex{building block}s 
of amino acids or user defined building blocks.
When generating a topology ({\tt *.top}) using the {\tt \normindex{pdb2gmx}} 
program the information from these files is combined.
 
\begin{table}[h]
\centerline{
\begin{tabular}{|l|c|c|}
\dline
Property		& Symbol	& Unit		\\
\hline
Type			& -		& -		\\
Mass 			& m		& a.m.u.	\\
Charge 			& q		& electron	\\
epsilon                 & $\epsilon$    & kJ/mole       \\
sigma                   & $\sigma$      & nm            \\
\dline
\end{tabular}
}
\caption{Static atomtype properties in Gromacs}
\label{tab:statprop}
\end{table}

The following {\em dynamic} quantities are associated with an atom
\begin{itemize}
\item	Position {\bf x}
\item	Velocity {\bf v}
\end{itemize}
These quantities are listed in the coordinate file, \verb'*.gro'
(see section File format,~\ref{subsec:grofile}).

\subsection{Bonded parameters}
\label{subsec:bondparam}
The \normindex{bonded parameter}s (ie. bonds, angles, improper and proper
dihedrals) are listed in \verb'ffgmxbon.itp'. The term {\tt func} can
be ignored in {\gromacs} {\gmxver}, because for bonds and angles we only use 1
function, so far. For the dihedral, this is explained after this listing.
{\small\begin{verbatim}
[ bondtypes ]
  ; i    j func        b0          kb
    C    O    1   0.12300     502080.
    C   OM    1   0.12500     418400.
    ......

[ angletypes ]
  ; i    j    k func       th0         cth
   HO   OA    C    1   109.500     397.480
   HO   OA  CH1    1   109.500     397.480
   ......

[ dihedraltypes ]
  ; i    l func        q0          cq
 NR5*  NR5    2     0.000     167.360
 NR5* NR5*    2     0.000     167.360
 ......

[ dihedraltypes ]
  ; j    k func      phi0          cp   mult
    C   OA    1   180.000      16.736      2
    C    N    1   180.000      33.472      2
    ......

[ dihedraltypes ]
;
; Ryckaert-Bellemans Dihedrals
;
; aj    ak      funct
CP2     CP2     3       9.2789  12.156  -13.120 -3.0597 26.240  -31.495
\end{verbatim}}
Also in this file are the \normindex{Ryckaert-Bellemans}~\cite{Ryckaert78} 
parameters for
the CP2-CP2 dihedrals in alkanes or alkane tails with the following
constants:

\begin{center}
(kJ/mol)\\
\begin{tabular}{llrllrllr}
$C_0$ & $=$ & $~ 9.28$ & $C_2$ & $=$ & $-13.12$ & $C_4$ & $=$ & $ 26.24$ \\
$C_1$ & $=$ & $ 12.16$ & $C_3$ & $=$ & $~-3.06$ & $C_5$ & $=$ & $-31.5 $ \\
\end{tabular}
\end{center}

({\bf Note:} The use of this potential implies exclusions of LJ-interactions
between the first and the last atom of the dihedral, and $\psi$ is defined
according to the '\normindex{polymer convention}' ($\psi_{trans}=0$)).

So there are three types of dihedrals in the {\gromacs} force field:
\begin{itemize}
\item \normindex{proper dihedral} : funct = 1, with mult = multiplicity, so the
                                   number of possible angles
\item \normindex{improper dihedral} : funct = 2
\item Ryckaert-Bellemans dihedral : funct = 3
\end{itemize}
In the file \verb'ffgmxbon.itp' you can add bonded parameters. If you
want to include parameters for new atom types, make sure you define
this new atom type in \verb'ffgmx.atp' as well.

\subsection{Non-bonded parameters}
\label{subsec:nbpar}
The non-bonded parameters consist of the Van der Waals parameters
$A$ and $C$, as listed in \verb'ffgmxnb.itp', where {\tt ptype} is the
particle type (see Table~\ref{ta:ptype}):
{\small\begin{verbatim}
[ atomtypes ]
;name        mass      charge   ptype            c6           c12
    O    15.99940       0.000       A   0.22617E-02   0.74158E-06
   OM    15.99940       0.000       A   0.22617E-02   0.74158E-06
   .....

[ nonbond_params ]
  ; i    j func          c6           c12
    O    O    1 0.22617E-02   0.74158E-06
    O   OA    1 0.22617E-02   0.13807E-05
    .....

[ pairtypes ]
  ; i    j func         cs6          cs12    ; THESE ARE 1-4 INTERACTIONS
    O    O    1 0.22617E-02   0.74158E-06
    O   OM    1 0.22617E-02   0.74158E-06
    .....
\end{verbatim}}
With $A$ and $C$ being defined as
\beq
	A_{ii} = 4\epsilon_i\sigma_i^{12}
\eeq
\beq
	C_{ii} = 4\epsilon_i\sigma_i^{6}
\eeq
and computed according to the \normindex{combination rules} :
\beq
	A_{ij} = (A_{ii}A_{jj})^{\frac{1}{2}}
\eeq
\beq
	C_{ij} = (C_{ii}C_{jj})^{\frac{1}{2}}
\eeq
It is also possible to use the combination rules based on the
Lennard-Jones parameters $\epsilon$ and $\sigma$ with : 
\beq
 	\sigma_{ij} = \frac{1}{2}(\sigma_{ii}+\sigma_{jj})
\eeq
\beq
 	\epsilon_{ij}=\sqrt{\epsilon_{ii} \epsilon_{jj}}
\eeq
This is useful if you want to use for example the
\normindex{OPLS}~\cite{Jorgensen88} force field, as described in
section~\ref{sec:otherff}. We note however, that is not yet possible
to use this in {\gromacs} {\gmxver}.

\subsection{Exclusions and 1-4 interaction}
The \normindex{exclusions} (first and second neighbours) are generated by
\verb'pdb2gmx' so no worries on that point.

The \normindex{1-4 interactions} are also listed for the 47 atom types in
\verb'ffgmxnb.itp' under [~pairtypes~]. It is possible to change them there
of course, or add new parameters for different/new atom types.

\subsection{Residue database}
\label{subsec:rtp}
The file holding the residue database is {\tt ffgmx.rtp}. Originally
this file contained building blocks (amino acids) for proteins, and is
the {\gromacs} interpretation of the {\tt rt37c4.dat} file of GROMOS. So
the residue file contains information (bonds, charge, charge groups
and improper dihedrals) for a frequently used building block. It is
better {\em not} to change this file because it is standard input for
\verb'pdb2gmx', but if changes are needed make them in the
\verb'*.top' file (see section Topology file, ~\ref{subsec:topfile}). 
However, in the {\tt ffgmx.rtp} file the user can define a new
\normindex{building block} or molecule: see for example 2,2,2-trifluoroethanol
(TFE) or {\em n}-decane (C10). But when defining new molecules
(non-protein) it is preferable to create a {\tt *.itp}
file. This will be discussed in a next section (section~\ref{subsec:molitp}).

The file {\tt ffgmx.rtp} is only used by \verb'pdb2gmx'.
As mentioned before, the only extra information this
program needs from {\tt ffgmx.rtp} is bonds, charges of atoms,
charge groups and improper dihedrals, because the rest is read from
the coordinate input file (in the case of \verb'pdb2gmx', a pdb format
file). Some proteins contain residues that are not standard, but are
listed in the coordinate file. You have to construct a building block
for this ``strange'' residue, otherwise you will not obtain a
\verb'*.top' file. This also holds for molecules in the
coordinate file like phosphate or sulphate ions.
The residue database is constructed in the following way:
{\small\begin{verbatim}
[ bondedtypes ]  ; mandatory
; bonds  angles  dihedrals  impropers
     1       1          1          2  ; mandatory

[ GLY ]  ; mandatory

 [ atoms ]  ; mandatory 
; name  type  charge  chargegroup 
     N     N  -0.280     0
     H     H   0.280     0
    CA   CH2   0.000     1
     C     C   0.380     2
     O     O  -0.380     2

 [ bonds ]  ; optional
;atom1 atom2      b0      kb
     N     H
     N    CA
    CA     C
     C     O
    -C     N

 [ angles ]  ; optional
;atom1 atom2 atom3    th0    cth

 [ dihedrals ]  ; optional
;atom1 atom2 atom3 atom4   phi0     cp   mult

 [ impropers ]  ; optional
;atom1 atom2 atom3 atom4     q0     cq
     N    -C    CA     H
    -C   -CA     N    -O


[ ZN ]
 [ atoms ]
    ZN    ZN   2.000     0
\end{verbatim}}

The file is free format, the only restriction is that there can be at most
one entry on a line.
The first field in the file is the {\tt [bondedtypes]} field, which is
followed by four numbers, that indicate the interaction type for bonds,
angles, dihedrals and improper dihedrals.
The file contains residue entries, which consist of atoms and optionally 
bonds, angles dihedrals and impropers.
The charge group codes denote the charge group numbers.
Atoms in the same charge group should always be below each other.
The atom names in the bonded interaction can be preceded by a minus
or a plus, indicating that the atom is in the preceding or following
residue respectively.
Parameters can be added to bonds, angles, dihedrals and impropers,
these parameters override the standard parameters in the {\tt .itp} files.
This should only be used in special cases. Instead of parameters, a
string can be added for each bonded interaction, this is used in
GROMOS96 {\tt .rtp} files. These strings are copied to the topolgy file
and can be replaced by force field parameters by the C-preprocessor in
{\tt grompp} using {\tt \#define} statements.

{\tt pdb2gmx} automatically generates all angles, this means that the
{\tt [angles]} field is only useful for overriding {\tt .itp} parameters. 

{\tt pdb2gmx} automatically generates one proper dihedral for every rotatable
bond, preferably on heavy atoms. When the {\tt [dihedrals]} field is used,
no other dihedrals will be generated for the bonds corresponding to the
specified  dihedrals. It is possible to put more than one dihedral on a
rotatable bond. 

\subsection{Hydrogen and termini database}
The \myindex{hydrogen database} is stored in {\tt ffgmx.hdb}.
The \myindex{termini database}s are stored in {\tt ffgmx-n.tdb}
and {\tt ffgmx-c.tdb} for the N- and C-terminus respectively.
They contain information for the {\tt pdb2gmx} program on how to connect
new atoms to existing ones. The termini database files also contain
information on which atoms should be removed or changed and which
(improper) dihedral angles should be added.

{\small\begin{verbatim}
; res   # additions
        # H add type    i       j       k
ALA     1
        1       1       N       -C      CA
ARG     4
        1       2       N       CA      C
        1       1       NE      CD      CZ
        2       3       NH1     CZ      NE
        2       3       NH2     CZ      NE
\end{verbatim}
}

On the first line we see the residue name (ARG) and the number of additions.
On each subsequent line we then see 
\begin{itemize}
\item The number of H atoms added
\item The way of adding H atoms, can be any of
\begin{enumerate}
\item[1]{\em one planar hydrogen, e.g. rings or peptide bond}\\
one hydrogen atom (n) is generated, lying in the plane of atoms
(i,j,k) on the line bisecting angle (j-i-k) at a distance of 0.1 nm
from atom i, such that the angles (n-i-j) and (n-i-k) are $>$ 90
degrees
\item[2]{\em one single hydrogen, e.g. hydroxyl}\\
one hydrogen atom (n) is generated at a distance of 0.1 nm from atom
i, such that angle (n-i-j)=109.5 degrees and dihedral (n-i-j-k)=trans
\item[3]{\em two planar hydrogens, e.g. -NH{$_2$}}\\
two hydrogens (n1,n2) are generated at a distance of 0.1 nm from atom
i, such that angle (n1-i-j)=(n2-i-j)=120 degrees and dihedral
(n1-i-j-k)=cis and (n2-i-j-k)=trans, such that names are according to
IUPAC standards~\cite{iupac70}
\item[4]{\em two or three tetrahedral hydrogens, e.g. -CH{$_3$}}\\
three (n1,n2,n3) or two (n1,n2) hydrogens are generated at a distance
of 0.1 nm from atom i, such that angle
(n1-i-j)=(n2-i-j)=(n3-i-j)=109.5, dihedral (n1-i-j-k)=trans,
(n2-i-j-k)=trans+120 and (n3-i-j-k)=trans+240 degrees
\item[5]{\em one tetrahedral hydrogen, e.g. C{$_3$}CH}\\
one hydrogen atom (n1) is generated at a distance of 0.1 nm from atom
i in tetrahedral conformation such that angle
(n1-i-j)=(n1-i-k)=(n1-i-l)=109.5 degrees
\item[6]{\em two tetrahedral hydrogens, e.g. C-CH{$_2$}-C}\\
two hydrogen atoms (n1,n2) are generated at a distance of 0.1 nm from
atom i in tetrahedral conformation on the plane bissecting angle i-j-k
with angle (n1-i-n2)=(n1-i-j)=(n1-i-k)=109.5
\item[7]{\em two water hydrogens}\\
two hydrogens are generated around atom i according to GROMOS water
geometry. The symmetry axis will alternate between three coordinate
axes in both directions
\item[8]{\em two carboxyl oxygens, -COO{$^-$}}\\
two oxygens (n1,n2) are generated according to rule 3, at a distance
of 0.136 nm from atom i and an angle (n1-i-j)=(n2-i-j)=117 degrees
\item[9]{\em carboxyl oxygens and hydrogen, -COOH}\\
two oxygens (n1,n2) are generatd according to rule 3, at distances of
0.123 nm and 0.125 nm from atom i for n1 and n2 resp. and angles
(n1-i-j)=121 and (n2-i-j)=115 degrees. One hydrogen (n') is generated
around n2 according to rule 2, where n-i-j and n-i-j-k should be read
as n'-n2-i and n'-n2-i-j resp.
\end{enumerate}
\item
Three or four control atoms (i,j,k,l), where the first always is the
atom to which the H atoms are connected. The other two or three depend
on the code selected.
\end{itemize}

The program code to \normindex{generate hydrogen} positions were taken
from the Gromos Package~\cite{biomos} and extended for options 5
through 9. Note that option 9 is designed specifically to add a COOH
terminus. The others are in principle generic, i.e. they can be used
to add any type of atom, except for the bondlength and angles which
are fixed (as specified above). Note however that only the termini
database supports type specification for the atoms, the hydrogen
database will (naturally) always add hydrogens.

\section{File formats}
\subsection{\normindex{Topology file}}
\label{subsec:topfile}
The topology file is built following the {\gromacs} specification for a
molecular topology.  A \verb'*.top' file can be generated by
\verb'pdb2gmx'.

Description of the file layout:
\begin{itemize}
\item semicolon (;) and newline surround comments
\item on a line ending with $\backslash$ the newline character is ignored.
\item directives are surrounded by \verb'[' and \verb']'
\item the topology consists of three levels:
\begin{itemize}
\item the parameter level (see Table~\ref{ta:topfile1})
\item the molecule level, which should contain one or more molecule
      definitions (see Table~\ref{ta:topfile2})
\item the system level: \verb'[ system ]', \verb'[ molecules ]'
\end{itemize}
\item items should be seperated by spaces or tabs, not commas
\item atoms in molecules should be numbered consecutively starting at 1
\item the file is parsed once only which implies that no forward
      references can be treated: items must be defined before they
      can be used
\item exclusions can be generated from the bonds or
      overridden manually
\item the bonded force types can be generated from the atom types or
      overridden per bond
\item descriptive comment lines and empty lines are highly recommended
\item using one of the \verb'[ atoms ]', \verb'[ bonds ]', 
      \verb'[ pairs ]', \verb'[ angles ]', etc. without having used 
      \verb'[ moleculetype ]' 
      before is meaningless and generates a warning.
\item using \verb'[ molecules ]' without having used
      \verb'[ system ]' before is meaningless and generates a warning.
\item using an unknown string in \verb'[ ]' causes all the data until
      the next directive to be ignored, and generates a warning.
\end{itemize}

\newcommand{\kJmol}{kJ mol$^{-1}$}
\newcommand{\kJmolnm}[1]{\kJmol nm$^{#1}$}
\newcommand{\kJmolrad}[1]{\kJmol rad$^{#1}$}
\newcommand{\kJmoldeg}[1]{\kJmol deg$^{#1}$}
\begin{table}[p]
\centerline{\small\begin{tabular}{|l|lllll|}
\multicolumn{6}{c}{\bf \large Parameters} \\
\hline
interaction 	& directive   	      & \#  & f. & parameters 				& pert \\
type	&		      	      & at. & tp &					& 	\\
\dline
{\em mandatory} & {\tt defaults}	& & &	non-bonded function type; & \\
		&			& & &	combination rule; 	 &\\
		&			& & &   generate pairs (no/yes); & \\
		&			& & &	fudge LJ (); fudge QQ () & \\
\hline
{\em mandatory} & {\tt atomtypes}	&   & 	& atom type; m (u); q (e); particle type; & \\
		&			&   &	& c$_6$ (\kJmolnm{6}); c$_{12}$ (\kJmolnm{12}) & \\
\hline
		& {\tt bondtypes}	& \multicolumn{3}{l}{(see Table~\ref{ta:topfile2}, directive {\tt bonds})}		& 	\\
		& {\tt constrainttypes}	& \multicolumn{3}{l}{(see Table~\ref{ta:topfile2}, directive {\tt constraints})}		& 	\\
		& {\tt pairtypes}	& \multicolumn{3}{l}{(see Table~\ref{ta:topfile2}, directive {\tt pairs})}		& 	\\
		& {\tt angletypes}	& \multicolumn{3}{l}{(see Table~\ref{ta:topfile2}, directive {\tt angles})}		& 	\\
proper dih.	& {\tt dihedraltypes}	& 2$^{(b)}$ & 1	& $\theta_{max}$ (deg); f$_c$ (\kJmol); mult & X$^{(a)}$ \\
improper dih.	& {\tt dihedraltypes}	& 2$^{(c)}$ & 2	& $\theta_0$ (deg); f$_c$ (\kJmolrad{-2}) & X	\\
RB dihedral	& {\tt dihedraltypes}	& 2$^{(b)}$ & 3	&  C$_0$, C$_1$, C$_2$, C$_3$, C$_4$, C$_5$ (\kJmol) 		&	\\
LJ 		& {\tt nonbond\_params}	& 2 & 1	& c$_6$ (\kJmolnm{6}); c$_{12}$ (\kJmolnm{12}) & \\
Buckingham    	& {\tt nonbond\_params}	& 2 & 2	& a (\kJmol); b (nm$^{-1})$;  & \\
 & & & & c$_6$ (\kJmolnm{6}) & \\
\hline
\multicolumn{6}{c}{~} \\
\multicolumn{6}{l}{'\# at' is the number of atom types} \\
\multicolumn{6}{l}{'f. tp' is function type} \\
\multicolumn{6}{l}{'pert' indicates if this interaction type
can be modified during free energy perturbation} \\
\multicolumn{6}{l}{~$^{(a)}$ multiplicities can not be modified} \\
\multicolumn{6}{l}{~$^{(b)}$ the outer two atoms in the dihedral} \\
\multicolumn{6}{l}{~$^{(c)}$ the inner two atoms in the dihedral} \\
\multicolumn{6}{l}{For free energy perturbation, the parameters for topology 'B' (lambda = 1) should be added} \\
\multicolumn{6}{l}{on the same line, after the normal parameters,
in the same order as the normal parameters.} \\
\end{tabular}
}
\caption{The topology ({\tt *.top}) file, part 1.}
\label{ta:topfile1}
\end{table}
\cleardoublepage
\begin{table}[p]
\centerline{\small\begin{tabular}{|l|lllll|}
\multicolumn{6}{c}{\bf \large Molecule definition} \\
\hline
interaction 	& directive   	      & \#  & f. & parameters 				& pert \\
type	&		      	      & at. & tp &					& 	\\
\dline
{\em mandatory} & {\tt moleculetype}	& & & 	molecule name; &	\\
		&			& & & 	exclude neighbours \# bonds away &	\\
		&			& & & 	for non-bonded interactions & \\
\hline
{\em mandatory} & {\tt atoms}		& 1 & 	& atom type; residue number; 	& 	\\
		&			&   &	& residue name; atom name; 	& 	\\
		&			&   &	& charge group number; q (e); m (u) 	& X$^{(b)}$ \\
\hline
bond		& {\tt bonds}		& 2 & 1	& b$_0$ (nm); f$_c$ (\kJmolnm{-2})	& X	\\
G96 bond	& {\tt bonds}		& 2 & 2	& b$_0$ (nm); f$_c$ (\kJmolnm{-4})	& X	\\
morse		& {\tt bonds}		& 2 & 3	& b$_0$ (nm); D (\kJmol); $\beta$ (nm$^{-1}$) & X \\
LJ 1-4		& {\tt pairs}		& 2 & 1	& c$_6$ (\kJmolnm{6}); & \\
 & & & & c$_{12}$ (\kJmolnm{12}) & X \\
angle		& {\tt angles}		& 3 & 1	& $\theta_0$ (deg); f$_c$ (\kJmolrad{-2}) & X	\\
G96 angle	& {\tt angles}		& 3 & 2	& $\theta_0$ (deg); f$_c$ (\kJmol) & X	\\
proper dih.	& {\tt dihedrals}	& 4 & 1	& $\theta_{max}$ (deg); f$_c$ (\kJmol); mult & X$^{(a)}$	\\
improper dih.	& {\tt dihedrals}	& 4 & 2	& $\theta_0$ (deg); f$_c$ (\kJmolrad{-2}) & X	\\
RB dihedral	& {\tt dihedrals}	& 4 & 3	& C$_0$, C$_1$, C$_2$, C$_3$, C$_4$, C$_5$ (\kJmol) 		&	\\
constraint	& {\tt constraints}	& 2 & 1	& b$_0$ (nm) 				& X	\\
constr. n.c.    & {\tt constraints}	& 2 & 2	& b$_0$ (nm) 				& X	\\
settle		& {\tt settles}		& 3 & 1	& d$_{\mbox{\sc oh}}$, d$_{\mbox{\sc hh}}$ (nm) 		& 	\\
dummy2		& {\tt dummies2}	& 2 & 1	& a ()					& 	\\
dummy3		& {\tt dummies3}	& 3 & 1	& a, b ()				& 	\\
dummy3fd	& {\tt dummies3}	& 3 & 2	& a (); d (nm)				& 	\\
dummy3fad	& {\tt dummies3}	& 3 & 3	& d (nm); $\theta$ (deg) 		& 	\\
dummy3out	& {\tt dummies3}	& 3 & 4	& a, b (); c (nm$^{-1}$) 		& 	\\
dummy4fd	& {\tt dummies4}	& 4 & 1	& a, b (); d (nm);	   		& 	\\
position res.	& {\tt position\_restraints}	& 1 & 1	& k$_{x}$, k$_{y}$, k$_{z}$ (\kJmolnm{-2}) & 	\\
%wpol	& {\tt position\_restraints}	& 1 & 2	& ???	& 	\\
distance res.	& {\tt distance\_restraints}	& 2 & 1	& type; index; low, up$_1$, up$_2$ (nm); & \\
 & & & & factor () & \\
angle res.	& {\tt angle\_restraints}	& 4 & 1	& $\theta_0$ (deg); f$_c$ (\kJmol); mult & X$^{(a)}$	\\
angle res. z & {\tt angle\_restraints\_z}	& 2 & 1	& $\theta_0$ (deg); f$_c$ (\kJmol); mult & X$^{(a)}$	\\
exclusions	& {\tt exclusions}	& 1 & 	& one or more atom indices				& 	\\
\hline
\multicolumn{6}{c}{~} \\
\multicolumn{6}{c}{\bf \large System} \\
\hline
{\em mandatory} & {\tt system}		& & &	system name				&	\\
\hline
{\em mandatory} & {\tt molecules}	& & &	\multicolumn{2}{l|}{molecule name; number of molecules}	\\
\hline
\multicolumn{6}{c}{~} \\
\multicolumn{6}{l}{'\# at' is the number of atom indices} \\
\multicolumn{6}{l}{'f. tp' is function type} \\
\multicolumn{6}{l}{'pert' indicates if this interaction type
can be modified during free energy perturbation} \\
\multicolumn{6}{l}{~$^{(a)}$ multiplicities can not be modified} \\
\multicolumn{6}{l}{~$^{(b)}$ only the atom type, charge and mass can be modified} \\
\multicolumn{6}{l}{For free energy perturbation, the parameters for topology 'B' (lambda = 1) should be added} \\
\multicolumn{6}{l}{on the same line, after the normal parameters,
in the same order as the normal parameters.} \\
\end{tabular}
}
\caption{The topology ({\tt *.top}) file, part 2.}
\label{ta:topfile2}
\end{table}
\clearpage

Here is an example of a topology file, \verb'urea.top':
{\small\begin{verbatim}
;
;       Example topology file
;
; The force field files to be included
#include "ffgmx.itp"    

[ moleculetype ]
; name  nrexcl
Urea         3

[ atoms ]
;   nr    type   resnr  residu    atom    cgnr  charge
     1       C       1    UREA      C1       1   0.683  
     2       O       1    UREA      O2       1  -0.683
     3      NT       1    UREA      N3       2  -0.622
     4       H       1    UREA      H4       2   0.346
     5       H       1    UREA      H5       2   0.276
     6      NT       1    UREA      N6       3  -0.622
     7       H       1    UREA      H7       3   0.346
     8       H       1    UREA      H8       3   0.276

[ bonds ]
;  ai    aj funct           b0           kb
    3     4     1 1.000000e-01 3.744680e+05 
    3     5     1 1.000000e-01 3.744680e+05 
    6     7     1 1.000000e-01 3.744680e+05 
    6     8     1 1.000000e-01 3.744680e+05 
    1     2     1 1.230000e-01 5.020800e+05 
    1     3     1 1.330000e-01 3.765600e+05 
    1     6     1 1.330000e-01 3.765600e+05 

[ pairs ]
;  ai    aj funct           c6          c12
    2     4     1 0.000000e+00 0.000000e+00 
    2     5     1 0.000000e+00 0.000000e+00 
    2     7     1 0.000000e+00 0.000000e+00 
    2     8     1 0.000000e+00 0.000000e+00 
    3     7     1 0.000000e+00 0.000000e+00 
    3     8     1 0.000000e+00 0.000000e+00 
    4     6     1 0.000000e+00 0.000000e+00 
    5     6     1 0.000000e+00 0.000000e+00 

[ angles ]
;  ai    aj    ak funct          th0          cth
    1     3     4     1 1.200000e+02 2.928800e+02 
    1     3     5     1 1.200000e+02 2.928800e+02 
    4     3     5     1 1.200000e+02 3.347200e+02 
    1     6     7     1 1.200000e+02 2.928800e+02 
    1     6     8     1 1.200000e+02 2.928800e+02 
    7     6     8     1 1.200000e+02 3.347200e+02 
    2     1     3     1 1.215000e+02 5.020800e+02 
    2     1     6     1 1.215000e+02 5.020800e+02 
    3     1     6     1 1.170000e+02 5.020800e+02 

[ dihedrals ]
;  ai    aj    ak    al funct          phi           cp         mult
    2     1     3     4     1 1.800000e+02 3.347200e+01 2.000000e+00 
    6     1     3     4     1 1.800000e+02 3.347200e+01 2.000000e+00 
    2     1     3     5     1 1.800000e+02 3.347200e+01 2.000000e+00 
    6     1     3     5     1 1.800000e+02 3.347200e+01 2.000000e+00 
    2     1     6     7     1 1.800000e+02 3.347200e+01 2.000000e+00 
    3     1     6     7     1 1.800000e+02 3.347200e+01 2.000000e+00 
    2     1     6     8     1 1.800000e+02 3.347200e+01 2.000000e+00 
    3     1     6     8     1 1.800000e+02 3.347200e+01 2.000000e+00 

[ dihedrals ]
;  ai    aj    ak    al funct           q0           cq
    3     4     5     1     2 0.000000e+00 1.673600e+02 
    6     7     8     1     2 0.000000e+00 1.673600e+02 
    1     3     6     2     2 0.000000e+00 1.673600e+02 
 
[ position_restraints ]
; This you wouldn't use normally for a molecule like Urea,
; but it's here for didactical purposes
;     ai   funct      fc
       1       1      1000      1000      1000  ; Restrain to a point
       2       1      1000         0      1000  ; Restrain to a line (Y-axis)
       3       1      1000         0         0  ; Restrain to a plane (Y-Z-plane)
   
[ distance_restraints ]
; This you wouldn't use normally for a molecule like Urea,
; but it's here for didactical purposes
;     ai      aj     type    index     type'  low     up1     up2     fac
       1       6        1        0         1  0.0     0.4     0.5     1.0 
; The following two restraints are taken together (index both 1)
       1       7        1        1         1  0.0     0.3     0.4     1.0 
       1       8        1        1         1  0.0     0.3     0.4     1.0 

; Include SPC water topology
#include "spc.itp"

[ system ]
Urea in Water

[ molecules ]
;molecule name   nr.
Urea             1
SOL              1000
\end{verbatim}}
Here follows the explanatory text.

{\bf \verb'[ defaults ]' :}
\begin{itemize}
\item non-bond type = 1 (Lennard-Jones) or 2 (Buckingham)\\
{\bf note:} when using the buckingham potential no combination rule can
be used, and a full interaction matrix must be provided under the 
{\tt nonbond\_params} section.
\item combination rule = 1 (based on Van der Waals) or 2 (based on
$\sigma$ and $\epsilon$)
\item generate pairs = no (get 1-4 interactions from pairlist) or yes
(generate 1-4 interactions from normal Lennard-Jones parameters using
FudgeLJ and FudgeQQ)
\item FudgeLJ = factor to change Lennard-Jones 1-4 interactions
\item FudgeQQ = factor to change electrostatic 1-4 interactions
\end{itemize}
{\bf note:} FudgeLJ and FudgeQQ only need to be specified when
generate pairs is set to 'yes'.

{\bf \verb'#include "ffgmx.itp"' :} this includes the bonded and
non-bonded {\gromacs} parameters

{\bf \verb'[ moleculetype ]' :} defines the name of your molecule in this
\verb'*.top' and nrexcl = 3 stands for excluding 3 bonded atoms

{\bf \verb'[ atoms ]' :} defines the molecule, where {\tt nr} and {\tt type}
are fixed, the rest is user defined. So {\tt atom} can be named as you
like, {\tt cgnr} made larger or smaller (but beware of the fact that
the total charge of a charge group should be zero!), and charges can
be changed here too.

{\bf \verb'[ bonds ]' :} no comment.

{\bf \verb'[ pairs ]' :} 1-4 interactions

{\bf \verb'[ angles ]' :} no comment

{\bf \verb'[ dihedrals ]' :} in this case there are 9 proper dihedrals
(funct = 1), 3 improper (funct = 2) and no Ryckaert-Bellemans type
dihedrals. If you want to include Ryckaert-Bellemans type dihedrals
in a topology, do the following (in case of e.g. decane):
\begin{verbatim}
[ dihedrals ]
;  ai    aj    ak    al funct           c0           c1           c2
    1    2     3     4     3 
    2    3     4     5     3
\end{verbatim}
and do not forget to {\em erase the 1-4 interaction} 
in \verb'[ pairs ]'!!

{\bf \verb'[ position_restraints ]' :} harmonically restrain particles
to reference positions (section~\ref{sec:posre}). 
The reference positions are read from a 
separate coordinate file by \myindex{grompp}.

{\bf \verb'[ distance_restraints ]' :} put restraints on the distance 
between particles, usually based on experimental data from NMR experiments
(NOE data). The form of the potential function is determined by the
parameters given below (see section~\ref{sec:disre}).

{\bf \verb'#include "spc.itp"' :} includes a topology file that was already
constructed (see next section, molecule.itp).

{\bf \verb'[ system ]' :} title of your system, user defined

{\bf \verb'[ molecules ]' :} this defines the total number of (sub)molecules
in your system that are defined in this \verb'*.top'. In this
example file it stands for 1 urea molecules dissolved in 1000 water
molecules. Make sure that in the \verb'spc.itp' file the molecule is
defined as \normindex{SOL}, otherwise it is not read.

\subsection{Molecule.itp file}
\label{subsec:molitp}
If you construct a topology file you will use more often (like a water
molecule, {\tt spc.itp}) it is better to make a \verb'molecule.itp'
file, which only lists the information of the molecule: 
{\small\begin{verbatim}
[ moleculetype ]
; name  nrexcl
Urea       3

[ atoms ]
;   nr    type   resnr  residu    atom    cgnr  charge
     1       C       1    UREA      C1       1	 0.683	
     .................
     .................
     8       H       1    UREA      H8       3	 0.276

[ bonds ]
;  ai    aj funct           c0           c1
    3     4     1 1.000000e-01 3.744680e+05 
     .................
     .................
    1     6     1 1.330000e-01 3.765600e+05 

[ pairs ]
;  ai    aj funct           c0           c1
    2     4     1 0.000000e+00 0.000000e+00 
     .................
     .................
    5     6     1 0.000000e+00 0.000000e+00 

[ angles ]
;  ai    aj    ak funct           c0           c1
    1     3     4     1 1.200000e+02 2.928800e+02 
     .................
     .................
    3     1     6     1 1.170000e+02 5.020800e+02 

[ dihedrals ]
;  ai    aj    ak    al funct           c0           c1           c2
    2     1     3     4     1 1.800000e+02 3.347200e+01 2.000000e+00 
     .................
     .................
    3     1     6     8     1 1.800000e+02 3.347200e+01 2.000000e+00 

[ dihedrals ]
;  ai    aj    ak    al funct           c0           c1
    3     4     5     1     2 0.000000e+00 1.673600e+02 
    6     7     8     1     2 0.000000e+00 1.673600e+02 
    1     3     6     2     2 0.000000e+00 1.673600e+02 
\end{verbatim}
}
This results in a very short \verb'*.top' file as described in the
previous section, but this time you only need to include files:
{\small\begin{verbatim}
; The force field files to be included
#include "ffgmx.itp"
	
; Include urea topology
#include "urea.itp"

; Include SPC water topology
#include "spc.itp"

[ system ]
Urea in Water

[ molecules ]
;molecule name  number
Urea              1
SOL               1000
\end{verbatim}}

\subsection{Ifdef option}
\label{subsec:ifdef}
A very powerful feature in {\gromacs} is the use of \verb'#ifdef'
statements in your {\tt *.top} file. By making use of this statement,
different parameters for one molecule can be used in the same {\tt
*.top} file. An example is given for TFE, where there is an option to
use different charges on the atoms: charges derived by De Loof
{\etal}~\cite{Loof92} or by Van Buuren and
Berendsen~\cite{Buuren93a}. In fact you can use all the options of the
C-Preprocessor, {\tt cpp}, because this is used to scan the file.  The
way to make use of the \verb'#ifdef' option is as follows:
\begin{itemize}
\item in {\tt grompp.mdp} (the {\gromacs} preprocessor input
      parameters) use the option \verb'define = -DDeloof' or
      \verb'define = -DVanBuuren'
\item put the {\tt \#ifdef} statements in your {\tt *.top}, as
      shown below: 
\end{itemize}
{\small\begin{verbatim}
[ atoms ]
;   nr    type   resnr  residu    atom    cgnr        charge          mass
#ifdef DeLoof
; Use Charges from DeLoof
     1       C       1     TFE       C       1		 0.74
     2       F       1     TFE       F       1		-0.25
     3       F       1     TFE       F       1		-0.25
     4       F       1     TFE       F       1		-0.25
     5     CH2       1     TFE     CH2       1		 0.25
     6      OA       1     TFE      OA       1		-0.65
     7      HO       1     TFE      HO       1		 0.41
#else
; Use Charges from VanBuuren
     1       C       1     TFE       C       1		 0.59
     2       F       1     TFE       F       1		-0.2
     3       F       1     TFE       F       1		-0.2
     4       F       1     TFE       F       1		-0.2
     5     CH2       1     TFE     CH2       1		 0.26
     6      OA       1     TFE      OA       1		-0.55
     7      HO       1     TFE      HO       1		 0.3
#endif

#ifdef BONDS
[ bonds ]
;  ai    aj funct           c0           c1
    6     7     1 1.000000e-01 3.138000e+05 
    1     2     1 1.360000e-01 4.184000e+05 
    1     3     1 1.360000e-01 4.184000e+05 
    1     4     1 1.360000e-01 4.184000e+05 
    1     5     1 1.530000e-01 3.347000e+05 
    5     6     1 1.430000e-01 3.347000e+05 
#else
[ constraints ]
;  ai    aj funct         dist
    6     7     1 1.000000e-01
    1     2     1 1.360000e-01
    1     3     1 1.360000e-01
    1     4     1 1.360000e-01
    1     5     1 1.530000e-01
    5     6     1 1.430000e-01
#endif
\end{verbatim}
}
Also in this example is the option {\tt \#ifdef BONDS}, which results
in \normindex{constraints} instead of normal bonds.

\subsection{Coordinate file}
\label{subsec:grofile}
Files with the {\tt .gro} file extension contain a molecular structure in 
GROMOS87 format. A sample piece is included below:
{\small\begin{verbatim}
MD of 2 waters, reformat step, PA aug-91
    6
    1WATER  OW1    1   0.126   1.624   1.679  0.1227 -0.0580  0.0434
    1WATER  HW2    2   0.190   1.661   1.747  0.8085  0.3191 -0.7791
    1WATER  HW3    3   0.177   1.568   1.613 -0.9045 -2.6469  1.3180
    2WATER  OW1    4   1.275   0.053   0.622  0.2519  0.3140 -0.1734
    2WATER  HW2    5   1.337   0.002   0.680 -1.0641 -1.1349  0.0257
    2WATER  HW3    6   1.326   0.120   0.568  1.9427 -0.8216 -0.0244
   1.82060   1.82060   1.82060
\end{verbatim}}
This format is fixed, ie. all columns are in a fixed position. If you
want to read such a file in your own program without using the
{\gromacs} libraries you can use the following formats:

{\bf C-format:} \verb'"%5i%5s%5s%5i%8.3f%8.3f%8.3f%8.4f%8.4f%8.4f"'

Or to be more precise, with title etc., it looks like this:
\begin{verbatim}
  "%s\n", Title
  "%5d\n", natoms
  for (i=0; (i<natoms); i++) {
    "%5d%5s%5s%5d%8.3f%8.3f%8.3f%8.4f%8.4f%8.4f\n",
      residuenr,residuename,atomname,atomnr,x,y,z,vx,vy,vz
  }
  "%10.5f%10.5f%10.5f%10.5f%10.5f%10.5f%10.5f%10.5f%10.5f\n",
    box[X][X],box[Y][Y],box[Z][Z],
    box[X][Y],box[X][Z],box[Y][X],box[Y][Z],box[Z][X],box[Z][Y]
\end{verbatim}

{\bf Fortran format:} \verb'(i5,2a5,i5,3f8.3,3f8.4)'

So \verb'confin.gro' is the gromacs coordinate file and is almost the
same as the GROMOS87 file (for GROMOS users: when used with ntx=7). 
The only difference is the box for which gromacs uses a tensor, not a
vector.
