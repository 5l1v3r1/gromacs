\chapter{Getting Started.}
\label{ch:start}
In this chapter we assume the reader is familiar with Molecular Dynamics 
and familiar with \normindex{Unix}, including the use of a text editor such as 
{\tt emacs} or {\tt vi}. 
We furthermore assume the {\gromacs} software is
installed properly on your system (see appendix~\ref{ch:install}).
When you see a line like \type{ls -l} you are supposed to type
the contents of that line (not the {\tt \%}) on your computer
that runs {\gromacs}.

\section{Setting up your environment}
Edit your {\tt .cshrc} file to include the following statement: \\
{\tt source {\TILDE}gmx/\normindex{GMXRC}} \\
which has some (system dependent) PATH settings etcetera. Then type
\type{source {\TILDE}gmx/GMXRC} to start right away, or, alternatively 
log off and on again to automatically source the {\gromacs} environment.

Check if you now have the proper environment by
\type{echo \$GMXROOT}
It should read something like: 
{\tt \\ /usr/local/gmx\\}
This is the directory in which the {\gromacs} binaries and
library files live. If you see nothing (a blank line) something is
wrong with the installation, check with your system manager, or (if
{\em you} are the system manager) see appendix~\ref{ch:install}. 

\section{Examples.}
Before starting the \normindex{examples}, you have to copy all the neccesary
files, to your own directory. Chdir to the directory you want to put
the examples directory. This directory (named {\tt \normindex{tutor}}) 
will need
about 20 MB of disk space, when it is completely filled.
\type{cd ``your own directory''} 
then copy the examples: 
\type{cp -r {\TILDE}gmx/tutor .} 
You now have a subdirectory {\tt tutor}. Move there 
\type{cd tutor} 
and view the contents of this directory
\type{ls -l} 
If all is well you will have three subdirectories with examples 
with names like {\tt gmxdemo}, {\tt water}, {\tt speptide}.

You are encouraged to look up the different {\gromacs} programs and
file formats in the online manual eg.
\type{man gromacs}
while you are browsing through the examples. If a World Wide Web (WWW
) browser is installed (like Mosaic or Netscape) on your system, it
is also possible to view the online manual.
\type{mosaic http://rugmd0.chem.rug.nl/{\TILDE}gmx}

\subsection{GMX Demo.}
The {\gromacs} demo is designed to demonstrate the user-friendlyness
of the {\gromacs} software package. Start the {\gromacs} demo by going
to your {\tt tutor/gmxdemo} directory:
\type{cd tutor/gmxdemo} Start the demo: \type{demo}
This demo handles a complete Molecular Dynamics simulation of a
peptide in water, starting from a {\tt pdb} structure. When we start a
Molecular Dynamics simulation with {\gromacs} we need the following
files. 

\subsubsection{Molecular Topology file ({\tt .top})}
The molecular \normindex{topology} file is generated by the program {\tt
pdb2gmx}. {\tt pdb2gmx} translates a {\tt pdb} structure file of any peptide
or protein
to a molecular topology file. This topology file contains a complete
description of all the interactions in your peptide or protein.
 
\subsubsection{Molecular Structure file ({\tt .gro})}
When the {\tt pdb2gmx} program is executed to generate a molecular
topology, it also translates the structure file ({\tt .pdb} file) to a gromos
structure file ({\tt .gro} file). The main difference between a {\tt pdb} 
file and a  gromos file is their format and that
a {\tt .gro} file can also hold \normindex{velocities}.
To generate a box of solvent molecules
around the peptide, the program {\tt genbox} is used. {\tt genbox}
dissolves a solute molecule (the peptide) into any solvent (in this
case water). The output of {\tt genbox} is a gromos structure file of
the peptide dissolved in water. The genbox program also changes the
molecular topology file (generated by {\tt pdb2gmx}) to add solvent
to the topology. 

\subsubsection{Molecular Dynamics parameter file ({\tt .mdp})}
The Molecular Dynamics Parameter ({\tt .mdp}) file contains all
information about the Molecular Dynamics simulation itself 
e.g. time-step, number of steps, temperature, pressure etc. The
easiest way of handling such a file is by adapting a sample {\tt mdp}
file. A sample {\tt mdp} file is generated by the program {\tt
sample}. 

\subsubsection{Index file ({\tt .ndx})}
This file splits the system into several groups. By this way it is
possible to, couple different groups to different temperature baths,
accelerations etc. This index file is generated by the program {\tt
make\_ndx}.  

\subsubsection{Binary Topology file ({\tt .tpb})}
The next step is to combine the molecular structure ({\tt .gro} file),
topology ({\tt .top} file) MD-parameters ({\tt .mdp file}) and the
index file ({\tt ndx}) to generate a binary topology file ({\tt .tpb
extension}). This file contains all information needed to start a
Simulation with {\gromacs}. The {\gromacs} preproccessor program ({\tt
grompp}) processes all input files and generates the binary topology
({\tt tpb}) file.   

\subsubsection{Trajectory file ({\tt .trj})}
Once the binary topology is available, {\gromacs} can start the
simulation. The program which starts the simulation is called {\tt
mdrun}. The only input file of {\tt mdrun} is the binary topology file
({\tt .tpb} file). The output files of {\tt mdrun} are the trajectory
file ({\tt .trj} file) and a logfile ({\tt .log} file).

\subsubsection{Dataflow through GROMACS }
\begin {figure}[H]
\centerline{\psfig {figure=plots/gromdflow.eps,height=10cm}}
\caption {Dataflow through GROMACS} 
\label{fig:grom_dflow}
\end {figure}
In figure \ref{fig:grom_dflow} the dataflow in {\gromacs} is presented
graphically. Normally we start with some Molecular Structure file (
{\tt pdb} file) of some peptide or protein. After that we move
downwards through the programs mentiond in the diagram until we have
finished {\tt mdrun}. Now we can choose to go up, and perform another
simulation, or go down to do some analysis.
The common procedure is to do the {\tt mdrun} three times. The first
time for Energy Minimisation (EM), the second time for Position
Restrained Molecular Dynamics (PR-MD), and the third time for full
Molecular Dynamics (MD). 


\subsection{Water}
Now you are going to simulate 216 molecules of SPC water
\cite{Berendsen81} in a rectangular box. In this example the {\gromacs
} software team already generated most of the neccesary input
files. The files needed in this example are:
\begin{itemize}
\item Initial structure of a box of 216 water molecules ({\tt gro})
\item Topology file of water ({\tt top})
\item Molecular Dynamics parameter file ({\tt mdp})
\item index file ({\tt ndx})
\end{itemize}

Change your directory to {\tt tutor/water }:   
\type{cd tutor/water}
Let's first have a look at the coordinate file:
\type{more spc216.gro}
Or to view the water box graphically:
\type{rasmol spc216.pdb} 
Have a look at the topology file:
\type{more water.top}
Have a look at the MD-parameters file:
\type{more water.mdp}
Since all the neccesary files are available, we are going to,
preprocess all the input files to create a binary topology ({\tt tpb}
) file. This binary topology file is the only input file for the
MD-program {\tt mdrun}. 
\type{grompp -f water -p water -c spc216.gro -n water.ndx -o water.tpb} 
Since the binary topology file is not a regular ASCII file, it is only
viewable with the program {\tt rstat}. By this way it is possible to
check if the {\gromacs} preprocessor {\tt grompp} worked well.
\type{rstat -s water.tpb | more}
Now it's time to start to the simulation
\type{mdrun -s water.tpb -o water.trj -c water\_out.gro -v -g log}
After the MD simulation is finished, it is possible to view the
trajectory with the {\tt ngmx} program:
\type{ngmx -f water.trj -s water.tpb -n water.ndx}
Calculate a radial distribution function of the Oxygen atoms. The
index file {\tt oxygen.ndx} contains one group with all the oxygen
atoms. 
\type{g\_rdf -f water.trj -n oxygen.ndx -o rdf.xvg -s water.tpb}
view the output graph of {\tt g\_rdf}
\type{xvgr rdf.xvg} 
Which shows you the radial distribution function for Oxygen-Oxygen in
SPC water. 

\subsection{Ribonuclease S-peptide.}
Ribonuclease A is a digestive enzyme, secreted by the pancreas. The enzyme
can be cleaved by subtilisin at a single peptide bond to yield 
Ribonuclease-S, a catalytically active complex of an S-peptide moiety
(residues 1-20) and an S-protein  moiety (residues 21-124), bound together
by multiple non-covalent links~\cite{Stryer88}. 

The S-Peptide has been studied in many ways, experimentally
as well as theoretically (simulation) because of the high $\alpha$-helix 
content in solution, which is remarkable in such a small peptide.

All the files of speptide are stored in the directory {\tt
tutor/speptide}. First go to this directory:\type{cd speptide}  

To be able to simulate the S-Peptide we need a starting structure. This can
be taken from the protein data bank. There are a number of different
structure for Ribonuclease S, from one of which we have cut out the
first 20 residues, and stored it in {\tt speptide.pdb}. Have a look at the
file \type{more speptide.pdb} If you have access to a molecular
graphics program such as \normindex{rasmol}, \normindex{xmol}, 
or a commercial package,
you can look at the molecule on screen, eg: \type{rasmol speptide.pdb}

The following steps have to be taken to perform a simulation of the peptide.
\begin{enumerate}
\item	Convert the pdb-file {\tt speptide.pdb} to a {\gromacs} structure
	file and a {\gromacs} topology file.
\item	Solvate the peptide in water
\item   Generate index file ({\tt .ndx} extension)
\item	Perform an energy minimization of the peptide in solvent
\item	Add ions if necessary (we will omit this step here)
\item	Perform a short MD run with position restraints on the peptide
\item	Perform full MD without restraints
\item	Analysis
\end{enumerate}

We will describe in detail how such a simulation can be done using
{\gromacs}, starting from a pdb-file.

\subsubsection{Convert the pdb-file ({\tt pdb}) to a gromos structure file ({\tt gro}) and a topology file ({\tt top})}

Generate a molecular topology and a structure file in {\gromacs}
format. This can be done with the {\tt pdb2gmx} program: 
\type{pdb2gmx -f speptide.pdb -p speptide.top -o speptide.gro}
Note that the correct file extension are added automatically to the
filenames on the command line. The program also interactively queries
about which forcefield to use, about protonation of residues, and
about protonation of N- and C-terminus. 

The interactive selection of residues goes a little like this:

\begin{verbatim}
Which LYSINE type do you want for residue 1
0. Not protonated (LYS)
1. Protonated (LYSH)

Type a number:
\end{verbatim}

Select Protonated ({\bf 1}) for all Lysines. For Histidines the
procedure is similar: 
\begin{verbatim}
Which HISTIDINE type do you want for residue 11
0. H on ND1 only (HISA)
1. H on NE2 only (HISB)
2. H on ND1 and NE2 (HISH)
3. Coupled to Heme (HIS1)

Type a number:
\end{verbatim} 
Select HisH ({\bf 2}) in this case.

The force field selection goes as follows: 
\begin{verbatim}
Select the Force Field:
 0: Gromacs Forcefield (see manual)
 1: Gromos-87 with all hydrogens.
 2: OPLS forcefield version (version 8/94, will be added later) 
\end{verbatim}

Select option {\bf 0}.
	
Then the program requests whether you want hydrogens on the N-terminus:
\begin{verbatim}
Select N-terminus type (start)
 0: None
 1: NH3+
 2: NH2
 3: PRO-NH2+
 4: PRO-NH
\end{verbatim}

Select {\bf 1}. Finally the program asks you whether you want
to protonate the C-terminal oxygen:
\begin{verbatim}
Select C-terminus type (end)
 0: None
 1: COO-
 2: COOH
\end{verbatim}

Select {\bf 1} here.

The program will generate a topology file {\tt speptide.top} and a
{gromacs} structure file {\tt speptide.gro} and it will 
\normindex{generate hydrogen} 
positions. The {\tt -p} and {\tt -o} options with he
filenames are optional; without them the files {\tt topol.top} and {\tt
conf.gro} will be generated (see \ref{sec:fileformats}).  Now have a
look at the output from {\tt pdb2gmx},
\type{more speptide.gro}
You will see a close resemblance to the pdb file, only the layout of
the file is a bit different. For the exact layout read the online
manual 
\type{man confin.gro}
Also do have a look at the topology 
\type{more speptide.top}
You will see a large file containing the atom types, the physical
bonds between atoms, etcetera. If you want to know more about the
program check the online manual 
\type{man pdb2gmx}
to read more about the topology file read chapter~\ref{ch:top}.

\subsubsection{Solvate the peptide in a periodic box filled with water}
This can be done using the {\tt genbox} program. This will read the
structure file, calculate the size of the peptide, make a rectangular
box (or truncated octahedron, see section~\ref{sec:pbc}), 
with a water layer of user specified size. The {\tt genbox} program
requires an input file for which we can get an example with 
\type{sample box.gbp > box.gbp}
You can look at the contentents of the {\tt box.gbp} file, but
you do not have to change them because the variables in this file are
correct. Now it is time to do the generation
\type{genbox -f box.gbp -cp speptide -cs spc216 -p speptide -o b4em}
The program prints some lines of user information, Like density and
the number of water molecules added to your peptide. {\tt genbox} also
changes the topology file {\tt speptide.top} to include these water
molecules in the topology. This can been seen by looking at the bottom
of the {\tt speptide.top} file 
\type{tail speptide.top}
You will see some lines like 
\begin{verbatim}
[ system ]
; Name		Number
Protein		1
SOL		N
\end{verbatim}
where {\tt N} is the number of water molecules added to your system by
{\tt genbox}. 

The {\tt genbox} program also reads and generates a file {\tt out.vdw},
containing the \normindex{Vanderwaals radii} 
used in generating the water box. It
is also possible to solvate a peptide in another solvent such as
dimethylsulfoxide (DMSO), as has been done by Mierke {\etal}~\cite{Mierke91}.  

\subsubsection{Generate index file ({\tt .ndx} extension)}
The {\tt .ndx } or index file is also needed to be able to perform an
MD run. This {\tt ndx} file is used for splitting up our system into
several groups. These groups can be exposed to different external
influences during the simulation. It is now possible to couple the
peptide and the solvent to different temperature baths.
\type{make\_ndx -f b4em -o analysis -no}
This reads up the coordinate file {\tt b4em.gro}, 
analyses it and generates {\em groups} of atoms. On these \normindex{groups}
of atoms, a number of different \normindex{analysis programs} can act
(see Section~\ref{sec:groups}). 

\subsubsection{Perform an energy minimization of the peptide in solvent}
Now we have to perform an {\em energy minimization} of the structure
to remove the local strain in the peptide (due to generation of
hydrogen positions) and to remove bad Van der Waals contacts
(particles that are too close). This can be done with the {\tt mdrun}
program which is the MD and EM program. Before we can use the {\tt
mdrun} program however, we have to preprocess the topology file ({\tt
speptide.top}), the structure file ({\tt speptide.gro}) and a special
parameter file ({\tt em.mdp}). Check the contents of this file
\type{more em.mdp} Preprocessing is done with the {\gromacs}
preprocessor called {\tt grompp}. This reads up the files just
mentioned: 
\type{grompp -v -f em -c b4em -n analysis -o em -p speptide}
In this command the {\tt -v} option turns on verbose mode, which gives
a little bit of clarifying info on what the program is doing, 
(the use of index files for {\tt mdrun} is described in 
chapter~\ref{ch:algorithms}). 
We now have made a
{\em binary topology} ({\tt em.tpb}) which serves as input for the
{\tt mdrun} program. Now we can do the energy minimization: 
\type{mdrun -v -s em -o em -c after\_em -g emlog}
In this command the {\tt -v} option turns on verbose mode again.
The {\tt -o} option sets the filename for the trajectory file,
which is not very important in energy minimizations. The {\tt -c}
option sets the filename of the structure file after energy
minimization. This file we will subsequently use as input
for the MD run.
The energy minimization takes some time, the amount
depending on the CPU in your computer, the
load of your computer, etc. The {\tt mdrun} program is
automatically {\em niced}; it runs at low priority. All {\gromacs}
programs that do extensive computations are automatically run
at low priority. For most modern workstations this computation
should be a matter of minutes. The minimization is finished when
either the minimization has converged or
a fixed number of steps has been performed.
Since the system consists merely of water, a quick check on the
potential energy should reveal whether the minimization was 
successful: the potential energy of 1 SPC water molecule at 300 K
is {\tt -42} kJ mole$^{-1}$. Since we have about {\tt 9e+02} SPC
molecules the potential energy should be about {\tt-40e+3} kJ
mole$^{-1}$. If the potential energy after minimization is lower than
{\tt -30e+03} kJ mole$^{-1}$ it is acceptable and the structure can be
used for MD calculations. 
After our EM calculation the program prints something like:
\begin{verbatim}
STEEPEST DESCENTS converged to 0.001000 
  Function value at minimum =  -3.4772e+04
\end{verbatim}
which means our criterium is met, and we can proceed to the next step.

\subsubsection{Perform a short MD run with position restraints on the peptide}
Position restrained MD means Molecular Dynamics in which a part of the
system is not allowed to move far off their starting positions (see
Section\ref{sec:posre}). To be able to run with position restraints we
must add a section to the {\tt speptide.top} file, describing which
atoms are to be restrained. Such a section can be generated using the
{\tt genpr} program. The usage of the {\tt genpr } program is: 
\type{genpr -a1 firstatom -a2 lastatom -fc forceconstant -o posre.itp}
In the S-peptide we are simulating there are 187 atoms, a suitable
force constant is 1000 (kJ mole$^{-1}$ nm$^{-2}$) so the command
line should be: 
\type{genpr -a1 1 -a2 187 -fc 1000 -o posre.itp}
You now have a file called {\tt posre.itp} which should be included
in the topology file eg. after the {\tt [ atoms ]} section
of the S-peptide moleculetype like:
\begin{verbatim}
#ifdef POSRES
#include "posre.itp"
#endif
\end{verbatim}
Here we use conditional inclusion, i.e. only if a variable {\tt POSRES}
is set in the preprocessor do we include the file, this allows 
us to use the same topology file for runs with and without
position restraints. In the parameter file for the position restraints
this variable is set indeed:
\begin{verbatim}
define              =  -DPOSRES
\end{verbatim}
See also chapter~\ref{ch:top}.

At last we can generate the input for the position restrained MDrun:
\type{grompp -f pr -o pr -c after\_em -r after\_em -n analysis -p speptide}
Now it's MDrun time:
\type{mdrun -v -s pr -o pr -c after\_pr -g prlog $>$\& pr.job \&} 
This run is started in the background (it will take a while), you
can watch how long it will take by typing:
\type{tail -f pr.job}
With the {\tt Ctrl-C} key you can \normindex{kill} the {\tt tail} command.

\subsubsection{Perform full MD without restraints}
Full MD is very similar to the restrained MD as far as {\gromacs} is
concerned.  
\type{grompp -v -f full -o full -c after\_pr -n analysis}
Then we can start mdrunning
\type{mdrun -v -s full -o full -c after\_full -g flog $>$\& full.job \&}

\subsubsection{Analysis}
We will not describe analysis in detail, because most analysis tools
are described in the Analysis chapter (chapter
\ref{ch:analysis}) and the online manual ({\wwwpage}). We
just list a few of the possibilities within {\gromacs}. By now you should be
able to start programs yourself.

\begin{itemize}
\item View the trajectory on your own X-screen (program {\tt ngmx}).
\item Root Mean Square Deviation with respect to the crystal
	structure (program {\tt g\_rms}).
\item Radius of Gyration (program {\tt g\_gyrate}).
\item Secondary Structure analysis (program {\tt do\_dssp})
	For this analysis you should have the dssp\cite{Kabsch83}
	software installed. This program also produces
	the solvent accesible surface area as a function of time.
\item Ramachandran Plots (program {\tt g\_rama}).
\item Salt Bridge analysis (program {\tt g\_saltbr}).
\end{itemize}

The data-flow of this simulation up to
the position restrained MD is visualised in Figure~\ref{fig:spep}. 
\begin {figure}[H]
\centerline{\psfig {figure=plots/speptide.eps,width=10cm,angle=-90}}
\caption {Data Flow in the S-Peptide simulation.}
\label{fig:spep}
\end {figure}

You have been witness of a full MD simulation starting from a pdb-file.
It's that easy, but then again, maybe it was not that easy. The
example presented here is a {\em real} example, this is how a 
production run should be performed, the complexity is in the process 
itself and not in the software (at least, that's our opinion).

\section{Step 3. Your own System.}

For proteins in water (or other solvent) the route is described above in (). For other systemd (eg. pure liquids or mixtures) on needs:
\begin{itemize} 
\item 	The atomic coordinates, which can be generated by a variety of 
	interactive programs (eg. Quanta, Cerius, HyperChem). 
	Coordinate files can be exported in pdb-format and 
	converted to {\gromacs} format by:
	\type{pdb2gro -f conf.pdb -o conf.gro}
	where {\tt conf.gro} is the {\gromacs} coordinatefile, 
	or converted back to pdb-format by
	\type {gro2pdb -f conf.gro -o conf.pdb }
	where {\tt conf} is a file with {\gromacs} coordinates, and {\tt
	conf.pdb} is the target file in .pdb format.
	{\bf NOTE:} Make sure that the graphics programs export 
	{\bf whole} molecules instead of molecules that are cut in pieces
	(due to the periodic boundary conditions)
	If you have the coordinates of single molecules, you can also 
	build systems (pure liquids or mixtures) with {\tt genbox}.
	The program {\tt genconf} produces the lattice of molecules 
	with random displacements.
\item 	The topology you have to build yourself. Of course you can 
	include topologies of part of your system (eg. {\tt spc.top}, 
	{\tt decane.top} etc.) If GROMOS topologies are available you 
	can convert them with:
	\type{rmt2top -f <rmtfile>.rmt -p <gromacsfile>.top}
	See also:
	\type{man topol.top}
\end{itemize}

