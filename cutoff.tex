\subsection{Treatment of cut-offs}
\newcommand{\rs}{$r_{short}$}
\newcommand{\rl}{$r_{long}$}
This section is involved with the practical aspects of treating cut-offs.
{\gromacs} is quite flexible in this respect, which implies that there are
quite a number of parameters to set. The parameters are set in the input file
for \myindex{grompp}. One should distinguish two parts of the parameters:
1$^{st}$ the paramaters that describe the function (Coulomb / VDW, 
\tabref{funcparm}) and
2$^{nd}$ the parameters that describe neighboursearching.

\begin{table}[h]
\centering
\caption{Parameters describing the functional form. Column VDW indicates whether this applies to Van der Waals forces as well as to Coulomb.}
\label{tab:funcparm}
\begin{tabular}{|l|c|c|l|}
\dline
Type		& VDW	& \# params 	& Description		\\
\hline
Plain cut-off	& X	& 2		& $r_c$, $\epsr$	\\
Reaction field	&	& 2		& $r_c$, $\epsrf$	\\
Shift function	& X	& 2		& $r_1$, $r_c$ 		\\
Switch function &	& 2		& $r_1$, $r_c$ 		\\
\dline
\end{tabular}
\end{table}
In summary, for both VDW and Coulomb there are a type selector
({\tt vdw\_type} resp. {\tt eel\_type}) and two parameters (see above), for
a total of six parameters.

The neighboursearching (NS) maybe done using a single-range, or a twin-range 
approach. Since the former is merely a special case of the latter we will 
discuss the more general twin-range. In this case NS is described by two
radii {\rs} and {\rl}. Usually one builds the neighbourlist (NBL)
every 10 time steps
or every 20 fs (parameter {\tt nstlist}). In the NBL all interaction 
pairs that  fall within {\rs} are stored. Furthermore, the 
interactions between pairs that do not
fall within {\rs} but do fall within {\rl} are computed during NS, and the
forces and energy are stored separately, and added to short-range forces
at every time step between successive NS. If {\rs} = {\rl} no forces
are evaluated during NBL generation.

Except for the plain cut-off,
all of the interaction functions in \tabref{funcparm}
require that neighboursearching is done with a larger radius than the $r_c$
specified for the functional form, because of the use of charge groups.
The extra radius is typically of the order of 0.25 nm (roughly the 
largest distance between two atoms in a charge group plus the distance a 
charge group can diffuse within NBL updates).
If your charge groups are very large it may be interesting to turn off charge
groups, by setting the option 
{\tt bAtomList = yes} in your {\tt grompp.mdp} file.
In this case only a small extra radius to account for diffusion needs to be 
added (0.1 nm). Do not however use this together with the plain cut-off
method, as it will generate large artefacts (\secref{cg}).
In summary, there are four parameters that describe NS behaviour:
{\tt nstlist} (update frequency in number of time steps),
{\tt bAtomList} (whether or not charge groups are used to generate neighbourlist, the default is to use charge groups, so {\tt bAtomList = no}),
{\tt rshort} and {\tt rlong} which are the two radii {\rs} and {\rl}
described above.

