%
% 
%       This source code is part of
% 
%        G   R   O   M   A   C   S
% 
% GROningen MAchine for Chemical Simulations
% 
%               VERSION 2.0
% 
% Copyright (c) 1991-1999
% BIOSON Research Institute, Dept. of Biophysical Chemistry
% University of Groningen, The Netherlands
% 
% Please refer to:
% GROMACS: A message-passing parallel molecular dynamics implementation
% H.J.C. Berendsen, D. van der Spoel and R. van Drunen
% Comp. Phys. Comm. 91, 43-56 (1995)
% 
% Also check out our WWW page:
% http://md.chem.rug.nl/~gmx
% or e-mail to:
% gromacs@chem.rug.nl
% 
% And Hey:
% Gnomes, ROck Monsters And Chili Sauce
%

% The following definition allows for a way to write
% '{\tt [ nonbond_params ]}' such that the underscore
% renders in the right font, and the text search string
% 'nonbond_params' in the resulting PDF finds this
% instance.

\chardef\us=`\_

% Correct usage is '{\tt [ nonbond\us{}params ]}'.
%
% LaTeX will complain if you try '{\tt [ nonbond_params ]}',
% for which the obvious work-around will be
% '\tt [ nonbond\_params]}'. This renders the underscore
% in the wrong font, and confounds text searches. Short
% of writing a pre-processor, there's no way to enforce
% the use of \us.
%
% An alternative might be to use \verb'[ nonbond_params ]',
% which works well until you go to embed it in a
% multicolumn table entry, index entry, or other such command
% environment. Verbatim text really does not play nicely
% in those contexts. It would also enforce non-breaking
% spaces without needing to revert to '[~nonbond_params~]'.

\setlength {\parindent}{0.0cm}
\setlength {\parskip}{1ex}
\newcommand{\ve}[1]{\mbox{\boldmath ${#1}$}} 
  % defines bold italic vectors. To be used in text or math mode.
  % Example: \ve{F}
\newcommand{\de}{\mbox{d}} 
  % defines a straight d for derivatives.
\newcommand{\intel}{Intel {\em i\/}860}
\newcommand{\gromacs}{GROMACS}
\newcommand{\gromosv}[1]{GROMOS-#1}
\newcommand{\gromos}{GROMOS}
\newcommand{\dline}{\hline\hline}
\newcommand{\etal}{{\em et al.}}
\newcommand{\beq}{\begin{equation}}
\newcommand{\eeq}{\end{equation}}
\newcommand{\bea}{\begin{eqnarray}}
\newcommand{\eea}{\end{eqnarray}}
\newcommand{\Dt}{{\Delta t}}
\newcommand{\half}{\frac{1}{2}}
\newcommand{\hDt}{\half \Dt}
\newcommand{\rvi}{\ve{r}_i}
\newcommand{\rvj}{\ve{r}_j}
\newcommand{\rvk}{\ve{r}_k}
\newcommand{\rij}{r_{ij}}
\newcommand{\rvij}{\ve{r}_{ij}}
\newcommand{\rnorm}{\frac{\rvij}{\rij}}
\newcommand{\Fvi}{\ve{F}_i}
\newcommand{\Fvj}{\ve{F}_j}
\newcommand{\Fvk}{\ve{F}_k}
\newcommand{\Fvij}{\ve{F}_{ij}}
\newcommand{\Fvji}{\ve{F}_{ji}}
\newcommand{\vvi}{\ve{v}_i}
\newcommand{\al}{\alpha}
\newcommand{\be}{\beta}
\newcommand{\ab}{\alpha\beta}
\newcommand{\rnij}{\ve{r}_{ij}^n}
\newcommand{\rni}{\ve{r}_i^n}
\newcommand{\hdt}{\frac{\Delta t}{2}}
\newcommand{\type}[1]{\\ {\tt \% #1}\\}
\newcommand{\normindex}[1]{#1{\index{#1}}}
\newcommand{\swapindex}[2]{#1 #2{\index{#1 #2}}{\index{#2, #1|see{#1 #2}}}}
\newcommand{\pawsindex}[2]{#1 #2{\index{#1 #2|see{#2, #1}}}{\index{#2, #1}}}
\newcommand{\boldindex}[1]{#1\index{#1@\textbf{#1}}}
\newcommand{\ttusindex}[2]{{\tt #1\us{}#2}{\index{#1\_#2}}}
\newcommand{\seeindex}[2]{#1{\index{#1|see{#2}}}}
\newcommand{\eg}{\em e.g.\@}
\newcommand{\ie}{\em i.e.\@}
\newcommand{\mc}[3]{\multicolumn{#1}{#2}{#3}}

% Commands for correct spacing in tables
%       TeX and TUG News, Vol.2, No.3, p10, 1993. 
%
\newcommand\T{\rule{0pt}{2.6ex}}         % Top strut
\newcommand\B{\rule[-1.2ex]{0pt}{0pt}}   % Bottom strut
\newcommand{\Ts}{\rule{0pt}{2.4ex}}      % Smaller top strut (To be used in 
                                         % math mode in \frac, together with 
                                         % \displaystyle

\newcommand{\captspace}{\vspace{2mm}}

\newif\ifpdfman
\ifx\pdfoutput\undefined
  \pdfmanfalse % not using pdflatex  
\else
  \pdfoutput=1 % using pdflatex
  \pdfmantrue
\fi

\ifpdfman
  \usepackage[pdftex]{graphicx}
  \usepackage[pdftex,plainpages=false,pdfpagelabels,colorlinks=true,urlcolor=blue,citecolor=blue,pdfstartview=FitV]{hyperref}
\else
  \usepackage{graphicx}
  \usepackage[plainpages=false,colorlinks=true,urlcolor=blue,citecolor=blue]{hyperref}
\fi
% Slightly darker colors print better on b/w printers
\usepackage{color}
\definecolor{blue}{rgb}{0,0,0.5}
\definecolor{red}{rgb}{0.5,0,0}

\makeindex
% The total a4 paper width is 210 mm, and the margins are defined
% relative to the standard margin of 1 inch.
\setlength{\textwidth}{15cm}
\setlength{\oddsidemargin}{9.6mm}  %35mm-1inch
\setlength{\evensidemargin}{-0.4mm} %25mm-1inch
\setlength{\topmargin}{-4.4mm} % 21mm top margin
\setlength{\textheight}{22cm}

\renewcommand{\textfraction}{0.0}
\newcommand{\qtw}{3.75cm}  % 1/4 of the textwidth
\newcommand{\ttw}{4.33cm}  % 1/3 of the textwidth
\newcommand{\htw}{7.5cm}  % 1/2 of the textwidth
\newcommand{\ntw}{13cm} % 0.9 of the textwidth
\newcommand{\gmxmajor}{4}
\newcommand{\gmxver}{4.5}
\newcommand{\figref}[1]{Fig.~\ref{fig:#1}}
\newcommand{\figsref}[2]{Figs.~\ref{fig:#1} and ~\ref{fig:#2}}
\newcommand{\tabref}[1]{Table~\ref{tab:#1}}
\newcommand{\eqnref}[1]{eqn.~\ref{eqn:#1}}
\newcommand{\eqnsref}[2]{eqns.~\ref{eqn:#1} and \ref{eqn:#2}}
\newcommand{\secref}[1]{sec.~\ref{sec:#1}}
\newcommand{\ssecref}[1]{\ref{subsec:#1}}
\newcommand{\chref}[1]{chapter~\ref{ch:#1}}
\newcommand{\appref}[1]{Appendix~\ref{app:#1}}
\newcommand{\wwwpage}{\href{http://www.gromacs.org}{www.gromacs.org}}
% NOTE: \wwwpage is explicitly included in the 'verbatim' citing instructions!
\newcommand{\email}{\href{mailto:gromacs@gromacs.org}{gromacs@gromacs.org}}
\newcommand{\mcc}[2]{\multicolumn{#1}{c|}{#2}}
\newcommand{\mcl}[2]{\multicolumn{#1}{l}{#2}}
\setlength{\headwidth}{\textwidth}
\setlength{\headheight}{1.0cm}

% LocalWords:  GROMOS et al ij ji rgb eqn eqns
