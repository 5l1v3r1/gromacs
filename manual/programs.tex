%
% This file is part of the GROMACS molecular simulation package.
%
% Copyright (c) 2013, by the GROMACS development team, led by
% Mark Abraham, David van der Spoel, Berk Hess, and Erik Lindahl,
% and including many others, as listed in the AUTHORS file in the
% top-level source directory and at http://www.gromacs.org.
%
% GROMACS is free software; you can redistribute it and/or
% modify it under the terms of the GNU Lesser General Public License
% as published by the Free Software Foundation; either version 2.1
% of the License, or (at your option) any later version.
%
% GROMACS is distributed in the hope that it will be useful,
% but WITHOUT ANY WARRANTY; without even the implied warranty of
% MERCHANTABILITY or FITNESS FOR A PARTICULAR PURPOSE.  See the GNU
% Lesser General Public License for more details.
%
% You should have received a copy of the GNU Lesser General Public
% License along with GROMACS; if not, see
% http://www.gnu.org/licenses, or write to the Free Software Foundation,
% Inc., 51 Franklin Street, Fifth Floor, Boston, MA  02110-1301  USA.
%
% If you want to redistribute modifications to GROMACS, please
% consider that scientific software is very special. Version
% control is crucial - bugs must be traceable. We will be happy to
% consider code for inclusion in the official distribution, but
% derived work must not be called official GROMACS. Details are found
% in the README & COPYING files - if they are missing, get the
% official version at http://www.gromacs.org.
%
% To help us fund GROMACS development, we humbly ask that you cite
% the research papers on the package. Check out http://www.gromacs.org.

\chapter{Run parameters and Programs}
\label{ch:programs}

\section{On-line and HTML manuals}
\index{online manual}
\index{html manual}
% TODO Check that these statements and default file locations are
% current when the help-writing module has stabilized
All the information in this chapter can also be found in HTML
format in your {\gromacs} data directory. The path depends on
where your files are installed, but the default location is
{\tt /usr/local/gromacs/share/gromacs/html/online.html}.
If you installed from Linux packages it can typically be found as
{\tt /usr/share/gromacs/html/online.html}.
You can also use the online manual from the {\gromacs} web site,
\url{http://manual.gromacs.org/current}.

In addition, we install standard UNIX man pages for all the programs. If
you have sourced the {\tt GMXRC} script in the {\gromacs} binary directory for
your host they should already be present in your {\tt MANPATH} environment variable, and you
should be able to type {\eg} {\tt man gmx-grompp}. You can also use
the {\tt -h} flag on the command line (e.g. {\tt gmx grompp -h}) to
see the same information, as well as {\tt gmx help grompp}.
The list of all programs are available from {\tt gmx help}.

\section{File types\swapindexquiet{file}{type}}
\label{sec:fileformats}
\tabref{form} lists the file types used by {\gromacs} along with
a short description, and you can find a more detail description for
each file in your HTML reference, or in our online version.

{\gromacs} files written in \normindex{XDR} format can be read on any
architecture with {\gromacs} version 1.6 or later if the configuration
script found the XDR libraries on your system. They should always be
present on UNIX since they are necessary for NFS support.

\begin{table}
\begin{tabularx}{\linewidth}{|r@{\tt.}lccX|}
\dline
\mc{2}{|c}{Default} &      & Default &  \\[-0.1ex]
\mc{1}{|c}{Name} & \mc{1}{c}{Ext.} & Type &  Option & Description \\[-0.1ex]
\hline
\tt   atomtp & \tt atp & Asc & \tt    & Atomtype file used by pdb2gmx \\[-0.1ex]
\tt    eiwit & \tt brk & Asc & \tt -f & Brookhaven data bank file \\[-0.1ex]
\tt   nnnice & \tt dat & Asc & \tt    & Generic data file \\[-0.1ex]
\tt     user & \tt dlg & Asc & \tt    & Dialog Box data for ngmx \\[-0.1ex]
\tt      sam & \tt edi & Asc & \tt    & ED sampling input \\[-0.1ex]
\tt      sam & \tt edo & Asc & \tt    & ED sampling output \\[-0.1ex]
\tt     ener & \tt edr &     & \tt    & Generic energy: \tt edr ene \\[-0.1ex]
\tt     ener & \tt edr & xdr & \tt    & Energy file in portable xdr format \\[-0.1ex]
\tt     ener & \tt ene & Bin & \tt    & Energy file \\[-0.1ex]
\tt    eiwit & \tt ent & Asc & \tt -f & Entry in the protein date bank \\[-0.1ex]
\tt     plot & \tt eps & Asc & \tt    & Encapsulated PostScript (tm) file \\[-0.1ex]
\tt    gtraj & \tt g87 & Asc & \tt    & Gromos-87 ASCII trajectory format \\[-0.1ex]
\tt     conf & \tt g96 & Asc & \tt -c & Coordinate file in Gromos-96 format \\[-0.1ex]
\tt     conf & \tt gro &     & \tt -c & Generic structure: \tt gro g96 pdb tpr tpb tpa \\[-0.1ex]
\tt      out & \tt gro &     & \tt -o & Generic structure: \tt gro g96 pdb \\[-0.1ex]
\tt     conf & \tt gro & Asc & \tt -c & Coordinate file in Gromos-87 format \\[-0.1ex]
\tt    polar & \tt hdb & Asc & \tt    & Hydrogen data base \\[-0.1ex]
\tt   topinc & \tt itp & Asc & \tt    & Include file for topology \\[-0.1ex]
\tt      run & \tt log & Asc & \tt -l & Log file \\[-0.1ex]
\tt       ps & \tt m2p & Asc & \tt    & Input file for mat2ps \\[-0.1ex]
\tt       ss & \tt map & Asc & \tt    & File that maps matrix data to colors \\[-0.1ex]
\tt       ss & \tt mat & Asc & \tt    & Matrix Data file \\[-0.1ex]
\tt   grompp & \tt mdp & Asc & \tt -f & grompp input file with MD parameters \\[-0.1ex]
\tt  hessian & \tt mtx & Bin & \tt -m & Hessian matrix \\[-0.1ex]
\tt    index & \tt ndx & Asc & \tt -n & Index file \\[-0.1ex]
\tt    hello & \tt out & Asc & \tt -o & Generic output file \\[-0.1ex]
\tt    eiwit & \tt pdb & Asc & \tt -f & Protein data bank file \\[-0.1ex]
\tt     pull & \tt pdo & Asc & \tt    & Pull data output \\[-0.1ex]
\tt     pull & \tt ppa & Asc & \tt    & Pull parameters \\[-0.1ex]
\tt  residue & \tt rtp & Asc & \tt    & Residue Type file used by pdb2gmx \\[-0.1ex]
\tt      doc & \tt tex & Asc & \tt -o & LaTeX file \\[-0.1ex]
\tt    topol & \tt top & Asc & \tt -p & Topology file \\[-0.1ex]
\tt    topol & \tt tpa & Asc & \tt -s & Ascii run input file \\[-0.1ex]
\tt    topol & \tt tpb & Bin & \tt -s & Binary run input file \\[-0.1ex]
\tt    topol & \tt tpr &     & \tt -s & Generic run input: \tt tpr tpb tpa \\[-0.1ex]
\tt    topol & \tt tpr &     & \tt -s & Structure+mass(db): \tt tpr tpb tpa gro g96 pdb \\[-0.1ex]
\tt    topol & \tt tpr & xdr & \tt -s & Portable xdr run input file \\[-0.1ex]
\tt     traj & \tt trj & Bin & \tt    & Trajectory file (cpu specific) \\[-0.1ex]
\tt     traj & \tt trr &     & \tt    & Full precision trajectory: \tt trr trj \\[-0.1ex]
\tt     traj & \tt trr & xdr & \tt    & Trajectory in portable xdr format \\[-0.1ex]
\tt     root & \tt xpm & Asc & \tt    & X PixMap compatible matrix file \\[-0.1ex]
\tt     traj & \tt xtc &     & \tt -f & Generic trajectory: \tt xtc trr trj gro g96 pdb \\[-0.1ex]
\tt     traj & \tt xtc & xdr & \tt    & Compressed trajectory (portable xdr format) \\[-0.1ex]
\tt    graph & \tt xvg & Asc & \tt -o & xvgr/xmgr file \\[-0.1ex]
\dline
\end{tabularx}
\caption{The {\gromacs} file types.}
\label{tab:form}
\end{table}


\section{Run Parameters\swapindexquiet{run}{parameter}}
\subsection{ General}

Default values are given in parentheses. The first option is
always the default option. Units are given in square brackets The
difference between a dash and an underscore is ignored. 

A sample {\tt .mdp} file is
available. This should be appropriate to start a normal
simulation. Edit it to suit your specific needs and desires. 

\subsection{ Preprocessing}
\begin{description}
\item[{\bf title:}]\mbox{}\\
this is reduntant, so you can type anything you want
\item[{\bf cpp: }(/lib/cpp)]\mbox{}\\
your preprocessor
\item[{\bf include:}]\mbox{}\\
directories to include in your topology. format: 
\\{\tt-I/home/john/my\_lib -I../more\_lib}\\
\item[{\bf define: }()]\mbox{}\\
defines to pass to the preprocessor, default is no defines. You can use
any defines to control options in your customized topology files. Options
that are already available by default are:
\vspace{-2ex}\begin{description}
\item[{\bf -DFLEX\_SPC}]\mbox{}\\
Will tell grompp to include FLEX\_SPC in stead of SPC into your
topology, this is necessary to make 
{\bf conjugate gradient} work and will allow 
{\bf steepest descent} to minimize further.
\item[{\bf -DPOSRE}]\mbox{}\\
Will tell grompp to include posre.itp into your topology, used for
position restraints.
\end{description}
\end{description}

\subsection{ Run control}
\begin{description}
\item[{\bf integrator:}]\mbox{}\\
\vspace{-2ex}\begin{description}
\item[{\bf md} ]\mbox{}\\
A leap-frog algorithm for integrating Newtons equations.
\item[{\bf steep}]\mbox{}\\
A steepest descent algorithm for energy minimization.
The maximum stepsize is {\bf emstep} [nm], the tolerance is 
{\bf emtol} [kJ mol$^{-1}$ nm$^{-1}$].
\item[{\bf cg}]\mbox{}\\
 A conjugate gradient algorithm for energy minimization,
the tolerance is {\bf emtol} [kJ mol$^{-1}$ nm$^{-1}$]. 
CG is more efficient
when a steepest descent step is done every once in a while,
this is determined by {\bf nstcgsteep}.
\item[{\bf ld}]\mbox{}\\
 An Euler integrator for position Langevin dynamics, the
velocity is the force divided by a friction coefficient 
({\bf ld\_fric} [amu ps$^{-1}$])
plus random thermal noise ({\bf ld\_temp} [K]). 
The random generator is initialized with {\bf ld\_seed}
\end{description}
\item[{\bf tinit: }(0) {[ps]}]\mbox{}\\
starting time for your run (only makes sense for integrators {\bf md} 
and {\bf ld})
\item[{\bf dt: }(0.001) {[ps]}]\mbox{}\\
time step for integration (only makes sense for integrators {\bf md} 
and {\bf ld})
\item[{\bf nsteps: }(1)]\mbox{}\\
maximum number of steps to integrate
\item[{\bf nstcomm: }(1) {[steps]}]\mbox{}\\
if positive: frequency for center of mass motion removal
if negative: frequency for center of mass motion and rotational 
motion removal (should only be used for vacuum simulations)
\end{description}

\subsection{ Langevin dynamics}
\begin{description}
\item[{\bf ld\_temp: }(300) {[K]}]\mbox{}\\
temperature in ld run (controls thermal noise level)
\item[{\bf ld\_fric: }(0) {[amu ps$^{-1}$]}]\mbox{}\\
ld friction coefficient
\item[{\bf ld\_seed: }(1993) {[integer]}]\mbox{}\\
used to initialize random generator for thermal noise
when {\bf ld\_seed} is set to -1, the seed is calculated as
{\tt (time() + getpid()) \% 65536}
\end{description}

\subsection{ Energy minimization}
\begin{description}
\item[{\bf emtol: }(1.0) {[kJ mol$^{-1}$ nm$^{-1}$]}]\mbox{}\\
the minimization is converged when the maximum force is smaller than 
this value
\item[{\bf emstep: }(0.1) {[nm]}]\mbox{}\\
initial step-size
\item[{\bf nstcgsteep: }(1000) {[steps]}]\mbox{}\\
frequency of performing 1 steepest descent step while doing
conjugate gradient energy minimization.
\end{description}

\subsection{ Output control}
\begin{description}
\item[{\bf nstxout: }(100) {[steps]}]\mbox{}\\
frequency to write coordinates to output trajectory
\item[{\bf nstvout: }(100) {[steps]}]\mbox{}\\
frequency  to write velocities to output trajectory
\item[{\bf nstfout: }(0) {[steps]}]\mbox{}\\
frequency to write forces to output trajectory
\item[{\bf nstlog: }(100) {[steps]}]\mbox{}\\
frequency to write energies to log file
\item[{\bf nstenergy: }(100) {[steps]}]\mbox{}\\
frequency to write energies to energy file
\item[{\bf nstxtcout: }(0) {[steps]}]\mbox{}\\
frequency to write coordinates to xtc trajectory
\item[{\bf xtc\_precision: }(1000) {[real]}]\mbox{}\\
precision to write to xtc trajectory
\item[{\bf xtc\_grps:}]\mbox{}\\
group(s) to write to xtc trajectory, default the whole system is written
(if {\bf nstxtcout} is larger than zero)  
\item[{\bf energygrps:}]\mbox{}\\
group(s) to write to energy file
\end{description}

\subsection{ Neighborsearching}
\begin{description}
\item[{\bf nstlist: }(10) {[steps]}]\mbox{}\\
frequency to update neighborlist
\item[{\bf ns\_type:}]\mbox{}\\
\vspace{-2ex}\begin{description}
\item[{\bf grid}]\mbox{}\\
Make a grid in the box and only check atoms in neighboring 
grid cells when constructing a new neighborlist every {\bf nstlist} steps. 
The number of grid cells per Coulomb cut-off length is set with 
{\bf deltagrid},
this number should be 2 for optimal performance.
In large systems grid search is much faster than simple search. 
\item[{\bf simple}]\mbox{}\\
Check every atom in the box when constructing a new neighborlist
every {\bf nstlist} steps.
\end{description}
\item[{\bf deltagrid: }(2)]\mbox{}\\
number of grid cells per Coulomb cut-off length
\item[{\bf box:}]\mbox{}\\
\vspace{-2ex}\begin{description}
\item[{\bf rectangular}]\mbox{}\\
Selects a rectangular box shape.
\item[{\bf triclinic}]\mbox{}\\
(NOTE: not fully implemented) Selects a triclinic box shape.
\item[{\bf none}]\mbox{}\\
Selects no box, for use in vacuum simulations.
\end{description}
\item[{\bf rlist: }(1) {[nm]}]\mbox{}\\
cut-off distance for making the neighbor list
\end{description}

\subsection{ Electrostatics and VdW}
\begin{description}
\item[{\bf coulombtype:}]\mbox{}\\
\vspace{-2ex}\begin{description}
\item[{\bf Cut-off}]\mbox{}\\
Twin range cut-off's with neighborlist cut-off {\bf rlist} and 
Coulomb cut-off {\bf rcoulomb},
where {\bf rlist} {\tt $<$} {\bf rvdw} {\tt $<$} {\bf rcoulomb}.
The dielectric constant is set with {\bf epsilon\_r}.
\item[{\bf PPPM}]\mbox{}\\
Particle-Particle Particle-Mesh algorithm for long range
electrostatic interactions.
Use for example {\bf rlist}{\tt =1.0}, {\bf rcoulomb\_switch}{\tt =0.0},
{\bf rcoulomb}{\tt =0.85}, {\bf rvdw\_switch}{\tt =1.0}
and {\bf rvdw}{\tt =1.0}. The grid
dimensions are controlled by {\bf fourierspacing}.
Reasonable grid spacing for PPPM is 0.05-0.1 nm.
See {\tt Shift} for the details of the particle-particle potential.
NOTE: Pressure scaling is not possible with PPPM.
\item[{\bf Ewald}]\mbox{}\\
Classical Ewald sum electrostatics. Use e.g. {\bf rlist}=0.9,
{\bf rvdw}=0.9, {\bf rcoulomb}=0.9. The highest magnitude of
wavevectors used in reciprocal space is controlled by
{\bf fourier\_nx}, {\bf fourier\_ny}, {\bf fourier\_nz}. Reasonable
values are about 2 for each nm of box size. The relative accuracy of direct/reciprocal space
is controlled by {\bf ewald\_rtol}. NOTE: Ewald scales as O(N$^{3/2}$) and
is thus extremely slow for large systems. It is included mainly for
reference - in most cases PME will perform much better.
\item[{\bf PME} ]\mbox{}\\
Fast Particle-Mesh Ewald electrostatics. Direct space is similar
to the Ewald sum, while the reciprocal part is performed with
FFTs. Grid dimensions are controlled as for PPPM, and the
interpolation order with {\bf pme\_order}. With a grid spacing of 0.1
nm and (cubic interpolation) the electrostatic forces have an accuracy
of 2-3e-4. Since the error from the vdw-cutoff is larger than this you
might try 0.15 nm. When running in parallel the interpolation
parallelizes better than the FFT, so try decreasing grid dimensions
while increasing interpolation.
\item[{\bf Reaction-Field}]\mbox{}\\
Reaction field with Coulomb cut-off {\bf rcoulomb},
where {\bf rcoulomb} {\tt $>$} {\bf rvdw} {\tt $>$} {\bf rlist}.
The dielectric constant beyond the cut-off is {\bf epsilon\_r}.
The dielectric constant can be set to infinity by setting {\bf epsilon\_r}=0.
\item[{\bf Generalized-Reaction-Field}]\mbox{}\\
Generalized reaction field with Coulomb cut-off {\bf rcoulomb},
where {\bf rcoulomb} {\tt $>$} {\bf rvdw} {\tt $>$} {\bf rlist}.
The dielectric constant beyond the cut-off is {\bf epsilon\_r}.
The ionic strength is computed from the number of charged 
(i.e. with non zero charge) charge groups.
The temperature for the GRF potential is set with 
{\bf ref\_t} [K].
\item[{\bf Shift}]\mbox{}\\
The Coulomb
potential is decreased over the whole range and the forces decay smoothly
to zero between {\bf rcoulomb\_switch} and {\bf rcoulomb}.
The neighborsearch cut-off {\bf rlist} should be 0.1 to 0.3 nm larger than
{\bf rcoulomb} to accommodate for the size of charge groups and diffusion
between neighborlist updates.
\item[{\bf User}]\mbox{}\\
Specify {\bf rshort} and {\bf rlong} to the same value, {\tt mdrun}
will now expect to find a file {\tt ctab.xvg} with user-defined functions.
This files should contain 5 columns:
the {\tt x} value, and the function value with its 1$^{st}$
to 3$^{rd}$ derivative. The {\tt x} should run from 0 [nm] to
{\bf rlist}{\tt +0.5} [nm], with a spacing of {\tt 0.002}
[nm] when you run in single precision, or {\tt 0.0005} [nm] when
you run in double precision. The function value at {\tt x=0} is not
important.
\end{description}
\item[{\bf rcoulomb\_switch: }(0) {[nm]}]\mbox{}\\
where to start switching the Coulomb potential
\item[{\bf rcoulomb: }(1) {[nm]}]\mbox{}\\
distance for the Coulomb cut-off
\item[{\bf epsilon\_r: }(1)]\mbox{}\\
dielectric constant
\item[{\bf vdwtype:}]\mbox{}\\
\vspace{-2ex}\begin{description}
\item[{\bf Cut-off}]\mbox{}\\
Twin range cut-off's with neighborlist cut-off {\bf rlist} and 
VdW cut-off {\bf rvdw},
where {\bf rvdw} {\tt $>$} {\bf rlist}.
\item[{\bf Shift}]\mbox{}\\
The LJ (not Buckingham)
potential is decreased over the whole range and the forces decay smoothly
to zero between {\bf rvdw\_switch} and {\bf rvdw}.
The neighborsearch cut-off {\bf rlist} should be 0.1 to 0.3 nm larger than
{\bf rvdw} to accommodate for the size of charge groups and diffusion
between neighborlist updates.
\item[{\bf User}]\mbox{}\\
{\tt mdrun} will now expect to find two files with user-defined
functions: {\tt rtab.xvg} for Repulsion, {\tt dtab.xvg} for Dispersion.
These files should contain 5 columns:
the {\tt x} value, and the function value with its 1$^{st}$
to 3$^{rd}$ derivative. The {\tt x} should run from 0 [nm] to
{\bf rvdw}{\tt +0.5} [nm], with a spacing of {\tt 0.002}
[nm] when you run in single precision, or {\tt 0.0005} [nm] when
you run in double precision. The function value at {\tt x=0} is not
important. When you want to use LJ correction, make sure that {\bf rvdw}
corresponds to the cut-off in the user-defined function.
\end{description}
\item[{\bf rvdw\_switch: }(0) {[nm]}]\mbox{}\\
where to start switching the LJ potential
\item[{\bf rvdw: }(1) {[nm]}]\mbox{}\\
distance for the LJ or Buckingham cut-off
\item[{\bf bDispCorr:}]\mbox{}\\
\vspace{-2ex}\begin{description}
\item[{\bf no}]\mbox{}\\
don't apply any correction
\item[{\bf yes}]\mbox{}\\
apply long range dispersion corrections for Energy and Pressure
\end{description}
\item[{\bf gauss\_width }(0.1)]\mbox{}\\
for future use
\item[{\bf fourierspacing: }(0.12) {[nm]}]\mbox{}\\
The maximum grid spacing for the FFT grid when using PPPM or PME.
Each direction can be overriden by entering a non-zero value for
{\bf fourier\_n*}.
\item[{\bf fourier\_nx }(0){\bf  ; fourier\_ny }(0){\bf  ; fourier\_nz: }(0)]\mbox{}\\
Highest magnitude of wavevectors in reciprocal space when using Ewald.
Grid size when using PPPM or PME. These values override
{\bf fourierspacing} per direction. The best choice is powers of
2, 3, 5 and 7. Avoid large primes.
\item[{\bf pme\_order }(4)]\mbox{}\\
Interpolation order for PME. 4 equals cubic interpolation. You might try
6/8/10 when running in parallel and simultaneously decrease grid dimension.
\item[{\bf ewald\_rtol }(1e-5)]\mbox{}\\
The relative strength of the Ewald-shifted direct potential at the cutoff
is given by {\bf ewald\_rtol}. Decreasing this will give a more accurate
direct sum, but then you need more wavevectors for the reciprocal sum.
\item[{\bf optimize\_fft:}]\mbox{}\\
\vspace{-2ex}\begin{description}
\item[{\bf no}]\mbox{}\\
Don't calculate the optimal FFT plan for the grid at startup.
\item[{\bf yes}]\mbox{}\\
Calculate the optimal FFT plan for the grid at startup. This saves a
few percent for long simulations, but takes a couple of minutes
at start.
\end{description}
\end{description}

\subsection{ Temperature coupling}
\begin{description}
\item[{\bf tcoupl:}]\mbox{}\\
\vspace{-2ex}\begin{description}
\item[{\bf no}]\mbox{}\\
No temperature coupling. 
\item[{\bf yes}]\mbox{}\\
Temperature coupling with a Berendsen-thermostat to a bath with
temperature {\bf ref\_t} [K], with time constant {\bf tau\_t} [ps].
Several groups can be coupled seperately, these are specified in the
{\bf tc\_grps} field seperated by spaces.
\end{description}
\item[{\bf ntcmemory: }(1) {[steps]}]\mbox{}\\
memory for running average to couple to bath
\item[{\bf tc\_grps:}]\mbox{}\\
groups to couple separately to temperature bath
\item[{\bf tau\_t: }[ps]]\mbox{}\\
time constant for coupling (one for each group in tc\_grps)
\item[{\bf ref\_t: }[K]]\mbox{}\\
reference temperature for coupling (one for each group in tc\_grps)
\end{description}

\subsection{ Pressure coupling}
\begin{description}
\item[{\bf pcoupl:}]\mbox{}\\
\vspace{-2ex}\begin{description}
\item[{\bf no}]\mbox{}\\
No pressure coupling. This means a fixed box size.
\item[{\bf isotropic}]\mbox{}\\
Pressure coupling with time constant {\bf tau\_p} [ps].
The compressibility and reference pressure are set with
{\bf compressibility} [bar$^{-1}$] and {\bf ref\_p} [bar], one
value is needed.
\item[{\bf semiisotropic}]\mbox{}\\
Pressure coupling which is isotropic in the x and y direction,
but different in the z direction.
This can be useful for membrane simulations.
2 values are needed for x/y and z directions respectively.
\item[{\bf anisotropic}]\mbox{}\\
Idem, but 3 values are needed for x, y and z directions respectively.
Beware that isotropic scaling can lead to extreme deformation
of the simulation box.
\item[{\bf surface-tension}]\mbox{}\\
Surface tension couling for surfaces parallel to the xy-plane.
Uses normal pressure coupling for the z-direction, while the surface tension
is coupled to the x/y dimensions of the box.
The first {\bf ref\_p} value is the reference surface tension times
the number of surfaces [bar nm], 
the second value is the reference z-pressure [bar].
The two {\bf compressibility} [bar$^{-1}$] values are the compressibility
in the x/y and z direction respectively.
The value for the z-compressibility should be reasonably accurate since it
influences the converge of the surface-tension, it can also be set to zero
to have a box with constant height.
\item[{\bf triclinic}]\mbox{}\\
Not supported yet.
\end{description}
\item[{\bf npcmemory: }(1) {[steps]}]\mbox{}\\
memory for running average to couple
\item[{\bf tau\_p: }(1) {[ps]}]\mbox{}\\
time constant for coupling
\item[{\bf compressibility: }[bar$^{-1}$]]\mbox{}\\
compressibility (NOTE: this is now really in bar$^{-1}$)
\item[{\bf ref\_p: }[bar]]\mbox{}\\
reference pressure for coupling
\end{description}

\subsection{ Simulated annealing}
\begin{description}
\item[{\bf annealing:}]\mbox{}\\
\vspace{-2ex}\begin{description}
\item[{\bf no}]\mbox{}\\
No simulated annealing. 
\item[{\bf yes}]\mbox{}\\
Simulated annealing to 0 [K] at time {\bf zero\_temp\_time} (ps).
Reference temperature for the Berendsen-thermostat is
{\bf ref\_t} x (1 - time / {\bf zero\_temp\_time}),
time constant is {\bf tau\_t} [ps]. Note that the reference temperature
will not go below 0 [K], i.e. after {\bf zero\_temp\_time} (if it is positive) 
the reference temperature will be 0 [K]. Negative {\bf zero\_temp\_time} 
results in heating, which will go on indefenitely.
\end{description}
\item[{\bf zero\_temp\_time: }(0) {[ps]}]\mbox{}\\
time at which temperature will be zero (can be negative). Temperature
during the run can be seen as a straight line going through 
T={\bf ref\_t} [K] at t=0 [ps], and 
T=0 [K] at t={\bf zero\_temp\_time} [ps]. Look in our 
FAQ for a schematic 
graph of temperature vs. time.
\end{description}

\subsection{ Velocity generation}
\begin{description}
\item[{\bf gen\_vel:}]\mbox{}\\
\vspace{-2ex}\begin{description}
\item[{\bf no}]\mbox{}\\
 Do not generate velocities at startup. The velocities are set to zero
when there are no velocities in the input structure file.
\item[{\bf yes}]\mbox{}\\
Generate velocities according to a Maxwell distribution at
temperature {\bf gen\_temp} [K], with random seed {\bf gen\_seed}. 
This is only meaningful with integrator {\bf md}.
\end{description}
\item[{\bf gen\_temp: }(300) {[K]}]\mbox{}\\
temperature for Maxwell distribution
\item[{\bf gen\_seed: }(173529) {[integer]}]\mbox{}\\
used to initialize random generator for random velocities
\end{description}

\subsection{ Solvent optimization}
\begin{description}
\item[{\bf solvent\_optimization:}]\mbox{}\\
\vspace{-2ex}\begin{description}
\item[{\bf $<$empty$>$}]\mbox{}\\
Do not use water specific non-bonded optimizations
\item[{\bf $<$solvent molecule name$>$}]\mbox{}\\
Use water specific non-bonded optimizations. This string should match the
solvent molecule name in your topology. Check your run time to see 
if it is faster. 
\end{description}
\item[{\bf nsatoms: }(3)]\mbox{}\\
Number of atoms in solvent model.
(Not implemented for non-three atom models)
\end{description}

\subsection{ Bonds}
\begin{description}
\item[{\bf constraints:}]\mbox{}\\
\vspace{-2ex}\begin{description}
\item[{\bf none}]\mbox{}\\
No constraints, i.e. bonds are represented by a harmonic or a
morse potential (depending on the setting of {\bf morse}) and angles
by a harmonic potential.
\item[{\bf hbonds}]\mbox{}\\
Only constrain the bonds with H-atoms.
\item[{\bf all-bonds}]\mbox{}\\
Constrain all bonds.
\item[{\bf h-angles}]\mbox{}\\
Constrain all bonds and constrain the angles that involve H-atoms
by adding bond-constraints.
\item[{\bf all-angles}]\mbox{}\\
Constrain all bonds and constrain all angles by adding bond-constraints.
\end{description}
\item[{\bf constraint\_alg:}]\mbox{}\\
\vspace{-2ex}\begin{description}
\item[{\bf lincs}]\mbox{}\\
LINear Constraint Solver. The accuracy in set with
{\bf lincs\_order}, which sets the number of matrices in the expansion
for the matrix inversion, 4 is enough for a "normal" MD simulation, 8 is
needed for LD with large timesteps. If a bond rotates more than
{\bf lincs\_warnangle} [degrees] in one step, 
a warning will be printed both to the log file and to {\tt stderr}. 
Lincs should not be used with coupled angle constraints.
\item[{\bf shake}]\mbox{}\\
Shake is slower and less stable than Lincs, but does work with 
angle constraints. 
The relative tolerance is set with {\bf shake\_tol}, 0.0001 is a good value
for "normal" MD. 
\end{description}
\item[{\bf unconstrained\_start:}]\mbox{}\\
\vspace{-2ex}\begin{description}
\item[{\bf no}]\mbox{}\\
apply constraints to the start configuration
\item[{\bf yes}]\mbox{}\\
do not apply constraints to the start configuration
\end{description}
\item[{\bf shake\_tol: }(0.0001)]\mbox{}\\
relative tolerance for shake
\item[{\bf lincs\_order: }(4)]\mbox{}\\
Highest order in the expansion of the constraint coupling matrix.
{\bf lincs\_order} is also used for the number of Lincs iterations
during energy minimization, only one iteration is used in MD.
\item[{\bf lincs\_warnangle: }(30) {[degrees]}]\mbox{}\\
maximum angle that a bond can rotate before Lincs will complain
\item[{\bf nstlincsout: }(1000) {[steps]}]\mbox{}\\
frequency to output constraint accuracy in log file
\item[{\bf morse:}]\mbox{}\\
\vspace{-2ex}\begin{description}
\item[{\bf no}]\mbox{}\\
bonds are represented by a harmonic potential
\item[{\bf yes}]\mbox{}\\
bonds are represented by a morse potential
\end{description}
\end{description}

\subsection{ NMR refinement}
\begin{description}
\item[{\bf disre:}]\mbox{}\\
\vspace{-2ex}\begin{description}
\item[{\bf none}]\mbox{}\\
no distance restraints (ignore distance restraints information in
topology file)
\item[{\bf simple}]\mbox{}\\
simple (per-molecule) distance restraints
\item[{\bf ensemble}]\mbox{}\\
distance restraints over an ensemble of molecules
\end{description}
\item[{\bf disre\_weighting:}]\mbox{}\\
\vspace{-2ex}\begin{description}
\item[{\bf equal}]\mbox{}\\
divide the restraint force equally over all atom pairs in the restraint
\item[{\bf conservative}]\mbox{}\\
the forces are the derivative of the restraint potential,
this results in an r$^{-7}$ weighting of the atom pairs
\end{description}
\item[{\bf disre\_mixed:}]\mbox{}\\
\vspace{-2ex}\begin{description}
\item[{\bf no}]\mbox{}\\
the violation used in the calculation of the restraint force is the
time averaged violation 
\item[{\bf yes}]\mbox{}\\
the violation used in the calculation of the restraint force is the
square root of the time averaged violation times the instantaneous violation 
\end{description}
\item[{\bf disre\_fc: }(1000) {[kJ mol$^{-1}$ nm$^{-2}$]}]\mbox{}\\
force constant for distance restraints, which is multiplied by a
(possibly) different factor for each restraint
\item[{\bf disre\_tau: }(10) {[ps]}]\mbox{}\\
time constant for distance restraints running average
\item[{\bf nstdisreout: }(100) {[steps]}]\mbox{}\\
frequency to write the running time averaged and instantaneous distances
of all atom pairs involved in restraints to the energy file
(can make the energy file very large)
\end{description}

\subsection{ Free energy}
\begin{description}
\item[{\bf free\_energy:}]\mbox{}\\
\vspace{-2ex}\begin{description}
\item[{\bf no}]\mbox{}\\
Only use topology A. 
\item[{\bf yes}]\mbox{}\\
Change the system from topology A (lambda=0) to topology B (lambda=1)
and calculate the free energy difference.
The starting value of lambda is {\bf init\_lambda} the increase
per time step is {\bf delta\_lambda}.
\end{description}
\item[{\bf init\_lambda: }(0)]\mbox{}\\
starting value for lambda
\item[{\bf delta\_lambda: }(0)]\mbox{}\\
increase per timestep for lambda
\end{description}

\subsection{ Non-equilibrium MD}
\begin{description}
\item[{\bf acc\_grps: }]\mbox{}\\
groups for constant acceleration (e.g.: {\tt Protein Sol})
all atoms in groups Protein and Sol will experience constant acceleration
as specified in the {\bf accelerate} line
\item[{\bf accelerate: }(0) {[nm ps$^{-2}$]}]\mbox{}\\
acceleration for {\bf acc\_grps}; x, y and z for each group
(e.g. {\tt 0.1 0.0 0.0 -0.1 0.0 0.0} means that first group has constant 
acceleration of 0.1 nm ps$^{-2}$ in X direction, second group the 
opposite).
\item[{\bf freezegrps: }]\mbox{}\\
Groups that are to be frozen (i.e. their X, Y, and/or Z position will
not be updated; e.g. {\tt Lipid SOL}). {\bf freezedim} specifies for
which dimension the freezing applies.
\item[{\bf freezedim: }]\mbox{}\\
dimensions for which groups in {\bf freezegrps} should be frozen, 
specify {\tt Y} or {\tt N} for X, Y and Z and for each group
(e.g. {\tt Y Y N N N N} means that particles in the first group 
can move only in Z direction. The particles in the second group can 
move in any direction).
\end{description}

\subsection{ Electric fields}
\begin{description}
\item[{\bf E\_x }(0) [V nm$^{-1}$]{\bf  ; E\_y }(0) [V nm$^{-1}$]{\bf  ; E\_z: }(0) {[V nm$^{-1}$]}]\mbox{}\\
strength of constant electric field in resp. X, Y and Z direction
\item[{\bf E\_xt }{\bf  ; E\_yt }{\bf  ; E\_zt: }]\mbox{}\\
not implemented yet
\end{description}

\subsection{ User defined thingies}
\begin{description}
\item[{\bf user1\_grps }{\bf  ; user2\_grps }{\bf  ; user3\_grps: }]\mbox{}\\
\item[{\bf userint1 }(0){\bf  ; userint2 }(0){\bf  ; userint3 }(0){\bf  ; userint4: }(0)]\mbox{}\\
\item[{\bf userreal1 }(0){\bf  ; userreal2 }(0){\bf  ; userreal3 }(0){\bf  ; userreal4: }(0)]\mbox{}\\
These you can use if you hack out code. You can pass integers and
reals to your subroutine. Check the inputrec definition in
{\tt src/include/types/inputrec.h}
\end{description}



% LocalWords:  online html GMXRC MANPATH grompp xdr NFS
