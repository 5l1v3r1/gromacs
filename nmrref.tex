\documentclass[11pt,dvips]{article}
\usepackage{here,picins,fancy,epsfig,array,tabularx,makeidx,vmargin,floatflt}
\setpapersize{A4}
%
% $Id$
% 
%       This source code is part of
% 
%        G   R   O   M   A   C   S
% 
% GROningen MAchine for Chemical Simulations
% 
%               VERSION 2.0
% 
% Copyright (c) 1991-1999
% BIOSON Research Institute, Dept. of Biophysical Chemistry
% University of Groningen, The Netherlands
% 
% Please refer to:
% GROMACS: A message-passing parallel molecular dynamics implementation
% H.J.C. Berendsen, D. van der Spoel and R. van Drunen
% Comp. Phys. Comm. 91, 43-56 (1995)
% 
% Also check out our WWW page:
% http://md.chem.rug.nl/~gmx
% or e-mail to:
% gromacs@chem.rug.nl
% 
% And Hey:
% Gnomes, ROck Monsters And Chili Sauce
%

\setlength {\parindent}{0.0cm}
\setlength {\parskip}{1ex}
\newcommand{\ve}[1]{\mbox{\boldmath ${#1}$}} 
  % defines bold italic vectors. To be used in text or math mode.
  % Example: \ve{F}
\newcommand{\de}{\mbox{d}} 
  % defines a straight d for derivatives.
\newcommand{\intel}{Intel {\em i\/}860}
\newcommand{\gromacs}{{\sl GROMACS\/}}
\newcommand{\gromosv}[1]{{\sl GROMOS-#1\/}}
\newcommand{\gromos}{{\sl GROMOS\/}}
\newcommand{\dline}{\hline\hline}
\newcommand{\etal}{{\em et al.}}
\newcommand{\beq}{\begin{equation}}
\newcommand{\eeq}{\end{equation}}
\newcommand{\bea}{\begin{eqnarray}}
\newcommand{\eea}{\end{eqnarray}}
\newcommand{\Dt}{{\Delta t}}
\newcommand{\hDt}{{\frac {\Delta t}{2}}}
\newcommand{\half}{\frac{1}{2}}
\newcommand{\rvi}{\ve{r}_i}
\newcommand{\rvj}{\ve{r}_j}
\newcommand{\rvk}{\ve{r}_k}
\newcommand{\rij}{r_{ij}}
\newcommand{\rvij}{\ve{r}_{ij}}
\newcommand{\rnorm}{\frac{\rvij}{\rij}}
\newcommand{\Fvi}{\ve{F}_i}
\newcommand{\Fvj}{\ve{F}_j}
\newcommand{\Fvk}{\ve{F}_k}
\newcommand{\Fvij}{\ve{F}_{ij}}
\newcommand{\Fvji}{\ve{F}_{ji}}
\newcommand{\vvi}{\ve{v}_i}
\newcommand{\al}{\alpha}
\newcommand{\be}{\beta}
\newcommand{\ab}{\alpha\beta}
\newcommand{\rnij}{\ve{r}_{ij}^n}
\newcommand{\rni}{\ve{r}_i^n}
\newcommand{\hdt}{\frac{\Delta t}{2}}
\newcommand{\type}[1]{\\ {\tt \% #1}\\}
\newcommand{\normindex}[1]{#1\lowercase{\index{#1}}}
\newcommand{\boldindex}[1]{#1\index{#1@\textbf{#1}}}
\newcommand{\seeindex}[2]{#1\lowercase{\index{#1|see{#2}}}}
\newcommand{\eg}{e.g.\@}
\newcommand{\ie}{i.e.\@}
\newcommand{\mc}[3]{\multicolumn{#1}{#2}{#3}}

% Commands for correct spacing in tables
% 	TeX and TUG News, Vol.2, No.3, p10, 1993. 
%
\newcommand\T{\rule{0pt}{2.6ex}}         % Top strut
\newcommand\B{\rule[-1.2ex]{0pt}{0pt}}   % Bottom strut
\newcommand{\Ts}{\rule{0pt}{2.4ex}}      % Smaller top strut (To be used in 
                                         % math mode in \frac, together with 
                                         % \displaystyle

\newcommand{\captspace}{\vspace{2mm}}

\renewcommand{\type}[1]{\\ {\tt \% #1 $\hookleftarrow$}}

\begin{document}
\title{\bf Structure Refinement Based on NOE Data from NMR Experiment using GROMACS Software}
\author{David van der Spoel}
\maketitle

\section{Introduction}
This document is part of the GBB introduction course. It describes some details
of NMR structure refinement using computer simulations techniques.
Furthermore, it has some practical exercises, that demonstrate the most
important parameters for refinement.

\section{Theory}
\label{sec:disre}
\normindex{Distance restraints} 
add a penalty to the potential when the distance
between specified pairs of atoms exceeds a threshold value. They are
normally used to impose experimental restraints, as from \normindex{NOE}
experiments in nuclear magnetic resonance (NMR), on the motion of the
system. Thus MD can be used for \normindex{structure refinement}  using 
\normindex{NMR} data. The
potential form is quadratic between two specified distances and linear
beyond the largest distance (see figure~\ref{fig:dist}).
\bea
V_{dr}(r_{ij})~=~&~0 \hspace{4cm}                  & r_{ij} < r_0         \\
	       =~&~\half k_{dr}(r_{ij}-r_0)^2	           & r_0 \le r_{ij} < r_1 \\
	       =~&~\half k_{dr}(r_1-r_0)(2r_{ij}-r_1-r_0) & r_{ij} \ge r_1
\label{eqn:disre}
\eea

\begin {figure}
\centerline{\psfig {figure=plots/f_dr.eps,angle=270,width=8cm}}
\caption {Distance Restraint potential}
\label{fig:dist}
\end {figure}

The forces are
\bea
\ve{F}_i~=~&~0 \hspace{4cm}  & r_{ij} < r_0         \\
  = ~&-~k_{dr}(r_{ij}-r_0)\frac{\ve{r}_{ij}}{r_{ij}} & r_0 \le r_{ij} < r_1 \\
  = ~&-~k_{dr}(r_1-r_0)\frac{\ve{r}_{ij}}{r_{ij}}    & r_{ij} \ge r_1	
\eea
The above formulae have been improved by introducing the
{\em time averaged} distance restraint~\cite{Torda89} where the distance
$r_{ij}$ is replaced by a time average:
\beq
\bar{r}_{ij} = < r_{ij}^{-3} >^{-1/3}
\label{eqn:rav}
\eeq
This way an atom can satisfy two incompatible distance restraints 
{\em on average} by moving between two protons. 
An example would be an amino-acid sidechain which is rotating around
its $\chi$ dihedral angle, thereby coming close to various other groups.
Such a mobile side chain may give rise to multiple NOEs. The use of
time averaged restraints allows the sidechain to fulfill all
experimental restraints.

The computation of the time
averaged distance in the {\tt mdrun} program is done in the following fashion:
\bea
\bar{r}_{ij}(0)^{-3} 	&=& r_{ij}(0)^{-3}	\\
\bar{r}_{ij}(t)^{-3}	&=& \bar{r}_{ij}(t-\Delta t)^{-3}~\exp{\left(-\frac{\Delta t}{\tau}\right)} + r_{ij}(t)^{-3}\left[1-\exp{\left(-\frac{\Delta t}{\tau}\right)}\right]
\label{eqn:ravdisre}
\eea
Sometimes it is unclear from experimental data which atom pair
gives rise to a single NOE, in other occasions it can be assumed that
more than one pair contributes due to the symmetry of the system, e.g. a
methyl group with three protons. For such a group it is not possible 
to distinguish between the protons, therefore they should all be taken into
account when calculating the distance between this methyl group and another
proton (or group of protons).
Due to the physical nature of magnetic resonance, the intensity of the
NOE signal is proportional to the distance between atoms to the power of -6.
Thus, when combining atom pairs, 
a fixed list of $N$ restraints may be taken together, 
where the net ``distance'' is given by:
\beq
r_{N}(t) = \left [\sum_{n=1}^{N} \bar{r}_{n}(t)^{-6} \right]^{-1/6}
\label{eqn:rsix}
\eeq
where we use eqn.\ref{eqn:rav} for the $\bar{r}_{n}$.

As more pairs of protons contribute to the same NOE signal, the intensity
will increase, and the summed ``distance'' will be shorter than any of
its components due to the reciprocal summation. 
It is also possible to use {\em ensemble averaging} using multiple
(protein)  molecules. In this case the bounds should be lowered as in:
\bea
r_0	&~=~&	r_0 * M^{-1/6}	\\
r_1	&~=~&	r_1 * M^{-1/6}
\eea
where $M$ is the number of molecules. The {\gromacs} preprocessor {\tt grompp}
can do this automatically when the appropriate option is given.
The resulting ``distance'' is 
then used to calculate the scalar force according to:
\bea
\ve{F}_i~=~&~0 \hspace{4cm}  & r_{N} < r_0         \\
  = ~&-~ k_{dr}(r_{N}-r_0)\frac{\ve{r}_{ij}}{r_{ij}} & r_0 \le r_{N} < r_1 \\
  = ~&-~ k_{dr}(r_1-r_0)\frac{\ve{r}_{ij}}{r_{ij}}    & r_{N} \ge r_1	
\eea
where $i$ and $j$ denote the atoms  of all the 
pairs that contribute to the NOE signal.

\section{Methods}
The {\gromacs} software~\cite{Berendsen95a,gmx2} 
can be used to do NMR structure refinements.
A list of distance restrains based on NOE data can be added in your
topology file, like in the following example:
\begin{verbatim}
[ distance_restraints ]
; ai    aj      type    index   rt-aver rx0      rx1     rx2     rx3
10      16      1       0       1       0.3      0.4     0.0     0.0 
10      28      1       1       1       0.3      0.4     0.0     0.0 
10      46      1       1       1       0.3      0.4     0.0     0.0 
16      22      1       2       1       0.3      0.4     0.0     0.0 
16      34      1       3       1       0.5      0.6     0.0     0.0 
\end{verbatim}
In this example a number of features can be found.
In columns {\tt ai} and {\tt aj} you find the atom numbers of the
particles to be restrained. The {\tt type} column should always be 1.
As explained in section~\ref{sec:disre}, multiple distances can add
to a single NOE signal. In the topology this can be set using the
{\tt index} column. In our example, the restraints 10-28 and 10-46
both have index 1, therefore they are treated simultaneously.
An extra requirement for treating restraints together, is that 
the restraints should be on successive lines, without any
other intervening restraint. The columns {\tt rx0} and {\tt rx1} hold
the values of $r_0$ and $r_1$ from eq.~\ref{eqn:disre}.
The columns {\tt rx2} and {\tt rx3} hold variables, which are as of yet 
not used. They must be specified in the topology file however.

Some more parameters are important for NMR refinement using
distance restraints in {\gromacs}.
\begin{description}
\item[Force constant $k_{dr}$ for distance restraints.] 
	$k_{dr}$  (eqn~\ref{eqn:disre}) can be set in the {\tt grompp.mdp} file
	as variable {\tt disre\_fc = 1000} for a force constant of
	1000 {kJ mole$^{-1}$ nm$^{-2}$}.
\item[Time constant $\tau$ for restraints.] 
	$\tau$ (eqn~\ref{eqn:ravdisre}) can be set in the {\tt grompp.mdp} file
	as variable {\tt disre\_tau = 10} for a time constant of
	10 ps.
\item[Ensemble averaging.] When multiple proteins or peptides are used
	in the simulation ensemble averaging 
	can be turned on by passing the option
	{\tt -ensemble} to {\tt grompp}, the {\gromacs}
	preprocessor.
\end{description}

\section{Exercises}
Using a simple peptide we will try to demonstrate the features
of the {\gromacs} software for NMR refinement.
The peptide we will use is Tyr-Ala-Ser-Thr.
In this peptide we will consider restraints 
due to NOES between the Tyr-HD and Thr-HG protons (6 restraints)
and between the Tyr-HD and the Ser-HB protons (4 restraints).
Since the all these protons are 
usually indistinguishable in a NMR experiment, we will sum the 6 and the
4 restraints
corresponding to these NOES (eqn~\ref{eqn:rsix}) in our refinements.

Now execute the following commands.
\type{cp -r ~gbbcursus/gmx/nmr nmr}
\type{cd nmr/test1}
\type{source ~gbbcursus/gmx/GMXRC}

You can view the peptide on screen with rasmol
\type{rasmol pep.pdb}

Before doing a simulation we must process the input data using
the {\gromacs} preprocessor program {\tt grompp}. This program needs
the following input files:
\begin{enumerate}
\item	a coordinate file ({\tt conf.gro})
\item	a {\em topology} file ({\tt topol.top}) that holds information on the
	covalent bonds, atomic charges and masses etc., and also the 
	distance restraint information (we'll come back to that later).
\item	a parameter file ({\tt grompp.mdp})
\item	an index file ({\tt index.ndx})
\end{enumerate}
It produces a single output file {\tt topol.tpb}.
For the purpose of this exercise we will modify two of these input files
using a text editor (e.g. {\tt jot}).

\subsection{Som 1}
The input files are prepared for the first exercise, so type:
\type{grompp}
\type{mdrun -v}

This will take a few seconds. The mdrun is 10 ps in vacuum and it contains
distance restraints without time averaging ($\tau$ = 0).
You can view the trajectory on screen with the {\tt ngmx} program:
\type{ngmx -f ctraj}

Select {\tt System} from the initial dialog box, and click {\tt OK}.
You can scale and rotate the molecule using the left and right mouse
buttons. Then, select {\tt Animate} from the {\tt Display} menu.
You wil see a VCR like control panel at the bottom of the window, press
the $>>$ button to replay the trajectory.
A somewhat more quantitative analysis can be done by calculating and
plotting the sum of violations:
\type{g\_energy -f ener -rsum -w}

Please note the values plotted on your screen:
\begin{verbatim}
Sum of violations averaged over simulation: 0.062522 nm
Largest violation averaged over simulation: 0.0500902 nm
\end{verbatim}
we will compare these numbers to the ones found in the next few sections.

\subsection{Som 2}
Edit the {\tt grompp.mdp} file to set the time averaging constant $\tau$ to
5.0 (ps). You can edit the file using
\type{jot grompp.mdp}

Redo the mdrun (first run {\tt grompp}!) and redo the analysis.
Compare the sum of violations to the ones from the previous run.
When you look at the graph you can see that the running average is not the
same as the instantaneous violations, like in the previous run.

\subsection{Som 3}
Go to the other test dir:
\type{cd ../test2}

Look at the molecule again using rasmol:
\type{rasmol pep.pdb}

As you see, there are two identical peptides. We will use 
time averaging {\em and} ensemble averaging
to see if we can satisfy the restraints by using two different
conformations.
\type{grompp -ensemble}

And redo the mdrun and the analysis again as you did before.

Now try to tune the parameters, ($\tau$, $k_{dr}$) to get better compliance
with the NOE data. You can also increase the number of MD steps
({\tt nsteps}) to see if this will help. Discuss the results...

\bibliographystyle{proteins}
\bibliography{bibtex/monster}

\end{document}
