\section{Image Calculation in gromacs.}
{\gromacs} uses an \myindex{interaction list} for nonbonded interactions,
usually called the {\em \myindex{neighbourlist}}.
This list is made every {\tt nstlist} MD steps, where {\tt nstlist} is
typically 10 MD steps. 
To make the neighbourlist all particles that are close 
(i.e. within the cut-off) to a given particle must be found.
This searching involves peridic boundary conditions, and 
determining the {\em image} (see Sec.~\ref{sec:pbc}).
When the cut-off is large compared to the box edge $l$ ($>$ 0.4$l$)
searching is done using an $O(N^2)$ algorithm that computes
all distances and compares them to the cut-off.
When the cut-off is smaller than 0.4$l$ in all directions (x,y and z)
searching is done using a grid. All particles are put on a grid
with the smallest spacing that is $>=$ 0.5 cut-off in each of the directions.
We have depicted the computational box, divided into grid-cells in 
Figure~\ref{fig:nsgrid}.
\begin{figure}[ht]
\centerline{\psfig{figure=plots/nsgrid.eps,width=6cm}}
\caption{The computational box in two dimensions, divided into grid-cells with three particles, $i$, $j$ and $k$. Each grid-cell is of size $>=$ 0.5 cut-off.}
\label{fig:nsgrid}
\end{figure}
In each spatial dimension, a particle $i$ has three images. For each direction
the image may be -1,0 or 1, corresponding to a translation over -1, 0 or +1
box vector. We do not search the surrounding grid-cells for
neighbours of $i$ and then calculate the image, 
but rather construct the images first and then 
search neighbours corresponding to that image of $i$.
Since we demand that the number of grid cells $>=$ 5 in each direction 
the same neighbour will not be found twice.

In the example of Figure~\ref{fig:nsgrid} the image $t_x$ = 0 of particle
$i$ will find $j$ as a neighbour, while image $t_x$ = 1 of particle $i$
will find $k$ as a neighbour.
