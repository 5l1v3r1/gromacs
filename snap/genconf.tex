\newpage
\section{genconf.}
{\bf VERSION 1.2\\Fri Feb 23 14:38:10 1996
}

\subsubsection*{Description.}
genconf multiplies a given coordinate file by simply stacking them
on top of each other, like a small child playing with wooden blocks.
The program makes a grid of {\em user defined}
proportions (nx, ny, nz in the input file), 
and interspaces the grid point with an extra space dx,dy and dz.
\subsubsection*{Files.}
\begin{table}[ht]
\begin{tabularx}{\linewidth}{llX}
-f & genconf.gcp & genconf input file with parameters etc. \\
-po & genconf.gcp & genconf input file with parameters etc. \\
-ci & conf.gro & Coordinate file in Gromos-87 format \\
-co & conf.gro & Coordinate file in Gromos-87 format \\
\end{tabularx}
\end{table}
Remember that filenames are not fixed, but 
file extensions are.
\subsubsection*{Other options.}
\begin{table}[ht]
\begin{tabularx}{\linewidth}{lX}
-r & Randomize, molecules are randomly rotated on their grid point.\\
-h & Print help information for genconf\\
\end{tabularx}
\end{table}
\subsubsection*{Diagnostics.}
\begin{itemize}
\item	When option -r is used (randomize) the program does not check for overlap between molecules on grid points. It is recommended to make the box in the input file at least as big as the coordinates + Vander Waals radius. 
\item	The program should be more flexible, to allow for random displacement off lattice points (for each cartesian coordinate), and specify the (maximum) random rotation per coordinate to be useful for building membranes.
\end{itemize}
