%
% $Id$
% 
%       This source code is part of
% 
%        G   R   O   M   A   C   S
% 
% GROningen MAchine for Chemical Simulations
% 
%               VERSION 2.0
% 
% Copyright (c) 1991-1999
% BIOSON Research Institute, Dept. of Biophysical Chemistry
% University of Groningen, The Netherlands
% 
% Please refer to:
% GROMACS: A message-passing parallel molecular dynamics implementation
% H.J.C. Berendsen, D. van der Spoel and R. van Drunen
% Comp. Phys. Comm. 91, 43-56 (1995)
% 
% Also check out our WWW page:
% http://md.chem.rug.nl/~gmx
% or e-mail to:
% gromacs@chem.rug.nl
% 
% And Hey:
% Gnomes, ROck Monsters And Chili Sauce
%

\chapter{Some implementation details}
In this chapter we will present some implementation details. This is
far from complete, but we deemed it necessary to clarify some things
that would otherwise be hard to understand.

\section{Single Sum Virial in {\gromacs}.}
\label{sec:virial}
The \normindex{virial} $\Xi$ can be written in full tensor form as:
\beq
\Xi~=~-\half~\sum_{i < j}^N~\rvij\otimes\Fvij
\eeq
where $\otimes$ denotes the {\em direct product} of two vectors\footnote
{$({\bf u}\otimes{\bf v})^{\ab}~=~{\bf u}_{\al}{\bf v}_{\be}$}. When this is 
computed in the inner loop of an MD program 9 multiplications and 9
additions are needed\footnote{The calculation of 
Lennard-Jones and Coulomb forces is about 50 floating point operations.}.

Here it is shown how it is possible to extract the virial calculation
from the inner loop~\cite{Bekker93b}.

\subsection{Virial.}
In a system with \swapindex{Periodic}{Boundary Conditions}, the
periodicity must be taken into account for the virial:
\beq
\Xi~=~-\half~\sum_{i < j}^{N}~\rnij\otimes\Fvij
\eeq
where $\rnij$ denotes the distance vector of the
{\em nearest image} of atom $i$ from atom $j$. In this definition we add
a {\em shift vector} $\delta_i$ to the position vector $\rvi$ 
of atom $i$. The difference vector $\rnij$ is thus equal to:
\beq
\rnij~=~\rvi+\delta_i-\rvj
\eeq
or in shorthand:
\beq
\rnij~=~\rni-\rvj
\eeq
In a triclinic system there are 27 possible images of $i$, when truncated 
octahedron is used there are 15 possible images.

\subsection{Virial from non-bonded forces.}
Here the derivation for the single sum virial in the {\em non-bonded force} 
routine is given. $i \neq j$ in all formulae below.
\newcommand{\di}{\delta_{i}}
\newcommand{\qrt}{\frac{1}{4}}
\bea
\Xi	
&~=~&-\half~\sum_{i < j}^{N}~\rnij\otimes\Fvij				\\
&~=~&-\qrt\sum_{i=1}^N~\sum_{j=1}^N ~(\rvi+\di-\rvj)\otimes\Fvij	\\
&~=~&-\qrt\sum_{i=1}^N~\sum_{j=1}^N ~(\rvi+\di)\otimes\Fvij-\rvj\otimes\Fvij	\\
&~=~&-\qrt\left(\sum_{i=1}^N~\sum_{j=1}^N ~(\rvi+\di)\otimes\Fvij~-~\sum_{i=1}^N~\sum_{j=1}^N ~\rvj\otimes\Fvij\right)	\\
&~=~&-\qrt\left(\sum_{i=1}^N~(\rvi+\di)\otimes\sum_{j=1}^N~\Fvij~-~\sum_{j=1}^N ~\rvj\otimes\sum_{i=1}^N~\Fvij\right)	\\
&~=~&-\qrt\left(\sum_{i=1}^N~(\rvi+\di)\otimes\Fvi~+~\sum_{j=1}^N ~\rvj\otimes\Fvj\right)	\\
&~=~&-\qrt\left(2~\sum_{i=1}^N~\rvi\otimes\Fvi+\sum_{i=1}^N~\di\otimes\Fvi\right)
\eea
In these formulae we introduced
\bea
\Fvi&~=~&\sum_{j=1}^N~\Fvij					\\
\Fvj&~=~&\sum_{i=1}^N~\Fvji
\eea
which is the total force on $i$ resp. $j$. Because we use Newton's third law
\beq
\Fvij~=~-\Fvji
\eeq
we must in the implementation double the term containing the shift $\delta_i$.

\subsection{The intramolecular shift (mol-shift).}
For the bonded-forces and shake it is possible to make a {\em mol-shift}
list, in which the periodicity is stored. We simple have an array {\tt mshift}
in which for each atom an index in the {\tt shiftvec} array is stored.

The algorithm to generate such a list can be derived from graph theory,
considering each particle in a molecule as a bead in a graph, the bonds 
as edges.
\begin{enumerate}
\item[1]	represent the bonds and atoms as bidirectional graph
\item[2]	make all atoms white
\item[3]	make one of the white atoms black (atom $i$) and put it in the
		central box
\item[4]	make all of the neighbors of $i$ that are currently 
		white, grey 
\item[5]	pick one of the grey atoms (atom $j$), give it the
		correct periodicity with respect to any of 
		its black neighbors
		and make it black
\item[6]	make all of the neighbors of $j$ that are currently 
		white, grey
\item[7]	if any grey atom remains, goto [5]
\item[8]	if any white atom remains, goto [3]
\end{enumerate}
Using this algorithm we can 
\begin{itemize}
\item	optimize the bonded force calculation as well as shake
\item	calculate the virial from the bonded forces
	in the single sum way again
\end{itemize}

Find a representation of the bonds as a bidirectional graph.

\subsection{Virial from Covalent Bonds.}
The covalent bond force gives a contribution to the virial, we have
\bea
b	&~=~&	\|\rnij\|					\\
V_b	&~=~&	\half k_b(b-b_0)^2				\\
\Fvi	&~=~&	-\nabla V_b					\\
	&~=~&	k_b(b-b_0)\frac{\rnij}{b}			\\
\Fvj	&~=~&	-\Fvi
\eea
The virial contribution from the bonds then is
\bea
\Xi_b	&~=~&	-\half(\rni\otimes\Fvi~+~\rvj\otimes\Fvj)	\\
	&~=~&	-\half\rnij\otimes\Fvi
\eea

\subsection{Virial from Shake.}
An important contribution to the virial comes from shake. Satisfying 
the constraints a force {\bf G} is exerted on the particles shaken. If this
force does not come out of the algorithm (as in standard shake) it can be
calculated afterwards (when using {\em leap-frog}) by:
\bea
\Delta\rvi&~=~&\rvi(t+\Dt)-
[\rvi(t)+{\bf v}_i(t-\frac{\Dt}{2})\Dt+\frac{\Fvi}{m_i}\Dt^2]	\\
{\bf G}_i&~=~&\frac{m_i\Delta\rvi}{\Dt^2}
\eea
but this does not help us in the general case. Only when no periodicity
is needed (like in rigid water) this can be used, otherwise
we must add the virial calculation in the inner loop of shake.

When it {\em is} applicable the virial can be calculated in the single sum way:
\beq
\Xi~=~-\half\sum_i^{N_c}~\rvi\otimes\Fvi
\eeq
where $N_c$ is the number of constrained atoms.

%Another method is the Non-Iterative shake as proposed (and implemented)
%by Yoneya. In this algorithm the Lagrangian multipliers are solved in a 
%matrix equation, and given these multipliers it is easy to get the periodicity
%in the virial afterwards. 

%More...


\section{Optimizations}
Here we describe some of the algorithmic optimizations used 
in {\gromacs}, apart from parallelism. 
One of these, the implementation of the 
1.0/sqrt(x) function is treated separately in \secref{sqrt}.
The most important other optimizations are described below.

\subsection{Inner Loops for Water}
{\gromacs} users special inner loop to calculate non-bonded
interactions for water molecules with other atoms, and yet
another set of loops for interactions between pairs of
water molecules. This very optimized loop assumes 
a water model similar to
SPC~\cite{Berendsen81}, {\ie}:
\begin{enumerate}
\item   There are three atoms in the molecule.
\item   The first atom has Lennard-Jones (\secref{lj}) and 
        coulomb (\secref{coul}) interactions.
\item   Atoms two and three have only coulomb interactions, 
        and equal charges.
\end{enumerate}

The loop also works for the SPC/E~\cite{Berendsen87} and 
TIP3P~\cite{Jorgensen83} water models. For more complicated molecules
there is a general solvent loop assuming
(note the order):
\begin{enumerate}
\item   At the beginning of the molecule topology there is
        an arbitrary number of atoms with Lennard-Jones and
        coulomb interactions.
\item   Then we have an arbitrary number of atoms with coulomb interactions only.
\item   And finally there can be an arbitrary number of 
        atoms with Lennard-Jones interactions only.
\end{enumerate}
Note that this loop provides much less optimization than the water
loop, but it is slightly better than the default routine.

The gain of these implementations is that there are more floating point
operations in a single loop, which implies that some compilers
can schedule the code better. However, it turns out that even
some of the most advanced compilers have problems with scheduling, 
implying that manual tweaking is necessary to get optimum 
\normindex{performance}.
This may include common-subexpression elimination, or moving
code around. 

\subsection{Fortran Code}
Unfortunately, \normindex{Fortran} compilers are still better than
C-compilers, for most machines anyway. For some machines ({\eg} SGI
Power Challenge) the difference may be up to a factor of 3, in the
case of vector computers this may be even larger. Therefore, some of
the routines that take up a lot of computer time have been translated
into Fortran and even assembly code for Intel and AMD x86 processors.
In most cases, the Fortran or assembly loops should be selected 
automatically by the configure script when appropriate, but you can
also tweak this by setting options to the configure script.

\section{Computation of the 1.0/sqrt function.}
\label{sec:sqrt}
\subsection{Introduction.}
The {\gromacs} project started with the development of a $1/\sqrt{x}$
processor which calculates
\begin{equation}
Y(x) = \frac{1}{ \sqrt{x} }
\end{equation}
As the project continued, the {\intel} processor was used to implement
{\gromacs}, which now turned into almost a full software project.  The
$1/\sqrt{x}$ processor was implemented using a Newton-Raphson
iteration scheme for one step. For this it needed lookup tables to
provide the initial approximation. The $1/\sqrt{x}$ function makes it
possible to use two almost independent tables for the exponent seed
and the fraction seed with the IEEE floating point representation.

\subsection{General}
According to~\cite{Bekker87} the $1/\sqrt{x}$ can be calculated using
the Newton-Raphson iteration scheme. The inverse function is
\begin{equation}
X(y) = \frac{1}{y^{2}}
\end{equation}
So instead of calculating
\begin{equation}
Y(a) = q
\end{equation}
the equation
\begin{equation}
X(q) - a = 0
\label{eqn:inversef}
\end{equation}
can now be solved using Newton-Raphson. An iteration is performed by
calculating
\begin{equation}
y_{n+1} = y_{n} - \frac{f(y_{n})}{f'(y_{n})}
\label{eqn:nr}
\end{equation}
The absolute error $\varepsilon$, in this approximation is defined by
\begin{equation}
\varepsilon \equiv y_{n} - q
\end{equation}
using Taylor series expansion to estimate the error results in
\begin{equation}
\varepsilon _{n+1} = - \frac{\varepsilon _{n}^{2}}{2}
                       \frac{ f''(y_{n})}{f'(y_{n})}
\label{eqn:taylor}
\end{equation}
according to~\cite{Bekker87} equation (3.2). This is an estimation of the
absolute error.

\subsection{Applied to floating point numbers}
Floating point numbers in IEEE 32 bit single precision format have a nearly
constant relative error of $\Delta x / x = 2^{-24}$. As seen earlier in the
Taylor series expansion equation (\eqnref{taylor}), the error in every
iteration step is absolute and in general dependent of $y$. If the error is
expressed as a relative error $\varepsilon_{r}$ the following holds
\begin{equation}
\varepsilon _{{r}_{n+1}} \equiv \frac{\varepsilon_{n+1}}{y}
\end{equation}
and so
\begin{equation}
\varepsilon _{{r}_{n+1}} =
- ( \frac{\varepsilon _{n}}{y} )^{2} y \frac{ f''}{2f'}
\end{equation}
for the function $f(y) = y^{-2}$ the term $y f''/2f'$ is constant (equal
to $-3/2$) so the relative error $\varepsilon _{r_{n}}$ is independent of $y$.
\begin{equation}
\varepsilon _{{r}_{n+1}} =
\frac{3}{2} (\varepsilon_{r_{n}})^{2}
\label{eqn:epsr}
\end{equation}

The conclusion of this is that the function $1/\sqrt{x}$ can be
calculated with a specified accuracy.

\begin{figure}
\begin{center}
\newcommand{\twltt}{\tt}
\setlength{\unitlength}{0.0080in}
\begin{picture}(489,176)(40,390)
\thicklines
\put(180,505){$\underbrace{\hspace{2.68in}}$}
\put( 60,505){$\underbrace{\hspace{0.88in}}$}
\put( 40,510){\framebox(480,30){}}
\put( 45,505){\vector( 0,-1){ 15}}
\put( 40,540){\dashbox{4}(0,0){}}
\multiput(250,540)(0.00000,-8.57143){4}{\line( 0,-1){  4.286}}
\multiput(220,540)(0.00000,-8.57143){4}{\line( 0,-1){  4.286}}
\multiput(190,540)(0.00000,-8.57143){4}{\line( 0,-1){  4.286}}
\multiput(280,540)(0.00000,-8.57143){4}{\line( 0,-1){  4.286}}
\multiput(310,540)(0.00000,-8.57143){4}{\line( 0,-1){  4.286}}
\multiput(340,540)(0.00000,-8.57143){4}{\line( 0,-1){  4.286}}
\multiput(370,540)(0.00000,-8.57143){4}{\line( 0,-1){  4.286}}
\multiput(400,540)(0.00000,-8.57143){4}{\line( 0,-1){  4.286}}
\multiput(430,540)(0.00000,-8.57143){4}{\line( 0,-1){  4.286}}
\multiput(460,540)(0.00000,-8.57143){4}{\line( 0,-1){  4.286}}
\multiput(490,540)(0.00000,-8.57143){4}{\line( 0,-1){  4.286}}
\multiput(205,540)(0.00000,-8.57143){4}{\line( 0,-1){  4.286}}
\multiput(235,540)(0.00000,-8.57143){4}{\line( 0,-1){  4.286}}
\multiput(265,540)(0.00000,-8.57143){4}{\line( 0,-1){  4.286}}
\multiput(295,540)(0.00000,-8.57143){4}{\line( 0,-1){  4.286}}
\multiput(325,540)(0.00000,-8.57143){4}{\line( 0,-1){  4.286}}
\multiput(355,540)(0.00000,-8.57143){4}{\line( 0,-1){  4.286}}
\multiput(385,540)(0.00000,-8.57143){4}{\line( 0,-1){  4.286}}
\multiput(415,540)(0.00000,-8.57143){4}{\line( 0,-1){  4.286}}
\multiput(445,540)(0.00000,-8.57143){4}{\line( 0,-1){  4.286}}
\multiput(475,540)(0.00000,-8.57143){4}{\line( 0,-1){  4.286}}
\multiput(505,540)(0.00000,-8.57143){4}{\line( 0,-1){  4.286}}
\put( 40,510){\framebox(480,30){}}
\put( 40,540){\dashbox{4}(0,0){}}
\multiput(250,540)(0.00000,-8.57143){4}{\line( 0,-1){  4.286}}
\multiput(220,540)(0.00000,-8.57143){4}{\line( 0,-1){  4.286}}
\multiput(190,540)(0.00000,-8.57143){4}{\line( 0,-1){  4.286}}
\multiput(160,540)(0.00000,-8.57143){4}{\line( 0,-1){  4.286}}
\multiput(130,540)(0.00000,-8.57143){4}{\line( 0,-1){  4.286}}
\multiput(100,540)(0.00000,-8.57143){4}{\line( 0,-1){  4.286}}
\multiput(280,540)(0.00000,-8.57143){4}{\line( 0,-1){  4.286}}
\multiput(310,540)(0.00000,-8.57143){4}{\line( 0,-1){  4.286}}
\multiput(340,540)(0.00000,-8.57143){4}{\line( 0,-1){  4.286}}
\multiput(370,540)(0.00000,-8.57143){4}{\line( 0,-1){  4.286}}
\multiput(400,540)(0.00000,-8.57143){4}{\line( 0,-1){  4.286}}
\multiput(430,540)(0.00000,-8.57143){4}{\line( 0,-1){  4.286}}
\multiput(460,540)(0.00000,-8.57143){4}{\line( 0,-1){  4.286}}
\multiput(490,540)(0.00000,-8.57143){4}{\line( 0,-1){  4.286}}
\multiput( 70,540)(0.00000,-8.57143){4}{\line( 0,-1){  4.286}}
\put( 55,540){\line( 0,-1){ 30}}
\multiput( 85,540)(0.00000,-8.57143){4}{\line( 0,-1){  4.286}}
\multiput(115,540)(0.00000,-8.57143){4}{\line( 0,-1){  4.286}}
\multiput(145,540)(0.00000,-8.57143){4}{\line( 0,-1){  4.286}}
\multiput(205,540)(0.00000,-8.57143){4}{\line( 0,-1){  4.286}}
\multiput(235,540)(0.00000,-8.57143){4}{\line( 0,-1){  4.286}}
\multiput(265,540)(0.00000,-8.57143){4}{\line( 0,-1){  4.286}}
\multiput(295,540)(0.00000,-8.57143){4}{\line( 0,-1){  4.286}}
\multiput(325,540)(0.00000,-8.57143){4}{\line( 0,-1){  4.286}}
\multiput(355,540)(0.00000,-8.57143){4}{\line( 0,-1){  4.286}}
\multiput(385,540)(0.00000,-8.57143){4}{\line( 0,-1){  4.286}}
\multiput(415,540)(0.00000,-8.57143){4}{\line( 0,-1){  4.286}}
\multiput(445,540)(0.00000,-8.57143){4}{\line( 0,-1){  4.286}}
\multiput(475,540)(0.00000,-8.57143){4}{\line( 0,-1){  4.286}}
\multiput(505,540)(0.00000,-8.57143){4}{\line( 0,-1){  4.286}}
\put(175,540){\line( 0,-1){ 30}}
\put(175,540){\line( 0,-1){ 30}}
\multiput(145,540)(0.00000,-8.57143){4}{\line( 0,-1){  4.286}}
\multiput(115,540)(0.00000,-8.57143){4}{\line( 0,-1){  4.286}}
\multiput( 85,540)(0.00000,-8.57143){4}{\line( 0,-1){  4.286}}
\put( 55,540){\line( 0,-1){ 30}}
\multiput( 70,540)(0.00000,-8.57143){4}{\line( 0,-1){  4.286}}
\multiput(100,540)(0.00000,-8.57143){4}{\line( 0,-1){  4.286}}
\multiput(130,540)(0.00000,-8.57143){4}{\line( 0,-1){  4.286}}
\multiput(160,540)(0.00000,-8.57143){4}{\line( 0,-1){  4.286}}
\put(345,470){\makebox(0,0)[lb]{\raisebox{0pt}[0pt][0pt]{\twltt $F$}}}
\put(110,470){\makebox(0,0)[lb]{\raisebox{0pt}[0pt][0pt]{\twltt $E$}}}
\put( 40,470){\makebox(0,0)[lb]{\raisebox{0pt}[0pt][0pt]{\twltt $S$}}}
\put(505,545){\makebox(0,0)[lb]{\raisebox{0pt}[0pt][0pt]{\twltt $0$}}}
\put(160,545){\makebox(0,0)[lb]{\raisebox{0pt}[0pt][0pt]{\twltt $23$}}}
\put( 40,545){\makebox(0,0)[lb]{\raisebox{0pt}[0pt][0pt]{\twltt $31$}}}
\put(140,400){\makebox(0,0)[lb]{\raisebox{0pt}[0pt][0pt]{\twltt $Value=(-1)^{S}(2^{E-127})(1.F)$}}}
\put(505,545){\makebox(0,0)[lb]{\raisebox{0pt}[0pt][0pt]{\twltt $0$}}}
\put(160,545){\makebox(0,0)[lb]{\raisebox{0pt}[0pt][0pt]{\twltt $23$}}}
\put( 40,545){\makebox(0,0)[lb]{\raisebox{0pt}[0pt][0pt]{\twltt $31$}}}
\put(140,400){\makebox(0,0)[lb]{\raisebox{0pt}[0pt][0pt]{\twltt $Value=(-1)^{S}(2^{E-127})(1.F)$}}}
\end{picture}
\end{center}
\caption{IEEE single precision floating point format}
\label{fig:ieee}
\end{figure}

\subsection{Specification of the lookup table}
To calculate the function $1/\sqrt{x}$ using the previously mentioned
iteration scheme, it is clear that the first estimation of the solution must
be accurate enough to get precise results. The requirements for the
calculation are
\begin{itemize}
\item Maximum possible accuracy with the used IEEE format
\item Use only one iteration step for maximum speed
\end{itemize}

The first requirement states that the result of $1/\sqrt{x}$ may have a
relative error $\varepsilon_{r}$ equal to the $\varepsilon_{r}$ of a IEEE 32
bit single precision floating point number. From this the $1/\sqrt{x}$
of the initial approximation can be derived, rewriting the definition of
the relative error for succeeding steps, equation (\eqnref{epsr})
\begin{equation}
\frac{\varepsilon_{n}}{y} =
\sqrt{\varepsilon_ {r_{n+1}} \frac{2f'}{yf''}}
\end{equation}
So for the lookup table the needed accuracy is
\begin{equation}
\frac{\Delta Y}{Y} = \sqrt{\frac{2}{3} 2^{-24}}
\label{eqn:accy}
\end{equation}
which defines the width of the table that must be $\geq 13$ bit.

At this point the relative error $\varepsilon_{r_{n}}$ of the lookup table
is known. From this the maximum relative error in the argument can be 
calculated as follows. The absolute error
$\Delta x$ is defined as
\begin{equation}
\Delta x \equiv \frac{\Delta Y}{Y'}
\end{equation}
and thus
\begin{equation}
\frac{\Delta x}{Y} = \frac{\Delta Y}{Y} (Y')^{-1}
\end{equation}
and thus
\begin{equation}
\Delta x = constant \frac{Y}{Y'}
\end{equation}
for the $1/\sqrt{x}$ function $Y / Y' \sim x$ holds, so
$\Delta x / x = constant$. This is a property of the used floating point
representation as earlier mentioned. The needed accuracy of the argument of the
lookup table follows from
\begin{equation}
\frac{\Delta x}{x} = -2 \frac{\Delta Y}{Y}
\end{equation}
so, using the floating point accuracy, equation (\eqnref{accy})
\begin{equation}
\frac{\Delta x}{x} = -2 \sqrt{\frac{2}{3} 2^{-24}}
\end{equation}
This defines the length of the lookup table which should be $\geq 12$ bit.

\subsection{Separate exponent and fraction computation}
The used IEEE 32 bit single precision floating point format specifies
that a number is represented by a exponent and a fraction. The previous 
section specifies for every possible floating point number the lookup table 
length and width. Only the size of the fraction of a floating point number 
defines the accuracy. The conclusion from this can be that the size of the 
lookup table is length of lookup table, earlier specified, times the size of 
the exponent ($2^{12}2^{8}, 1Mb$). The $1/\sqrt{x}$  function has the 
property that the exponent is independent of the fraction. This becomes clear 
if the floating point representation is used. Define
\begin{equation}
x \equiv (-1)^{S}(2^{E-127})(1.F)
\label{eqn:fpdef}
\end{equation}
see \figref{ieee} where $0 \leq S \leq 1$, $0 \leq E \leq 255$,
$1 \leq 1.F < 2$ and $S$, $E$, $F$ integer (normalization conditions). 
The sign bit ($S$) can be omitted because $1/\sqrt{x}$ is only defined 
for $x > 0$. The $1/\sqrt{x}$ function applied to $x$ results in
\begin{equation}
y(x) = \frac{1}{\sqrt{x}}
\end{equation}
or
\begin{equation}
y(x) = \frac{1}{\sqrt{(2^{E-127})(1.F)}}
\end{equation}
this can be rewritten as
\begin{equation}
y(x) = (2^{E-127})^{-1/2}(1.F)^{-1/2}
\label{eqn:yx}
\end{equation}
Define
\begin{equation}
(2^{E'-127}) \equiv (2^{E-127})^{-1/2}
\end{equation}
\begin{equation}
1.F'\equiv (1.F)^{-1/2}
\end{equation}
then $\frac{1}{\sqrt{2}} < 1.F' \leq 1$ holds, so the condition
$1 \leq 1.F' < 2$ which is essential for normalized real representation is
not valid anymore. By introducing an extra term this can be corrected.
Rewrite the $1/\sqrt{x}$ function applied to floating point numbers, equation
(\eqnref{yx}) as
\begin{equation}
y(x) = (2^{\frac{127-E}{2}-1}) (2(1.F)^{-1/2})
\end{equation}
and
\begin{equation}
(2^{E'-127}) \equiv (2^{\frac{127-E}{2}-1})
\label{eqn:exp}
\end{equation}
\begin{equation}
1.F'\equiv 2(1.F)^{-1/2}
\label{eqn:frac}
\end{equation}
then $\sqrt{2} < 1.F \leq 2$ holds. This is not the exact valid range as
defined for normalized floating point numbers in equation (\eqnref{fpdef}). 
The value $2$  causes the problem. By mapping this value on the nearest
representation $< 2$ this can be solved. The small error that is introduced
by this approximation is within the allowable range. 

The integer representation of the exponent is the next problem. Calculating
$(2^{\frac{127-E}{2}-1})$ introduces a fractional result if $(127-E) = odd$.
This is again easily accounted for by splitting up the calculation into an
odd and an even part. For $(127-E) = even$ $E'$ in equation (\eqnref{exp})
can be exactly calculated in integer arithmetic as a function of $E$.
\begin{equation}
E' = \frac{127-E}{2} + 126
\end{equation}

For $(127-E) = odd$ equation (\eqnref{yx}) can be rewritten as
\begin{equation}
y(x) = (2^{\frac{127-E-1}{2}})(\frac{1.F}{2})^{-1/2}
\end{equation}
thus
\begin{equation}
E' = \frac{126-E}{2} + 127
\end{equation}
which also can be calculated exactly in integer arithmetic.
Note that the fraction is automatically corrected for its range earlier
mentioned, so the exponent does not need an extra correction.

The conclusions from this are:
\begin{itemize}
\item The fraction and exponent lookup table are independent. The fraction
lookup table exists of two tables (odd and even exponent) so the odd/even
information of the exponent (lsb bit) has to be used to select the right
table.
\item The exponent table is an 256 x 8 bit table, initialized for $odd$
and $even$.% as specified before
\end{itemize}

\subsection{Implementation}
The lookup tables can be generated by a small C program, which uses
floating point numbers and operations with IEEE 32 bit single precision format.
Note that because of the $odd$/$even$ information that is needed, the
fraction table is twice the size earlier specified (13 bit i.s.o. 12 bit).
%\figref{expgen} in the appendix shows how to fill the exponent table,
%\figref{fractgen} shows how to fill the fraction table.

The function according to equation (\eqnref{nr}) has to be implemented. 
Applied to the $1/\sqrt{x}$ function, equation 
(\eqnref{inversef}) leads to
\begin{equation}
f = a - \frac{1}{y^{2}}
\end{equation}
and so
\begin{equation}
f' = \frac{2}{y^{3}}
\end{equation}
so
\begin{equation}
y_{n+1} = y_{n} - \frac{ a - \frac{1}{y^{2}_{n}} }{ \frac{2}{y^{3}_{n}} }
\end{equation}
or
\begin{equation}
y_{n+1} = \frac{y_{n}}{2} (3 - a y^{2}_{n})
\end{equation}
Where $y_{0}$ can be found in the lookup tables, and $y_{1}$ gives the result
to the maximum accuracy. 
%In \figref{as_implementation}) the assembler implementation is given. 
It is clear that only one iteration extra (in double 
precision) is needed for a double precision result.

