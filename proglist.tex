\newlength{\proglistwidth}
\newsavebox{\proglistsavebox}
\savebox{\proglistsavebox}{\ttfamily\small g_helixorient}
\settowidth{\proglistwidth}{\usebox{\proglistsavebox}}
\begin{description}[font=\bfseries\large]
\item[Generating topologies and coordinates] \ 
\begin{description}[font=\ttfamily\small, style=nextline, leftmargin=\proglistwidth, noitemsep, labelsep=0pt]
\item[pdb2gmx] converts pdb files to topology and coordinate files 
\item[x2top] generates a primitive topology from coordinates  
\item[editconf] edits the box and writes subgroups  
\item[genbox] solvates a system 
\item[genion] generates mono atomic ions on energetically favorable positions 
\item[genconf] multiplies a conformation in 'random' orientations 
\item[genrestr] generates position restraints or distance restraints for index groups 
\item[protonate] protonates structures 
\end{description}

\item[Running a simulation] \ 
\begin{description}[font=\ttfamily\small, style=nextline, leftmargin=\proglistwidth, noitemsep, labelsep=0pt]
\item[grompp] makes a run input file 
\item[tpbconv] makes a run input file for restarting a crashed run 
\item[mdrun] performs a simulation, do a normal mode analysis or an energy minimization 
\end{description}

\item[Viewing trajectories] \ 
\begin{description}[font=\ttfamily\small, style=nextline, leftmargin=\proglistwidth, noitemsep, labelsep=0pt]
\item[ngmx] displays a trajectory 
\item[highway] X-windows gadget for highway simulations 
\end{description}

\item[Processing energies] \ 
\begin{description}[font=\ttfamily\small, style=nextline, leftmargin=\proglistwidth, noitemsep, labelsep=0pt]
\item[g_energy] writes energies to xvg files and displays averages 
\item[g_enemat] extracts an energy matrix from an energy file 
\item[mdrun] with -rerun (re)calculates energies for trajectory frames 
\end{description}

\item[Converting files] \ 
\begin{description}[font=\ttfamily\small, style=nextline, leftmargin=\proglistwidth, noitemsep, labelsep=0pt]
\item[editconf] converts and manipulates structure files 
\item[trjconv] converts and manipulates trajectory files 
\item[trjcat] concatenates trajectory files 
\item[eneconv] converts energy files 
\item[xmp2ps] converts XPM matrices to encapsulated postscript (or XPM) 
\item[sigeps] convert c6/12 or c6/cn combinations to and from sigma/epsilon 
\end{description}

\item[Tools] \ 
\begin{description}[font=\ttfamily\small, style=nextline, leftmargin=\proglistwidth, noitemsep, labelsep=0pt]
\item[make_ndx] makes index files 
\item[mk_angndx] generates index files for g_angle 
\item[gmxcheck] checks and compares files 
\item[gmxdump] makes binary files human readable 
\item[gen_table] generate tables for use by mdrun 
\item[g_traj] plots x, v and f of selected atoms/groups (and more) from a trajectory 
\item[g_analyze] analyzes data sets 
\item[trjorder] orders molecules according to their distance to a group 
\item[g_filter] frequency filters trajectories, useful for making smooth movies 
\item[g_lie] free energy estimate from linear combinations 
\item[g_dyndom] interpolate and extrapolate structure rotations 
\item[g_morph] linear interpolation of conformations  
\item[g_wham] weighted histogram analysis after umbrella sampling 
\item[ffscan] scan and modify force field data for a single point energy calculation 
\item[xpm2ps] convert XPM (XPixelMap) file to postscript 
\item[g_sham] read/write xmgr and xvgr data sets 
\item[g_spatial] calculates the spatial distribution function (more control than g_sdf) 
\item[g_sdf] calculates the spatial distribution function (faster than g_spatial) 
\end{description}

\item[Distances between structures] \ 
\begin{description}[font=\ttfamily\small, style=nextline, leftmargin=\proglistwidth, noitemsep, labelsep=0pt]
\item[g_rms] calculates rmsd's with a reference structure and rmsd matrices 
\item[g_confrms] fits two structures and calculates the rmsd  
\item[g_cluster] clusters structures 
\item[g_rmsf] calculates atomic fluctuations 
\end{description}

\item[Distances in structures over time] \ 
\begin{description}[font=\ttfamily\small, style=nextline, leftmargin=\proglistwidth, noitemsep, labelsep=0pt]
\item[g_mindist] calculates the minimum distance between two groups 
\item[g_dist] calculates the distances between the centers of mass of two groups 
\item[g_bond] calculates distances between atoms 
\item[g_mdmat] calculates residue contact maps 
\item[g_polystat] calculates static properties of polymers 
\item[g_rmsdist] calculates atom pair distances averaged with power 2, -3 or -6 
\end{description}

\item[Mass distribution properties over time] \ 
\begin{description}[font=\ttfamily\small, style=nextline, leftmargin=\proglistwidth, noitemsep, labelsep=0pt]
\item[g_traj] plots x, v, f, box, temperature and rotational energy 
\item[g_gyrate] calculates the radius of gyration 
\item[g_msd] calculates mean square displacements 
\item[g_polystat] calculates static properties of polymers 
\item[g_rotacf] calculates the rotational correlation function for molecules 
\item[g_rotmat] plots the rotation matrix for fitting to a reference structure 
\item[g_vanhove] calculates Van Hove displacement functions 
\end{description}

\item[Analyzing bonded interactions] \ 
\begin{description}[font=\ttfamily\small, style=nextline, leftmargin=\proglistwidth, noitemsep, labelsep=0pt]
\item[g_bond] calculates bond length distributions 
\item[mk_angndx] generates index files for g_angle 
\item[g_angle] calculates distributions and correlations for angles and dihedrals 
\item[g_dih] analyzes dihedral transitions 
\end{description}

\item[Structural properties] \ 
\begin{description}[font=\ttfamily\small, style=nextline, leftmargin=\proglistwidth, noitemsep, labelsep=0pt]
\item[g_hbond] computes and analyzes hydrogen bonds 
\item[g_saltbr] computes salt bridges 
\item[g_sas] computes solvent accessible surface area 
\item[g_order] computes the order parameter per atom for carbon tails 
\item[g_principal] calculates axes of inertia for a group of atoms 
\item[g_rdf] calculates radial distribution functions 
\item[g_sdf] calculates solvent distribution functions 
\item[g_sgangle] computes the angle and distance between two groups 
\item[g_sorient] analyzes solvent orientation around solutes 
\item[g_spol] analyzes solvent dipole orientation and polarization around solutes 
\item[g_bundle] analyzes bundles of axes, {\eg} helices 
\item[g_disre] analyzes distance restraints 
\item[g_clustsize] calculate size distributions of atomic clusters 
\item[anadock] cluster structures from Autodock runs 
\end{description}

\item[Kinetic properties] \ 
\begin{description}[font=\ttfamily\small, style=nextline, leftmargin=\proglistwidth, noitemsep, labelsep=0pt]
\item[g_traj] plots x, v, f, box, temperature and rotational energy 
\item[g_velacc] calculates velocity autocorrelation functions 
\item[g_tcaf] calculates viscosities of liquids 
\item[g_kinetics] derives information about kinetic processes from you trajectories 
\end{description}

\item[Electrostatic properties] \ 
\begin{description}[font=\ttfamily\small, style=nextline, leftmargin=\proglistwidth, noitemsep, labelsep=0pt]
\item[genion] generates mono atomic ions on energetically favorable positions 
\item[g_potential] calculates the electrostatic potential across the box 
\item[g_dipoles] computes the total dipole plus fluctuations 
\item[g_dielectric] calculates frequency dependent dielectric constants 
\item[g_current] calculates dielectric constants for charged systems 
\end{description}

\item[Protein specific analysis] \ 
\begin{description}[font=\ttfamily\small, style=nextline, leftmargin=\proglistwidth, noitemsep, labelsep=0pt]
\item[do_dssp] assigns secondary structure and calculates solvent accessible surface area 
\item[g_chi] calculates everything you want to know about chi and other dihedrals 
\item[g_helix] calculates basic properties of alpha helices 
\item[g_helixorient] calculates local pitch/bending/rotation/orientation inside helices 
\item[g_rama] computes Ramachandran plots 
\item[xrama] shows animated Ramachandran plots 
\item[wheel] plots helical wheels 
\end{description}

\item[Interfaces] \ 
\begin{description}[font=\ttfamily\small, style=nextline, leftmargin=\proglistwidth, noitemsep, labelsep=0pt]
\item[g_potential] calculates the electrostatic potential across the box 
\item[g_density] calculates the density of the system 
\item[g_densmap] calculates 2D planar or axial-radial density maps 
\item[g_order] computes the order parameter per atom for carbon tails 
\item[g_h2order] computes the orientation of water molecules 
\item[g_bundle] analyzes bundles of axes, {\eg} transmembrane helices 
\end{description}

\item[Covariance analysis] \ 
\begin{description}[font=\ttfamily\small, style=nextline, leftmargin=\proglistwidth, noitemsep, labelsep=0pt]
\item[g_covar] calculates and diagonalizes the covariance matrix 
\item[g_anaeig] analyzes the eigenvectors 
\item[make_edi] generate input files for essential dynamics sampling 
\end{description}

\item[Normal modes] \ 
\begin{description}[font=\ttfamily\small, style=nextline, leftmargin=\proglistwidth, noitemsep, labelsep=0pt]
\item[grompp] makes a run input file 
\item[mdrun] finds a potential energy minimum 
\item[mdrun] calculates the Hessian 
\item[g_nmeig] diagonalizes the Hessian  
\item[g_nmtraj] generate oscillating trajectory of an eigenmode 
\item[g_anaeig] analyzes the normal modes 
\item[g_nmens] generates an ensemble of structures from the normal modes 
\end{description}

\end{description}

