\begin{description}
\item {\large\bf Generating topologies and coordinates}
\vspace{-2ex}\begin{tabbing}
{\bf mk\_angndx} \= \kill
{\bf pdb2gmx} \> converts pdb files to topology and coordinate files  \\
{\bf editconf} \> edits the box and writes subgroups  \\
{\bf genbox} \> solvates a system \\
{\bf genion} \> generates mono atomic ions on energetically favorable positions \\
{\bf genconf} \> multiplies a conformation in 'random' orientations \\
{\bf genpr} \> generates position restraints for index groups \\
{\bf protonate} \> protonates structures \\
\end{tabbing}\vspace{-2ex}

\item {\large\bf Running a simulation}
\vspace{-2ex}\begin{tabbing}
{\bf mk\_angndx} \= \kill
{\bf grompp} \> makes a run input file \\
{\bf tpbconv} \> makes a run input file for restarting a crashed run \\
{\bf mdrun} \> performs a simulation \\
\end{tabbing}\vspace{-2ex}

\item {\large\bf Viewing trajectories}
\vspace{-2ex}\begin{tabbing}
{\bf mk\_angndx} \= \kill
{\bf ngmx} \> displays a trajectory \\
{\bf trjconv} \> converts trajectories to {\eg} pdb which can be viewed with {\eg} rasmol \\
\end{tabbing}\vspace{-2ex}

\item {\large\bf Processing energies}
\vspace{-2ex}\begin{tabbing}
{\bf mk\_angndx} \= \kill
{\bf g\_energy} \> writes energies to xvg files and displays averages \\
{\bf g\_enemat} \> extracts an energy matrix from an energy file \\
{\bf mdrun} \> with -rerun (re)calculates energies for trajectory frames \\
\end{tabbing}\vspace{-2ex}

\item {\large\bf Converting files}
\vspace{-2ex}\begin{tabbing}
{\bf mk\_angndx} \= \kill
{\bf editconf} \> converts and manipulates structure files \\
{\bf trjconv} \> converts and manipulates trajectory files \\
{\bf trjcat} \> concatenates trajectory files \\
{\bf eneconv} \> converts energy files \\
{\bf xmp2ps} \> converts XPM matrices to encapsulated postscript (or XPM) \\
\end{tabbing}\vspace{-2ex}

\item {\large\bf Tools}
\vspace{-2ex}\begin{tabbing}
{\bf mk\_angndx} \= \kill
{\bf make\_ndx} \> makes index files \\
{\bf mk\_angndx} \> generates index files for g\_angle \\
{\bf gmxcheck} \> checks and compares files \\
{\bf gmxdump} \> makes binary files human readable \\
{\bf g\_analyze} \> analyzes data sets \\
\end{tabbing}\vspace{-2ex}

\item {\large\bf Distances between structures}
\vspace{-2ex}\begin{tabbing}
{\bf mk\_angndx} \= \kill
{\bf g\_rms} \> calculates rmsd's with a reference structure and rmsd matrices \\
{\bf g\_confrms} \> fits two structures and calculates the rmsd  \\
{\bf g\_cluster} \> clusters structures \\
{\bf g\_rmsf} \> calculates atomic fluctuations \\
\end{tabbing}\vspace{-2ex}

\item {\large\bf Distances in structures over time}
\vspace{-2ex}\begin{tabbing}
{\bf mk\_angndx} \= \kill
{\bf g\_mindist} \> calculates the minimum distance between two groups \\
{\bf g\_dist} \> calculates the distances between the centers of mass of two groups \\
{\bf g\_mdmat} \> calculates residue contact maps \\
{\bf g\_rmsdist} \> calculates atom pair distances averaged with power 2, -3 or -6 \\
\end{tabbing}\vspace{-2ex}

\item {\large\bf Mass distribution properties over time}
\vspace{-2ex}\begin{tabbing}
{\bf mk\_angndx} \= \kill
{\bf g\_com} \> calculates the center of mass \\
{\bf g\_gyrate} \> calculates the radius of gyration \\
{\bf g\_msd} \> calculates mean square displacements \\
{\bf g\_rotacf} \> calculates the rotational correlation function for molecules \\
{\bf g\_rdf} \> calculates RDF's \\
{\bf g\_rdens} \> calculates radial densities \\
\end{tabbing}\vspace{-2ex}

\item {\large\bf Analyzing bonded interactions}
\vspace{-2ex}\begin{tabbing}
{\bf mk\_angndx} \= \kill
{\bf g\_bond} \> calculates bond length distributions \\
{\bf mk\_angndx} \> generates index files for g\_angle \\
{\bf g\_angle} \> calculates distributions and correlations for angles and dihedrals \\
{\bf g\_dih} \> analyzes dihedral transitions \\
\end{tabbing}\vspace{-2ex}

\item {\large\bf Structural properties}
\vspace{-2ex}\begin{tabbing}
{\bf mk\_angndx} \= \kill
{\bf g\_hbond} \> computes and analyzes hydrogen bonds \\
{\bf g\_saltbr} \> computes salt bridges \\
{\bf g\_sas} \> computes solvent accessible surface area \\
{\bf g\_order} \> computes the order parameter per atom for carbon tails \\
{\bf g\_sgangle} \> computes the angle and distance between two groups \\
{\bf g\_disre} \> analyzes distance restraints \\
\end{tabbing}\vspace{-2ex}

\item {\large\bf Kinetic properties}
\vspace{-2ex}\begin{tabbing}
{\bf mk\_angndx} \= \kill
{\bf g\_velacc} \> calculates velocity autocorrelation functions \\
\end{tabbing}\vspace{-2ex}

\item {\large\bf Electrostatic properties}
\vspace{-2ex}\begin{tabbing}
{\bf mk\_angndx} \= \kill
{\bf genion} \> generates mono atomic ions on energetically favorable positions \\
{\bf g\_potential} \> calculates the electrostatic potential across the box \\
{\bf g\_dipoles} \> computes the total dipole plus fluctuations \\
{\bf g\_dielectric} \> calculates frequency dependent dielectric constants \\
\end{tabbing}\vspace{-2ex}

\item {\large\bf Protein specific analysis}
\vspace{-2ex}\begin{tabbing}
{\bf mk\_angndx} \= \kill
{\bf do\_dssp} \> assigns secondary structure and calculates solvent accessible surface area \\
{\bf g\_chi} \> calculates everything you want to know about chi and other dihedrals \\
{\bf g\_helix} \> calculates everything you want to know about helices \\
{\bf g\_rama} \> computes Ramachandran plots \\
{\bf xrama} \> shows animated Ramachandran plots \\
{\bf wheel} \> plots helical wheels \\
\end{tabbing}\vspace{-2ex}

\item {\large\bf Interfaces}
\vspace{-2ex}\begin{tabbing}
{\bf mk\_angndx} \= \kill
{\bf g\_potential} \> calculates the electrostatic potential across the box \\
{\bf g\_density} \> calculates the density of the system \\
{\bf g\_order} \> computes the order parameter per atom for carbon tails \\
{\bf g\_h2order} \> computes the orientation of water molecules \\
\end{tabbing}\vspace{-2ex}

\item {\large\bf Covariance analysis}
\vspace{-2ex}\begin{tabbing}
{\bf mk\_angndx} \= \kill
{\bf g\_covar} \> calculates and diagonalizes the covariance matrix \\
{\bf g\_anaeig} \> analyzes the eigenvectors \\
\end{tabbing}\vspace{-2ex}

\item {\large\bf Normal modes}
\vspace{-2ex}\begin{tabbing}
{\bf mk\_angndx} \= \kill
{\bf grompp} \> makes a run input file \\
{\bf mdrun} \> finds a potential energy minimum \\
{\bf nmrun} \> calculates the Hessian \\
{\bf g\_nmeig} \> diagonalizes the Hessian  \\
{\bf g\_anaeig} \> analyzes the normal modes \\
{\bf g\_nmens} \> generates an ensemble of structures from the normal modes \\
\end{tabbing}\vspace{-2ex}

\end{description}
