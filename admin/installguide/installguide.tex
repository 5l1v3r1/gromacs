% Process from LaTeX via XML to XHTML with
% latexml --destination installguide.xml --xml installguide.tex
% latexmlpost --destination installguide.xhtml --format=xhtml installguide.xml
%
% Crude hack to remove ugly symbols:
% sed -e 's/§//g' -i installguide.xhtml
%
% Strip off header for pasting into the website at
% http://www.gromacs.org/Documentation/Installation_Instructions:
%
% grep -A 99999 "class=\"main\"" installguide.xhtml > installguide_web.xhtml

\documentclass{article}[12pt,a4paper,twoside]
\usepackage{hyperref}
% haven't made these work with LaTeXML yet...
%\usepackage[strings]{underscore}
%\usepackage[english]{babel}

\title{GROMACS installation guide}

% macros to keep style uniform
\newcommand{\gromacs}{GROMACS}
\newcommand{\nvidia}{NVIDIA}
\newcommand{\cuda}{CUDA}
\newcommand{\fftw}{FFTW}
\newcommand{\mkl}{MKL}
\newcommand{\mpi}{MPI}
\newcommand{\threadmpi}{ThreadMPI}
\newcommand{\openmpi}{OpenMPI}
\newcommand{\openmp}{OpenMP}
\newcommand{\openmm}{OpenMM}
\newcommand{\lammpi}{LAM/MPI}
\newcommand{\mpich}{MPICH}
\newcommand{\cmake}{CMake}
\newcommand{\sse}{SSE}
\newcommand{\ssetwo}{SSE2}
\newcommand{\avx}{AVX}
\newcommand{\fft}{FFT}
\newcommand{\blas}{BLAS}
\newcommand{\lapack}{LAPACK}
\newcommand{\vmd}{VMD}
\newcommand{\pymol}{PyMOL}
\newcommand{\grace}{Grace}
%\newcommand{\}{}
%\newcommand{\}{}

% later, make CMake keep this version current for us
\newcommand{\fftwversion}{3.3.2}
\newcommand{\cmakeversion}{2.8.0}
\newcommand{\cudaversion}{3.2}

\begin{document}
\section{Building GROMACS}

These instructions pertain to building \gromacs{} 4.6 and newer releases
using our new CMake-based build system. 
For installations instructions for old \gromacs{} versions,
see the documentation at
\url{http://www.gromacs.org/Documentation/Installation_Instructions_4.5}.

\section{Quick and dirty installation}

\begin{enumerate}
\item Get the latest version of your compiler.
\item Check you have \cmake{} version 2.8.x or later.
\item Unpack the \gromacs{} tarball.
\item Make a separate build directory and change to it. 
\item Run \cmake{} with the path to the source as an argument
\item Run make and make install
\end{enumerate}
Or, as a sequence of commands to execute:
\begin{verbatim}
tar xfz gromacs-4.6.1.tar.gz
cd gromacs-4.6.1
mkdir build
cd build
cmake .. -DGMX_BUILD_OWN_FFTW=ON
make
sudo make install
\end{verbatim}
This will download and build first the prerequisite FFT library followed by \gromacs{}. If you already have
FFTW installed, you can remove that argument to cmake. Overall, this build 
of \gromacs{} will be correct and reasonably fast on the
machine upon which \cmake{} ran. It will generally be 30-50\% faster
than \gromacs{} 4.5.x, but if you want to get the maximum value
for your hardware with \gromacs{}, you'll have to read further.
Sadly, the interactions of hardware, libraries, and compilers
are only going to continue to get more complex. 

\section{Prerequisites}
\subsection{Platform}
\gromacs{} can be compiled for any distribution of Linux, Mac OS X,
Windows (native, Cygwin or MinGW), BlueGene, Cray and many other architectures.
Technically, it can be compiled on any platform with an ANSI C
compiler and supporting libraries, such as the GNU C library. However, Gromacs
also comes with many hardware-specific extensions to provide very high performance
on those platforms, and to enable these we have slightly more specific requirements
since old compilers do not support new features, or they can be buggy.

\subsection{Compiler}

\gromacs{} requires an ANSI C compiler that complies with the C89
standard. For best performance, the \gromacs{} team strongly
recommends you get the most recent version of your preferred compiler
for your platform (e.g. GCC 4.7 or Intel 12.0 or newer on x86
hardware). There is a large amount of \gromacs{} code introduced in
version 4.6 that depends on effective compiler optimization to get
high performance - the old raw assembly-language kernel routines are all gone.
Unfortunately this makes \gromacs{} more sensitive to the compiler
used, and the binary will only work on the hardware for which it is compiled,
but the good news is that it has enabled us to significantly accelerate performance
compared to version 4.5. 

\begin{itemize}
\item On Intel-based x86 hardware, we recommend you to use
the Intel compiler for best performance. It is usually better at instruction
scheduling, although it does not hurt to try gcc too. Recent versions can
give icc a run for the money.
\item On AMD-based x86 hardware up through the Magny-Cours architecture
(e.g. Opteron 6100-series processors), it is worth using the Intel compiler for
better performance, but gcc-4.7 and later are also reasonable.
\item On the AMD Bulldozer architecture (Opteron 6200), AMD introduced fused multiply-add
instructions and an "FMA4" instruction format not available on Intel x86 processors. Thus,
on the most recent AMD processors you want to use gcc-4.7 or later for better performance!
icc will only generate code for the subset also supported by Intel processors, and that
is significantly slower right now.
\item If you are running on Mac OS X, the best option is the Intel compiler.
Both clang and gcc will work, but they produce lower performance and each have some
shortcomings. Clang does not fully support OpenMP, and the current gcc ports do not
support AVX instructions. 
\item For all non-x86 platforms, your best option is typically to use the vendor's 
default compiler, and check for specialized information below.
\end{itemize}

\subsubsection{Running in parallel}

\gromacs{} can run in parallel on multiple cores of a single
workstation using its built-in \threadmpi. No user action is required
in order to enable this.

If you wish to use the excellent new native GPU support in \gromacs,
\nvidia{}'s \cuda{}
\url{http://www.nvidia.com/object/cuda_home_new.html} version
\cudaversion{} software development kit is required, and the latest
version is strongly encouraged. \nvidia{} GPUs with at least \nvidia{} compute
capability 2.0 are required, e.g. Fermi or Kepler cards.

The GPU support from \gromacs{} version 4.5 using \openmm{}
\url{https://simtk.org/home/openmm} is still contained in the code,
but in the ``user contributions'' section (\verb+src/contrib+). You
will need to edit \verb+src/contrib/CMakeLists.txt+ to enable it. It
also requires \cuda{}, and remains the only hardware-based
acceleration available for implicit solvent simulations in
\gromacs{} at the moment. However, the long-term plan is to enable 
this functionality in core Gromacs, and not have the OpenMM
interface supported by the \gromacs team. Right now there are
some build issues for OpenMM.

If you wish to run in parallel on multiple machines across a network,
you will need to have
\begin{itemize}
\item an \mpi{} library installed that supports the \mpi{} 1.3
  standard, and
\item wrapper compilers that will compile code using that library.
\end{itemize}
The \gromacs{} team recommends \openmpi{}
\url{http://www.open-mpi.org/} version 1.4.1 (or higher), \mpich{}
\url{http://www.mpich.org/} version 1.4.1 (or higher), or your
hardware vendor's \mpi{} installation. The most recent version of
either of this is likely to be the best. More specialized networks
might depend on accelerations only available in the vendor's library.
 \lammpi{}
\url{http://www.lam-mpi.org/} might work, but since it has been
deprecated for years, it is not supported.

In some cases, \openmp{} parallelism is an advantage for \gromacs{},
but support for this is generally built into your compiler and detected
automatically. The one common exception is Mac OS X, where the default
clang compiler currently does not fully support OpenMP. You can install
gcc-4.7 instead, but the currently available binary distribution of gcc 
uses an old system assembler that does not support AVX acceleration
instructions. There are some examples on the internet where people have
hacked this to work, but presently the only straightforward way to get
both OpenMP and AVX support on Mac OS X is to get the Intel compiler.

In summary, for maximum performance you will need to 
examine how you will use \gromacs{}, what hardware you plan to run
on, and whether you can afford a non-free compiler for slightly better
performance. The only way to find out is unfortunately to test different
options and parallelization schemes for the actual simulations you
want to run. You will still get {\em good}\, performance with the default
build and runtime options (better than in version 4.5), but if you truly
want to push your hardware to the performance limit the days of just blindly 
starting programs like '\verb+mdrun+' are gone. 

\subsection{CMake}

From version 4.6, \gromacs{} uses the build system
\cmake{}. The previous build system that used \verb+configure+ from
the GNU autotools package has been removed permanently. \cmake{}
permits the \gromacs{} team to support a very wide range of hardware,
compilers and build configurations while continuing to provide the
portability, robustness and performance for which \gromacs{} is known.

\gromacs{} requires \cmake{} version \cmakeversion{} or higher. Lower
versions will not work. You can check whether \cmake{} is installed,
and what version it is, with \verb+cmake --version+. If you need to
install \cmake{}, then first check whether your platform's package
management system provides a suitable version, or visit
\url{http://www.cmake.org/cmake/help/install.html} for pre-compiled
binaries, source code and installation instructions. The \gromacs{}
team recommends you install the most recent version of \cmake{} you
can. If you need to compile \cmake{} yourself and have a really old environment,
you might first have to compile a moderately recent version (say, 2.6) to
bootstrap version 2.8. This is a one-time job, and you can find lots of
documentation on the \cmake{} website if you run into problems.

\subsection{Fast Fourier Transform library}

Many simulations in \gromacs{} make extensive use of fast Fourier transforms,
and a software library to perform these is always required. We
recommend \fftw{} \url{http://www.fftw.org/} (version 3 or higher
only) or Intel's \mkl{} \url{http://software.intel.com/en-us/intel-mkl}. 

\subsubsection{\fftw{}}

\fftw{} is likely to be available for your platform via its package
management system, but there can be compatibility and significant
performance issues associated with these packages. In particular,
\gromacs{} simulations are normally run in single floating-point
precision whereas the default \fftw{} package is normally in double
precision, and good compiler options to use for \fftw{} when linked to
\gromacs{} may not have been used. Accordingly, the \gromacs{} team
recommends either
\begin{itemize}
\item that you permit the \gromacs{} installation to download and
  build \fftw{} \fftwversion{} from source automatically for you (use
  \verb+cmake -DGMX_BUILD_OWN_FFTW=ON+), or
\item that you build \fftw{} from the source code.
\end{itemize}

If you build \fftw{} from source yourself, get the most recent version
and follow its installation guide available from \url{http://www.fftw.org}.
Choose the precision (i.e. single or float vs.\ double) to match what you will
later require for \gromacs{}. There is no need to compile with
threading or \mpi{} support, but it does no harm. On x86 hardware,
compile \emph{only} with \verb+--enable-sse2+ (regardless of
precision) even if your processors can take advantage of \avx{}
extensions. Since \gromacs{} uses fairly short transform lengths we
do not benefit from the \fftw{} \avx{} acceleration, and because of
memory system performance limitations, it can even degrade \gromacs{}
performance by around 20\%. There is no way for \gromacs{} to
limit the use to \ssetwo{} acceleration at run time if \avx{}
support has been compiled into \fftw{}, so you need to set this at compile time.

\subsubsection{\mkl{}}

Using \mkl{} requires a set of linker flags that \gromacs{} is not
able to detect for you, so setting up optimal linking is tricky at the
moment. You will need to consult your compiler documentation and
use \verb+CMAKE_C_FLAGS+ and \verb+-DCMAKE_EXE_LINKER_FLAGS+
accordingly.

\subsection{Optional build components}

\begin{itemize}
\item Hardware-optimized \blas{} and \lapack{} libraries are useful for
  a few of the \gromacs{} utilities focused on normal modes and matrix manipulation, 
  but they does not provide any benefits for normal simulations.
\item The built-in \gromacs{} trajectory viewer \verb+ngmx+ requires
  X11 and Motif/Lesstif libraries and header files. Generally, the
  \gromacs{} team rather recommends you use third-party software for
  visualization, such as \vmd{}
  \url{http://www.ks.uiuc.edu/Research/vmd/} or \pymol{}
  \url{http://www.pymol.org/}.
\item A few \gromacs{} tools get some extra functionality when linked with the
GNU scientific library GSL.
\end{itemize}

\section{Doing a build of \gromacs}

This section will cover a general build of \gromacs{} with \cmake{},
but it is not an exhaustive discussion of how to use \cmake{}. There
are many resources available on the web, which we suggest you search
for when you encounter problems not covered here. The material below
applies specifically to builds on Unix-like systems, including Linux,
Mac OS X, MinGW and Cygwin. For other platforms, see the specialist
instructions below.

\subsection{Configuring with \cmake{}}

\cmake{} will run many tests on your system and do its best to work
out how to build \gromacs{} for you. If you are building \gromacs{} on
hardware that is identical to that where you will run \verb+mdrun+,
then you can be sure that the defaults will be pretty good. The build
configuration will for instance attempt to detect the specific hardware
instructions available in your processor. However, if
you want to control aspects of the build, there are plenty of things you
can set manually.

The best way to use \cmake{} to configure \gromacs{} is to do an
``out-of-source'' build, by making another directory from which you
will run \cmake{}. This can be a subdirectory or not, it doesn't
matter. It also means you can never corrupt your source code by trying
to build it! So, the only required argument on the \cmake{} command
line is the name of the directory containing the
\verb+CMakeLists.txt+ file of the code you want to build. For
example, download the source tarball and use
% TODO: keep up to date with new releases!
\begin{verbatim}
$ tar xfz gromacs-4.6.1.tgz
$ cd gromacs-4.6.1
$ mkdir build-cmake
$ cd build-cmake
$ cmake ..
\end{verbatim}

You will see \verb+cmake+ report the results of a large number of
tests on your system made by \cmake{} and by \gromacs{}. These are
written to the \cmake{} cache, kept in \verb+CMakeCache.txt+. You
can edit this file by hand, but this is not recommended because it is
easy to reach an inconsistent state. You should not attempt to move or
copy this file to do another build, because file paths are hard-coded
within it. If you mess things up, just delete this file and start
again with '\verb+cmake+'.

If there's a serious problem detected at this stage, then you will see
a fatal error and some suggestions for how to overcome it. If you're
not sure how to deal with that, please start by searching on the web
(most computer problems already have known solutions!) and then
consult the gmx-users mailing list. There are also informational
warnings that you might like to take on board or not. Piping the
output of \verb+cmake+ through \verb+less+ or \verb+tee+ can be
useful, too.

\cmake{} works in an iterative fashion, re-running each time a setting
is changed to try to make sure other things are consistent. Once
things seem consistent, the iterations stop. Once \verb+cmake+
returns, you can see all the settings that were chosen and information
about them by using e.g. the curses interface
\begin{verbatim}
$ ccmake ..
\end{verbatim}
You can actually use \verb+ccmake+ directly in the first step, but then
most of the status messages will merely blink in the lower part
of the terminal rather than be written to standard out. Some platforms
like Windows or Mac even have native graphical user interfaces for
\cmake{}, and it can create project files for almost any build environment
you want (including Visual Studio or Xcode).
Check out \url{http://www.cmake.org/cmake/help/runningcmake.html} for
general advice on what you are seeing and how to navigate and change
things. The settings you might normally want to change are already
presented. If you make any changes, then \verb+ccmake+ will notice
that and require that you re-configure (using '\verb+c+'), so that it
gets a chance to make changes that depend on yours and perform more
checking. This might require several configuration stages when you are
using \verb+ccmake+ - when you are using \verb+cmake+ the
iteration is done behind the scenes.

A key thing to consider here is the setting of
\verb+CMAKE_INSTALL_PREFIX+. You will need to be able to write to
this directory in order to install \gromacs{} later, and if you change
your mind later, changing it in the cache triggers a full re-build,
unfortunately. So if you do not have super-user privileges on your
machine, then you will need to choose a sensible location within your
home directory for your \gromacs{} installation.

When \verb+cmake+ or \verb+ccmake+ have completed iterating, the
cache is stable and a build tree can be generated, with '\verb+g+' in
\verb+ccmake+ or automatically with \verb+cmake+.

You should not attempt to change compilers after the initial run of
\cmake{}. If you need to change, clean up and start again.

\subsection{Using CMake command-line options}
Once you become comfortable with setting and changing options, you
may know in advance how you will configure GROMACS. If so, you can
speed things up by invoking \verb+cmake+ with a command like:
\begin{verbatim}
$ cmake .. -DGMX_GPU=ON -DGMX_MPI=ON -DCMAKE_INSTALL_PREFIX=/home/marydoe/programs
\end{verbatim}
to build with GPUs, MPI and install in a custom location. You can even
save that in a shell script to make it even easier next time. You can
also do this kind of thing with \verb+ccmake+, but you should avoid
this, because the options set with '\verb+-D+' will not be able to be
changed interactively in that run of \verb+ccmake+.

\subsection{CMake advanced options}
The options that can be seen with \verb+ccmake+ are ones that we
think a reasonable number of users might want to consider
changing. There are a lot more options available, which you can see by
toggling the advanced mode in \verb+ccmake+ on and off with
'\verb+t+'. Even there, most of the variables that you might want to
change have a '\verb+CMAKE_+' or '\verb+GMX_+' prefix.

\subsection{Helping CMake find the right libraries/headers/programs}

If libraries are installed in non-default locations their location can
be specified using the following environment variables:
\begin{itemize}
\item \verb+CMAKE_INCLUDE_PATH+ for header files
\item \verb+CMAKE_LIBRARY_PATH+ for libraries
\item \verb+CMAKE_PREFIX_PATH+ for header, libraries and binaries
  (e.g. '\verb+/usr/local+').
\end{itemize}
The respective '\verb+include+', '\verb+lib+', or '\verb+bin+' is
appended to the path. For each of these variables, a list of paths can
be specified (on Unix seperated with ":"). Note that these are
enviroment variables (and not \cmake{} command-line arguments) and in
a '\verb+bash+' shell are used like:
\begin{verbatim}
$ CMAKE_PREFIX_PATH=/opt/fftw:/opt/cuda cmake ..
\end{verbatim}

The \verb+CC+ and \verb+CXX+ environment variables are also useful
for indicating to \cmake{} which compilers to use, which can be very
important for maximising \gromacs{} performance. Similarly,
\verb+CFLAGS+/\verb+CXXFLAGS+ can be used to pass compiler
options, but note that these will be appended to those set by
\gromacs{} for your build platform and build type. You can customize
some of this with advanced options such as \verb+CMAKE_C_FLAGS+
and its relatives.

See also: \url{http://cmake.org/Wiki/CMake_Useful_Variables#Environment_Variables}

\subsection{Native GPU acceleration}
If you have the \cuda{} Software Development Kit installed, you can
use \cmake{} with:
\begin{verbatim}
cmake .. -DGMX_GPU=ON -DCUDA_TOOLKIT_ROOT_DIR=/usr/local/cuda
\end{verbatim}
(or whichever path has your installation). Note that this will require
a working C++ compiler, and in some cases you might need to handle
this manually, e.g. with the advanced option
\verb+CUDA_HOST_COMPILER+.

Historically, Linux GPU builds have received most testing, but we 
want to support GPU builds both under x86 Linux, Windows, Mac OS X and in the
future ARM. Any feedback on this build process (and fixes in particular) are very
welcome!

\subsection{Static linking}
Dynamic linking of the \gromacs{} executables will lead to a
smaller disk footprint when installed, and so is the default on
platforms where we believe it has been tested repeatedly and found to work.
In general, this includes Linux, Windows, Mac OS X and BSD systems.
Static binaries take much more space, but on some hardware and/or under
some conditions they are necessary, most commonly when you are running a parallel
simulation using MPI libraries. 

\begin{itemize}
\item To link \gromacs{} binaries
statically against the internal \gromacs{} libraries, set
\verb+BUILD_SHARED_LIBS=OFF+.
\item To link statically against external
libraries as well, the \verb+GMX_PREFER_STATIC_LIBS=ON+ option can be
used. Note, that in general \cmake{} picks up whatever is available,
so this option only instructs \cmake{} to prefer static libraries when
both static and shared are available. If no static version of an
external library is available, even when the aforementioned option is
ON, the shared library will be used. Also note, that the resulting
binaries will still be dynamically linked against system libraries if
that is all that is available (common on Mac OS X).
\end{itemize}

\subsection{Changing the names of GROMACS binaries and libraries}
It is sometimes convenient to have different versions of the same
\gromacs{} libraries installed. The most common use cases have been
single and double precision, and with and without \mpi{}. By default,
\gromacs{} will suffix binaries and libraries for such builds with
'\verb+_d+' for double precision and/or '\verb+_mpi+' for \mpi{} (and
nothing otherwise). This can be controlled manually with
\verb+GMX_DEFAULT_SUFFIX (ON/OFF)+, \verb+GMX_BINARY_SUFFIX+ (takes 
a string) and \verb+GMX_LIBS_SUFFIX+ (also takes a string). 
This can also be useful for resolving libary-naming conflicts with 
existing packges (\verb+GMX_PREFIX_LIBMD+ also can be useful).
For instance, to set a custom suffix for binaries and libraries, 
one might specify:

\begin{verbatim}
cmake .. -DGMX_DEFAULT_SUFFIX=OFF -DGMX_BINARY_SUFFIX=_mod -DGMX_LIBS_SUFFIX=_mod
\end{verbatim}

Thus the names of all binaries and libraries will be appended with
"_mod."

\subsection{Building \gromacs{}}

Once you have a stable cache, you can build \gromacs{}. If you're not
sure the cache is stable, you can re-run \verb+cmake ..+ or
\verb+ccmake ..+' to see. Then you can run \verb+make+ to start the
compilation. Before actual compilation starts, \verb+make+ checks
that the cache is stable, so if it isn't you will see \cmake{} run
again.

So long as any changes you've made to the configuration are sensible,
it is expected that the \verb+make+ procedure will always complete
successfully. The tests \gromacs{} makes on the settings you choose
are pretty extensive, but there are probably a few cases we haven't
thought of yet. Search the web first for solutions to problems, but if
you need help, ask on gmx-users, being sure to provide as much
information as possible about what you did, the system you are
building on, and what went wrong.

If you have a multi-core or multi-CPU machine with \verb+N+
processors, then using
\begin{verbatim}
$ make -j N
\end{verbatim}
will generally speed things up by quite a bit.

\subsection{Installing \gromacs{}}

Finally, \verb+make install+ will install \gromacs{} in the
directory given in \verb+GMX_INSTALL_PREFIX+. If this is an system
directory, then you will need permission to write there, and you
should use super-user privileges only for \verb+make install+ and
not the whole procedure.

\subsection{Getting access to \gromacs{} after installation}

\gromacs{} installs the script \verb+GMXRC+ in the \verb+bin+
subdirectory of the installation directory
(e.g. \verb+/usr/local/gromacs/bin/GMXRC+), which you should source
from your shell:
\begin{verbatim}
$ source your-installation-prefix-here/bin/GMXRC
\end{verbatim}

It will detect what kind of shell you are running and set up your
environment for using \gromacs{}. You may wish to arrange for your
login scripts to do this automatically; please search the web for
instructions on how to do this for your shell. 

Many of the \gromacs{} programs rely on data installed in our
\verb+share/gromacs+ directory. By default, the programs will use
the environment variables set in the GMXRC script, and if this is not
available they will try to guess the path based on their own location.
This usually works well unless you change the names of directories
inside the install tree. If you still need to do that, you might want to recompile
with the new install location properly set, or edit the \verb+GMXRC+ script.

\subsection{Testing \gromacs{} for correctness}
Since 2011, the \gromacs{} development uses an automated system where
every new patch is subject to regression testing. While this improves
reliability quite a lot, not everything is tested, and since we
increasingly rely on cutting edge compiler features there is
non-negligible risk that the default compiler on your system could
have bugs. We have tried our best to test and refuse to use known bad
versions in \cmake{}, but we strongly recommend that you run through
the regression tests yourself. It only takes a few minutes, after
which you can trust your build.

The simplest way to run the checks is to build \gromacs{} with
\verb+-DREGRESSIONTEST_DOWNLOAD+, and run \verb+make check+.
\gromacs{} will automatically download and run the tests for you.
Alternatively, you can download and unpack the tarball yourself from
\url{http://gerrit.gromacs.org/download/regressiontests-4.6.1.tar.gz},
and use the advanced \cmake{} option \verb+REGRESSIONTEST_PATH+ to
specify the path to the unpacked tarball, which will then be used for
testing. If this doesn't work, then please read on.

The regression tests are available from the \gromacs{} website and ftp
site.  Once you have downloaded them, unpack the tarball, source
\verb+GMXRC+ as described above, and run \verb+./gmxtest.pl all+
inside the regression tests folder. You can find more options
(e.g. adding \verb+double+ when using double precision) if you just
execute the script without options.

Hopefully you will get a report that all tests have passed. If there
are individual failed tests it could be a sign of a compiler bug, or
that a tolerance is just a tiny bit too tight. Check the output files
the script directs you too, and try a different or newer compiler if
the errors appear to be real. If you cannot get it to pass the
regression tests, you might try dropping a line to the gmx-users
mailing list, but then you should include a detailed description of
your hardware and an example logfile from mdrun (which contains
valuable information in the header).

\subsection{Testing \gromacs{} for performance}
We are still working on a set of benchmark systems for testing
the performance of \gromacs{}. Until that is ready, we recommend that
you start by comparing the performance to release 4.5, and also try
a few different parallelization options.

\subsection{Having difficulty?}
You're not alone - this can be a complex task! If you encounter a
problem with installing \gromacs{}, then there are a number of
locations where you can find assistance. It is recommended that you
follow these steps to find the solution:

\begin{enumerate}
\item Read the installation instructions again, taking note that you
  have followed each and every step correctly.
\item Search the \gromacs{} website and users emailing list for
  information on the error.
\item Search the internet using a search engine such as Google.
\item Post to the \gromacs{} users emailing list gmx-users for
  assistance. Be sure to give a full description of what you have done
  and why you think it didn't work. Give details about the system on
  which you are installing. 
  Copy and paste your command line and as
  much of the output as you think might be relevant - certainly from
  the first indication of a problem. In particular, please try to include at
  least the header from the mdrun logfile, and preferably the entire file.
  People who might volunteer to
  help you do not have time to ask you interactive detailed follow-up
  questions, so you will get an answer faster if you provide as much
  information as you think could possibly help. High quality bug reports 
  tend to receive rapid high quality answers.
\end{enumerate}

\section{Special instructions for some platforms}

\subsection{Building on Windows}
Building on Cygwin/MinGW/etc. works just like Unix. Please see the
instructions above.

Building on Windows using native compilers is rather similar to
building on Unix, so please start by reading the above. Then, download
and unpack the GROMACS source archive. The UNIX-standard .tar.gz
format can be managed on Windows, but you may prefer to browse
\url{ftp://ftp.gromacs.org/pub/gromacs} to obtain a zip format file,
which doesn't need any external tools to unzip on recent Windows
systems. Make a folder in which to do the out-of-source build of
\gromacs{}. For example, make it within the folder unpacked from the
source archive, and call it ``build-cmake''. 

For \cmake{}, you can either use the graphical user interface provided
on Windows, or you can use a command line shell with instructions
similar to the UNIX ones above. If you open a shell from within
your IDE (e.g. Microsoft Visual Studio), it will configure the
environment for you, but you might need to tweak this in order to 
get either a 32-bit or 64-bit build environment. The latter provides the
fastest executable. If you use a normal Windows command shell, then
you will need to either set up the environment to find your compilers
and libraries yourself, or run the \verb+vcvarsall.bat+ batch script
provided by MSVC (just like sourcing a bash script under
Unix). 

With the graphical user interface you will be asked about what compilers
to use at the initial configuration stage, and if you use the command line
they can be set in a similar way as under UNIX.
You will probably make your life easier and faster by using the
new facility to download and install \fftw{} automatically. 

For the build, you can either load the generated solutions file into
e.g. Visual Studio, or use the command line with \verb+cmake --build .+ 
so the right tools get used.

\subsection{Building on Cray}

Gromacs builds mostly out of the box on modern Cray machines,
but you want to use static libraries due to the peculiarities with
parallel job execution.

\subsection{Building on BlueGene}

\subsubsection{BlueGene/P}

There is currently no native acceleration on this platform, but the
default plain C kernels will work.

\subsubsection{BlueGene/Q}

There is currently no native acceleration on this platform, but the
default plain C kernels will work. We have accelerated kernels in
progress for this platform, but they are not quite done yet.

Only static linking with XL compilers is supported by \gromacs{}. Dynamic
linking would be supported by the architecture and \gromacs{}, but has no
advantages other than disk space, and is generally discouraged on
BlueGene for performance reasons.

Computation on BlueGene floating-point units is always done in
double-precision. However, single-precision builds of \gromacs{} are
still normal and encouraged since they use cache more efficiently. 
The BlueGene hardware automatically
converts values stored in single precision in memory to double
precision in registers for computation, converts the results back to
single precision correctly, and does so for no additional cost. As
with other platforms, doing the whole computation in double precision
normally shows no improvement in accuracy and costs twice as much time
moving memory around.

You need to arrange for FFTW to be installed correctly, following the
above instructions.

mpicc is used for compiling and linking. This can make it awkward to
attempt to use IBM's optimized BLAS/LAPACK called ESSL. Since mdrun is
the only part of \gromacs{} that should normally run on the compute
nodes, and there is nearly no need for linear algebra support for
mdrun, it is recommended to use the \gromacs{} built-in linear algebra
routines - it is rare for this to be a bottleneck.

\begin{verbatim}
cmake .. -DCMAKE_TOOLCHAIN_FILE=BlueGeneQ-static-XL-C \
         -DCMAKE_PREFIX_PATH=/your/fftw/installation/prefix
make mdrun
make install-mdrun
\end{verbatim}
It is possible to configure and make the remaining \gromacs{} tools
with the compute node toolchain, but as none of those tools are
\mpi{}-aware, this would not normally be useful. Instead, these should
be planned to run on the login node, and a seperate \gromacs{}
installation performed for that using the login node's toolchain.

\subsubsection{Fujitsu PRIMEHPC}

This is the architecture of the K computer, which uses Fujitsu Sparc64viiifx 
chips. Gromacs-4.6 will build with default C kernels on this architecture,
and Gromacs-4.6.2 will add accelerated kernels and a custom toolchain.

\section{Tested platforms}

While it is our best belief that \gromacs{} will build and run pretty
much everywhere, it's important that we tell you where we really know
it works because we've tested it. We do test on Linux, Windows, and
Mac with a range of compilers and libraries for a range of our
configuration options. Every commit in our git source code
repository is currently tested on x86 with gcc versions ranging
from 4.4 through 4.7, and versions 12 and 13 of the Intel compiler.
Under Windows we test both the visual studio compilers and icc,

We test irregularly on BlueGene/L, BlueGene/P, BlueGene/Q, Cray, 
Fujitsu PRIMEHPC, Google nativeclient and other environments. In 
the future we expect ARM to be an important test target too, but this
is currently not included.

Contributions to this section are welcome.

Later we might set up the ability for users to contribute test results
to Jenkins.

\section{Other issues}

The \gromacs{} utility programs often write data files in formats
suitable for the \grace{} plotting tool, but it is straightforward to
use these files in other plotting programs, too.

\end{document}
