\section{Longe Range Electrostatics using PPPM}
\label{sec:pppm}
{\undercons}
The Particle-Particle Particle-Mesh methods of Hockney \& Eastwood
can be applied in {\gromacs} for the treatment of longe range 
electrostatic interactions~\cite{Hockney81,Darden93,Luty95a}. 
The charges of all particles are spread over a grid of dimension
($n_x$,$n_y$,$n_z$) using a weighting function called the
triangle-shaped charged distribution:
\bea
W(\ve{r})	&=	W(x)~W(y)~W(z)	& \\
		&	\frac{3}{4} - \left(\frac{\xi}{h}\right)^2 & |\xi| \leq \frac{h}{2}	\\\nonumber
W(\xi)		=&	\frac{1}{2}\left(\frac{3}{2} - \frac{|\xi|}{h}\right)^2 & \frac{h}{2} < |\xi| < \frac{3h}{2}	\\
		&	0	& {\rm otherwise} \nonumber
\eea
where $\xi$ (is x, y or z) is the distance to a grid point in the corresponding
dimension. Only the 27 closest grid need to be taken into account for each charge.

Then, this charge distribution is fourier transformed using a 3D inverse FFT 
routine.
In fourier space a convolution with function $\hat{G}$ is performed:
\beq
\hat{G}(k)	~=~	\frac{\hat{g}(k)}{\epsilon_0 k^2}
\eeq
where $\hat{g}$ is the fourier transform of the charge spread function
g(r). This yield the long range potential $\hat{\phi}(k)$ on the mesh, which
can be transformed using a forward FFT routine into the real space potential.
Finally the potential and forces are retrieved using interpolation~\cite{Luty95a}.

\subsection{Using PPPM}
Applying the PPPM algorithm in {\gromacs} is rather straightforward using the
following steps.
\begin{enumerate}
\item	Specify the following lines in your {\tt .mdp} file
\begin{verbatim}
eeltype		= PPPM
rshort		= 0.0
rlong		= 0.8
fourier_nx	= 32
fourier_ny	= 32
fourier_nz	= 32
\end{verbatim}
The parameters rshort and rlong correspond to $r_1$ and $r_c$ in the preceeding
section. The fourier\_n* are the number of grid points in each direction.
We recommend to take at most 0.075 nm per gridpoint (e.g. 20 gridpoints for 1.5 nm).
The number of gridpoints should preferably not be prime numbers, but be multiples
of small numbers (2,3,5): 24 gridpoints will be much faster than 23. Documentation
about the FFT algorithms can be found at 
{\tt http://fftw}.

\item 	make a binary topology using {\tt\myindex{grompp}} as usual. 
\item	Before you can perform MD simulations you have to create a 
	$\hat{G}(k)$ function optimized for your grid spacing using the
	{\tt\myindex{mk\_ghat}} program
\item	start {\tt\myindex{mdrun}} passing the {\tt .hat} file created in the
	previous step
\end{enumerate}

It is possible to test the accuracy of your settings using the program 
{\tt\myindex{testlr}} in the {\tt src/gmxlib} dir. This program computes
forces and potentials using PPPM and an Ewald implementation and gives the
absolute and RMS errors in both:
\begin{verbatim}
ERROR ANALYSIS
Error:         Max Abs         RMS            
Force            1.132       0.251
Potential        0.113       0.035
\end{verbatim}
{\bf Note:} these numbers were generated using a grid spacing of
0.058 nm and $r_c$ = 1.0 nm.

You can also see what the accuracy is without optimizing the
$\hat{G}(k)$ function, if you pass the {\tt -ghat} option to {\tt
testlr}. Try it if you think the {\tt mk\_ghat} procedure is a waste
of time.
