\chapter{Technical Details.}
\label{ch:install}
\section{Installation.}
The {\gromacs} code is distributed in SOURCE form by our WWW server at\\
{\wwwpage}\\
On this server you will find all the information you need to 
\myindex{install}
the software, as well as the \myindex{license form} that you have to submit
before you are allowed to download the code. When you have filled in this
license form, a username and password will be sent to you by e-mail
with which you can download the files. The e-mail address you specify
on your license sheet will also be used to send you information on
updates, bug-fixes etc.

For \myindex{commercial use} of the software, please contact us directly:
{\tt gromacs@chem.rug.nl}

\section{Porting gromacs.}
The {\gromacs} system is designed with portability as one major design
goal. However there are a number of things we assume to be present on
the system {\gromacs} is being ported on. We assume the following
features:

\begin{enumerate}
\item 	the UNIX operating system (BSD 4.x or SYSTEM V rev.3 or higher) 
	or UNIX-like libraries
\item 	an ANSI C compiler 
\item	a Fortran-77 compiler or Fortran-90 compiler
	for faster (on some computers) innerloop routines
\item 	If you want to use the graphics, the X-window system version 
	11 Release 4 or higher and the X-lib graphics libraries
\end{enumerate}

These are the requirements of a single processor system. If you want
to compile Gromacs on a multi processor environment there is another
requirement:

\begin{enumerate}
\item Message-passing architecture
\item Ring structure.
\end{enumerate}

One can understand that a message passing architecture also can be
mapped onto a shared memory machine. This implementation is left to
the reader as an exercise in parallel programming. Also the ring
structure can be mapped onto eg. a hypercube.

\subsection{Multi-processor Porting}

In the case you want to run the {\gromacs} software on a
multi-processor machine, you have two options.
\begin{enumerate}
\item	Install MPI or PVM. The {\gromacs} WWW page has some pointers
	to relevant documents.
\item	Write communication routines yourself. 
\end{enumerate}

It may be clear that you will hardly ever need to write the routines
yourself, but if you can't avoid it, here are some clues.
The interface between these routines and the
rest of the gromacs system is described in the file {\tt
\$GMXHOME/src/include/network.h} We will give a short description of the
different routines below.

{\bf extern void gmx\_tx(int pid,void *buf,int bufsize);}\\ 

This routine, when called with the destination processor number, a
pointer to a (byte oriented) transfer buffer, and the size of the
buffer will send the buffer to the indicated processor (in our case
always the neigbouring processor). The routine does {\bf not} wait
until the transfer is finished.

\smallskip

{\bf extern void gmx\_tx\_wait(int pid);}\\
This routine waits until the previous, or the ongoing transmission is finished.

\smallskip


{\bf extern void gmx\_txs(int pid,void *buf,int bufsize);}\\
This routine implements a synchonous send by calling the async routine and then
the wait. It might come in handy to code this differently.

\smallskip

{\bf extern void gmx\_rx(int pid,void *buf,int bufsize);}\\
{\bf extern void gmx\_rx\_wait(int pid);}\\
{\bf extern void gmx\_rxs(int pid,void *buf,int bufsize);}\\
The very same routines for receiving a buffer and waiting untill the reception is finished.

\smallskip

{\bf extern void gmx\_init(int pid,int nprocs);}\\
This routine initializes the different devices needed to do the communication. In general it sets up the communication hardware (if it is accessible) or does an initalize call to the lower level communication subsystem.

\smallskip

{\bf extern void gmx\_stat(FILE *fp,char *msg);}\\
With this routine we can diagnose the ongoing communication. In the current implementation it prints the various contents of the hardware communication  registers of the (\intel) multiprocessor boards to a file.


\section{Environment Variables}
{\gromacs} programs may be influenced by the use of \myindex{environment} 
variables. First of all, the variables set in the \myindex{GMXRC} file
are essential for running and compiling {\gromacs}. Other variables are:
\begin{enumerate}
\item	\myindex{DUMP\_NL}, dump \myindex{neighborlist}. 
	If set to a positive number the {\em entire}
	neighborlist is printed in the log file (may be many megabytes).
	Mainly for debugging purposes, but may allso be handy for
	\myindex{porting} to other platforms.
\item	\myindex{IAMCOOL}, when set prints \myindex{cool quotes}, otherwise
	your {\gromacs} life will be dull and boring.
\item	\myindex{WHERE}, when set print debugging info on line numbers.
\item	\myindex{LOG\_BUFS}. The size of the buffer for file I/O. When set
	to 0, all file I/O will be unbuffered and therefore very slow.
	This can be handy for debugging purposes, because it ensures
	that all files are always totally up-to-date.
\item   \myindex{GMXNPRI}, for SGI systems only. When set, gives the
	default non-degrading priority (npri) for {\tt
	\myindex{mdrun}}, {\tt \myindex{nmrun}}, {\tt
	\myindex{g\_covar}} and {\tt \myindex{g\_nmeig}},
	e.g.\@ setting \verb'setenv GMXNPRI 250' causes all
	runs to be performed at near-lowest priority by default.
\end{enumerate}

Some other environment variables are specific to one program, such as
\myindex{TOTAL} for the {\tt \myindex{do\_shift}} program, and
\myindex{DSPP} for the {\tt \myindex{do\_dssp}} program.

\section{File types.}
\label{sec:fileformats}
This section lists the filetypes used by {\gromacs} along with a short
description.
\begin{table}
\begin{tabularx}{\linewidth}{|r@{\tt.}lccX|}
\dline
\mc{2}{|c}{Default} &      & Default &  \\[-0.1ex]
\mc{1}{|c}{Name} & \mc{1}{c}{Ext.} & Type &  Option & Description \\[-0.1ex]
\hline
\tt   atomtp & \tt atp & Asc & \tt    & Atomtype file used by pdb2gmx \\[-0.1ex]
\tt    eiwit & \tt brk & Asc & \tt -f & Brookhaven data bank file \\[-0.1ex]
\tt   nnnice & \tt dat & Asc & \tt    & Generic data file \\[-0.1ex]
\tt     user & \tt dlg & Asc & \tt    & Dialog Box data for ngmx \\[-0.1ex]
\tt      sam & \tt edi & Asc & \tt    & ED sampling input \\[-0.1ex]
\tt      sam & \tt edo & Asc & \tt    & ED sampling output \\[-0.1ex]
\tt     ener & \tt edr &     & \tt    & Generic energy: \tt edr ene \\[-0.1ex]
\tt     ener & \tt edr & xdr & \tt    & Energy file in portable xdr format \\[-0.1ex]
\tt     ener & \tt ene & Bin & \tt    & Energy file \\[-0.1ex]
\tt    eiwit & \tt ent & Asc & \tt -f & Entry in the protein date bank \\[-0.1ex]
\tt     plot & \tt eps & Asc & \tt    & Encapsulated PostScript (tm) file \\[-0.1ex]
\tt    gtraj & \tt g87 & Asc & \tt    & Gromos-87 ASCII trajectory format \\[-0.1ex]
\tt     conf & \tt g96 & Asc & \tt -c & Coordinate file in Gromos-96 format \\[-0.1ex]
\tt     conf & \tt gro &     & \tt -c & Generic structure: \tt gro g96 pdb tpr tpb tpa \\[-0.1ex]
\tt      out & \tt gro &     & \tt -o & Generic structure: \tt gro g96 pdb \\[-0.1ex]
\tt     conf & \tt gro & Asc & \tt -c & Coordinate file in Gromos-87 format \\[-0.1ex]
\tt    polar & \tt hdb & Asc & \tt    & Hydrogen data base \\[-0.1ex]
\tt   topinc & \tt itp & Asc & \tt    & Include file for topology \\[-0.1ex]
\tt      run & \tt log & Asc & \tt -l & Log file \\[-0.1ex]
\tt       ps & \tt m2p & Asc & \tt    & Input file for mat2ps \\[-0.1ex]
\tt       ss & \tt map & Asc & \tt    & File that maps matrix data to colors \\[-0.1ex]
\tt       ss & \tt mat & Asc & \tt    & Matrix Data file \\[-0.1ex]
\tt   grompp & \tt mdp & Asc & \tt -f & grompp input file with MD parameters \\[-0.1ex]
\tt  hessian & \tt mtx & Bin & \tt -m & Hessian matrix \\[-0.1ex]
\tt    index & \tt ndx & Asc & \tt -n & Index file \\[-0.1ex]
\tt    hello & \tt out & Asc & \tt -o & Generic output file \\[-0.1ex]
\tt    eiwit & \tt pdb & Asc & \tt -f & Protein data bank file \\[-0.1ex]
\tt     pull & \tt pdo & Asc & \tt    & Pull data output \\[-0.1ex]
\tt     pull & \tt ppa & Asc & \tt    & Pull parameters \\[-0.1ex]
\tt  residue & \tt rtp & Asc & \tt    & Residue Type file used by pdb2gmx \\[-0.1ex]
\tt      doc & \tt tex & Asc & \tt -o & LaTeX file \\[-0.1ex]
\tt    topol & \tt top & Asc & \tt -p & Topology file \\[-0.1ex]
\tt    topol & \tt tpa & Asc & \tt -s & Ascii run input file \\[-0.1ex]
\tt    topol & \tt tpb & Bin & \tt -s & Binary run input file \\[-0.1ex]
\tt    topol & \tt tpr &     & \tt -s & Generic run input: \tt tpr tpb tpa \\[-0.1ex]
\tt    topol & \tt tpr &     & \tt -s & Structure+mass(db): \tt tpr tpb tpa gro g96 pdb \\[-0.1ex]
\tt    topol & \tt tpr & xdr & \tt -s & Portable xdr run input file \\[-0.1ex]
\tt     traj & \tt trj & Bin & \tt    & Trajectory file (cpu specific) \\[-0.1ex]
\tt     traj & \tt trr &     & \tt    & Full precision trajectory: \tt trr trj \\[-0.1ex]
\tt     traj & \tt trr & xdr & \tt    & Trajectory in portable xdr format \\[-0.1ex]
\tt     root & \tt xpm & Asc & \tt    & X PixMap compatible matrix file \\[-0.1ex]
\tt     traj & \tt xtc &     & \tt -f & Generic trajectory: \tt xtc trr trj gro g96 pdb \\[-0.1ex]
\tt     traj & \tt xtc & xdr & \tt    & Compressed trajectory (portable xdr format) \\[-0.1ex]
\tt    graph & \tt xvg & Asc & \tt -o & xvgr/xmgr file \\[-0.1ex]
\dline
\end{tabularx}
\caption{The {\gromacs} file types.}
\label{tab:form}
\end{table}


\section{Data types}
Here some of the data types will be described that are necessary for writing
your own analysis programs.

\subsection{Block structure}
The block structure is a datatype that define {\em blocks} of items,
for example molecules can be described as blocks of atom numbers.
In Figure~\ref{fig:block} we have depicted the data structure graphically.
\begin{figure}
\centerline{\psfig{figure=plots/block.eps,width=10cm}}
\caption{Block data structure. Block 4 points to items 7 through 12.}
\label{fig:block}
\end{figure}

Here is, for completeness sake,
the type declaration for the block structure.
\begin{verbatim}
/*
 * Block data structure.
 * This structure is used for charge groups, exclusions
 * molecules and shake-blocks.
 */
typedef struct {
  int multinr[MAXPROC];         /* The indices for the multiprocessor 
                                 * version. For n=0, the blocks run from 0
                                 * upto multinr[index[0]]. The blocks for 
                                 * processor n (n>0) run from 
                                 * index[multinr[n-1]] to index[multinr[n]].
                                 */
  int nr;                       /* The number of blocks                     */
  atom_id *index;               /* Array of indices in a (dim: nr+1)        */
  int nra;                      /* The number of atoms                      */
  atom_id *a;                   /* Array of atom numbers in each group      */
                                /* (dim: nra)                               */
                                /* Block i (0<=i<nr) runs from              */
                                /* index[i] to index[i+1]-1. There will     */
                                /* allways be an extra entry in index       */
                                /* to terminate the table                   */
} t_block;
\end{verbatim}



