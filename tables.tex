\section{\normindex{Tabulated functions}}
In some of the inner loops of {\gromacs} lookup tables are used 
for computation of potential and forces. 
The tables are interpolated using a cubic
spline algorithm. 
There are separate tables for electrostatic, dispersion and repulsion
interactions,
but for the sake of caching performance these have been combined
into a single array. 
The cubic spline interpolation looks like this:
\beq
y(x)~=~\eta y_i + \epsilon y_{i+1} + \frac{h^2}{6}\left[(\eta^3-\eta)y_i^{''} + (\epsilon^3-\epsilon)y_{i+1}^{''}\right]
\label{eqn:spline}
\eeq
where $\epsilon$ = 1-$\eta$, and $y_i$ and $y_i^{''}$ 
are the tabulated values of a function $y(x)$ 
and its second derivative respectively. Furthermore,
\bea
h	&=&	x_{i+1} - x_i	\\
\epsilon&=&	(x - x_i)/h
\eea
so that $0 \le\epsilon < 1$. \eqnref{spline} can be rewritten as
\beq
y(x)~=~ y_i + \epsilon\left(y_{i+1}-y_i-\frac{h^2}{6}\left(2 y_i^{''}+y_{i+1}^{''}\right)\right) + \epsilon^2\left(\frac{h^2}{2}y_i^{''}\right) + \epsilon^3\frac{h^2}{6}\left(y_{i+1}^{''}-y_i^{''}\right)
\eeq
Note that the x-dependence is completely in $\epsilon$. This can abbreviated
to
\beq
y(x)~=~ y_i + \epsilon F_i + \epsilon^2 G_i + \epsilon^3 H_i
\eeq
From this we can calculate the derivative in order to determine the forces:
\beq
\frac{{\rm d}y(x)}{{\rm d}x}~=~\frac{{\rm d}y(x)}{{\rm d}\epsilon}\frac{{\rm d}\epsilon}{{\rm d}x}~=~(F_i + 2\epsilon G_i + 3\epsilon^2 H_i)/h
\eeq
If we store in the table $y_i$, $F_i$, $G_i$ and $H_i$ we need 
a table of length 4n. The number of points per nanometer should be on the
order of 500 to 1000, for accurate representation (relative 
error $<$ 10$^{-4}$ when n = 500 points/nm). The force routines get a 
scaling factor $s$ as a parameter that is equal to the number of points per
nm. (Note that $h$ is $s^{-1}$).

The algorithm goes a little something like this:
\begin{enumerate}
\item	Calc distance vector (\ve{r}$_{ij}$) and distance r$_{ij}$
\item	Multiply r$_{ij}$ by $s$ and truncate to an integer value $n_0$
	to get a table index
\item	Calculate fractional component ($\epsilon$ = $s$r$_{ij} - n_0$) 
	and $\epsilon^2$ 
\item	Do the interpolation to calc the potential $V$ and the the scalar force $f$
\item	Calc the vector force \ve{F} by multiplying $f$ with \ve{r}$_{ij}$
\end{enumerate}

The tables are stored as y$_i$, F$_i$, G$_i$, H$_i$ in the order
coulomb, dispersion, repulsion.
In total there are 12 values in each table entry.
Note that table lookup is significantly {\em slower} than computation
of the most simple Lennard-Jones and Coulomb interaction. However, it
is much faster than the shifted coulomb function used in
conjunction with the PPPM method. Finally it is much easier to modify
a table for the potential (and get a graphical representation of it)
than to modify the inner loops of the MD program.

\subsection{Your own potential function}
You can also use your own \normindex{potential function}s 
without editing the {\gromacs}
code. When you add the following lines in your {\tt .mdp} file:
\begin{verbatim}
electrostatics	= User
rshort          = 1.0
rlong           = 1.0
\end{verbatim}
the MD program will expect to find three files with five columns of table 
lookup data according to \tabref{usertab}. 

\begin{table}
\caption{User specified potential function data. f$^{(n)}$(x) denotes the n$^{th}$ 
derivative of f(x) with respect to x}
\label{tab:usertab}
\centerline{
\begin{tabular}{|l|l|lllll|}
\dline
File name	& Function	&\multicolumn{5}{c|}{Columns}	\\
\hline
rtab.xvg	& Repulsion	&&&&&\\
dtab.xvg	& Dispersion	& x	& f(x)	& -f$^{(1)}$(x) & f$^{(2)}$(x) & -f$^{(3)}$(x)\\
ctab.xvg	& Coulomb	&&&&&\\
\dline
\end{tabular}
}
\end{table}

As an example for the normal dispersion interaction the file would contain:\\
\centerline{\begin{tabular}{lllll}
x &-x$^{-6}$  &-6x$^{-7}$ & -42x$^{-8}$ &-336x$^{-9}$\\
\end{tabular}}\\
The x should run from 0 to r$_c$+0.5, with a spacing of 0.002 nm when you 
run in single precision, or 0.0005 when you run in double precision.
This and other functions contain a singularity at x=0, but since atoms are 
normally not closer to each other than 0.1 nm, the function value at x=0 is
not important.
In this context r$_c$ denotes the single cut-off denoted by the variables
rshort and rlong (see above). These variables should be the same (but need not be
1.0) and consistent with the table data.
The neighboursearching algorithm will search all atom-pairs  within a distance
rlong and compute the interactions using your potential functions.

This mechanism allows the user to use their own preferred programming language,
